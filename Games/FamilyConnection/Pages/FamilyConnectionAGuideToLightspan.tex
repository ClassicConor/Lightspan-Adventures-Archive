\chapter{Family Connection: A Guide to Lightspan}

\begin{figure}[H]
    \centering
    \includegraphics[width=0.5\textwidth]{"./Games/FamilyConnection/Images/FamilyConnectionAGuidetoLightspanCDCover.jpg"}
    \caption{Family Connection: A Guide to Lightspan - CD Cover}
\end{figure}

Family Connection: A Guide to Lightspan was a promotional CD that was produced by The Lightspan Partnership in order to showcase and highlight the importance of integrating technology with education.

The "game" itself, similar to 16 Tales, is split into three sections:

\begin{itemize}
    \item Introducing The Lightspan Partnership
    \item Powerful Learning Programs
    \item Lightspan Adventures
\end{itemize}

Introducing The Lightspan Partnership and Powerful Learning programs are individuals videos on their own.
Lightspan Adventures, on the other hand, is split into three separate videos detailing  the three age groups that Lightspan catered to:

\begin{itemize}
    \item Grades K-2
    \item Grades 3 \& 6
    \item Grades 5 \& 6
\end{itemize}

Each of the Lightspan Adventure series of videos also includes a selection of additional videos highlighting a number of the games which they have on offer.

The combined runtime of all the videos on this CD is just under an hour.

\newpage

\section{List of screenshots:}

\begin{figure}[H]
    \centering
    \includegraphics[width=0.5\textwidth]{"./Games/FamilyConnection/Images/FamilyConnectionAGuidetoLightspanMainMenu.jpg"}
    \caption{Screenshot of the main menu}
\end{figure}

\begin{figure}[H]
    \centering
    \includegraphics[width=0.5\textwidth]{"./Games/FamilyConnection/Images/FamilyConnectionAGuidetoLightspanScreenshot1.png"}
    \caption{Screenshot taken from the video "Powerful Learning Programs"}
\end{figure}

\begin{figure}[H]
    \centering
    \includegraphics[width=0.5\textwidth]{"./Games/FamilyConnection/Images/FamilyConnectionAGuidetoLightspanScreenshot2.png"}
    \caption{Screenshot from one of the Lightspan Adventures videos}
\end{figure}

\clearpage
\newpage

\section{Main Menu audio transcription}

Welcome to the Family Connection.
We hope this guide to Lightspan will be helpful as you and your family discover new learning opportunities.
Simply click on any of the three pictures to start exploring.

\section{Introducing The Lightspan Partnership audio transcription}

Schools today are surrounded by change - the communities we serve, the kinds of challenges kids face, the demands of the working world have all changed tremendously since I first started teaching.

But one thing will never change - we educators will always have to prepare kids to thrive.
If we're going to keep living up to that responsibility, I think we need to change too.
We have to take the best of what we've always done, reinvent it and make it better. We have to find new tools and new ways of doing things.

The Lightspan Partnership is working with teachers, families, and students to meet today's educational challenges.
We're creating powerful tools that motivate and help children to learn - tools that expand learning in schools and homes, tools that will prepare our children for the challenges of tomorrow.

"Watch what happens when we increase the gravity by just five percent."

"Good, now why do you think that happens?"

There's a natural love of learning in every young child. Our challenge is teachers is to find ways each day to engage that part of them, to nourish it, to focus it on meaningful work so that it keeps growing for a lifetime.

Knowledge and skill grow with practice - learning an idea and then applying it, thinking critically, experimenting, solving problems.
You have to give kids lots of ways to do that.
She can't do it all with a chalkboard.

Kids love great characters and interesting stories.
Who doesn't?
But we need to see more of the time kids spent on entertainment spent on learning.
What if the same people who make the best movies and games would help us educators make things that teach?
Then kids would run home from school and keep on learning whether we told them to or not.
Now that would be a big help.

Teachers and families need to work together to meet the needs of every child.
When parents can see what their children are doing every day, it's easier for them to work together as partners.

"So what did you do in school today?"

"I adjusted my angle by 16 degrees, see, so I would have hit the meteor."

"Oh so look out, look out! So you use geometry to figure out how the rock misses the ship?"

"Yeah!"

"Hey Nick, you did a great job on your history project."

"Thanks, you want to check out Family Activities?"

"Okay."

In sixth grade reading, your child will find more subject material in history and current events.

Powerful educational programming created by some of our finest educators.
Motivating tools designed to challenge and inspire.
An expanded partnership between teachers, students, and families.
This is the Lightspan Partnership.

\section{Powerful Learning Programs audio transcription}

Growing up, I was lucky enough to go to an excellent school with wonderful teachers and to get a terrific education.
And I've devoted most of my life to education as a teacher, counselor, principal, and superintendent.
I was academic vice president at Temple University and was appointed by three presidents - Lyndon Johnson, Jimmy Carter, and Bill Clinton - to national advisory councils on education.

When I went to school, and even when I started teaching, the teacher stood in front of a class while students listened and took notes.
And in most places, that's how teaching is still being done because, for the most part, it works pretty well.
But today there's also an opportunity to use technology to present some of these same ideas and information in ways that I never could with a textbook or a chalkboard - ways that really bring concepts to life and get kids excited about learning.

The Lightspan partnership is working with teachers, families, and students using some of this new technology to develop learning tools that are truly powerful.
The main objective of these programs is to help children learn, so the Lightspan programs start with solid educational content.
They are developed to match the textbooks and other curriculum currently used in your child's classroom, so your child's teacher can easily integrate them into his or her current instruction plans.
And they are developed by expert teachers who have years of successful experience in the classroom.

The focus of these programs is on reading, language arts, and mathematics - the basic skills that we all know are so important for future success.
One of the most important aspects of these programs is that they are flexible and can be tailored to meet the needs of your child.

The Lightspan programs not only teach our children, but they also do so in a way that's motivating, so kids will want to spend time learning because these programs are as much fun and are as exciting as all the other activities that kids enjoy.
And by repeating an activity, they'll master the educational concept, and they'll retain it.
And we all know nothing builds self-confidence and motivation more than success - that sense of accomplishment when we did something that we didn't think we could do.

And for families, Lightspan provides a tremendous opportunity to enhance what children are learning in the classroom by continuing that learning at home.
Parents can see what their child is doing in school, and families can work together with this technology in their own living room whenever they have time.
Lightspan also provides lots of ideas and suggestions for fun, rewarding activities that you can do with your child.

As you work with these programs and see for yourself how your child reacts to them, I am confident that you will see these programs are truly powerful and open up tremendous opportunities for the education of our children.

\section{Lightspan Adventures: Grades K-2 audio transcription}

"My children really love Lightspan.
They love the songs, they love the characters.
They liked interacting with the characters and with the the challenge of it."

"The Lightspan programs teach vocabulary, they teach reading, they teach math in a variety of different ways."

"Blue hat, four-sided head, triangular nose, shiny shoes - that's him."

"And it's very nice because the parent and the child can sit down and work on it together, and they're both involved.
And I think it's just a good way for parents and and children to work together and be working on some of being academic but in a more fun type of way."

"We had an open house, and the first thing the students wanted to show the parents was Lightspan."

"Why don't you tell me what's happening?"

"Okay?"

"And it was so wonderful to see the students showing the parents how they could use it, how they were reading, how they were doing math skills.
And it was wonderful."

\section{Additional Videos}

\subsection{Reading/Language Art: Mars Moose}

\subsubsection{Stay \& Play}

In Stay \& Play, children earn Ranger badges by completing a series of language arts activities.

"Would you guys like to become Rangers?"

Students build on their natural curiosity to explore things in their environment, in this case the inside of the Rangers Clubhouse.
Here students are introduced to basic scientific ideas and concepts that expand their vocabulary and build background knowledge important for future science exploration.
This poster assists students in organizing information and charting their progress.

"Cat. Cat."

In this activity students match audio to corresponding words and pictures.
They learn to associate written text to pictures and strengthen their vocabulary.

"See if you can put the pictures in the correct order."

On the bulletin board, students experiment with ordering events.
They see the pattern in the life cycle of a butterfly, and can extend that knowledge to other patterns in their daily lives.

"You got the life cycle of a butterfly!"

Students also get to experiment with scientific equipment, as in this activity in which students view the moon through a telescope.

Like all Mars Moose Adventures, Stay \& Play invites young children to explore, and helps build a sense of wonder about the natural world.

\subsubsection{Cosmic Quest}

"Wow Cosmo, come back here!"

In Cosmic Quest, students must help Mars Moose get his dog Cosmo down from a tree.
As they assist Mars and the shopkeepers of the town, students learn new words and practice processing and classifying visual information in different settings.

"Try the pet shop!"

This adventure builds students' awareness of environmental text as they read signs and labels around the town.

In the pet store, students identify appropriate animals and move them to specific places.
They use visual and written cues, learning to match words with pictures and developing listening and problem-solving skills.

"Put it to the right of the top fish tank."

They identify words like out, left, bottom, and inside.
In the world of Mars Moose, learning new things is a cause for celebration.
At the pizzeria, children build their background knowledge by expanding their vocabulary with words associated with this type of setting, reinforcing the skills they learned at the pet store.
Students' prior knowledge about recycling is enhanced in this third location, where their vocabulary and experience is extended.
As in the pet store and pizzeria, students practice directional words and learn about sentence structure.
All the while, students are working to help get Cosmo out of the tree.
The celebration in City Park when they succeed is part of the Learning Adventure.

"That's more like it!"

\subsubsection{Walkabout}

"I'm Dr. Diggs!"

In this Walkabout Adventure, students help a paleontologist find bones to reconstruct a dinosaur in a natural history museum.

"Wow, some of his bones are gone!"

"They must be in those three rooms over there."

In each of the rooms of the museum, students develop a variety of language arts skills as they discover facts about the natural world.
In the southwest room, for instance, your child will exercise his or her listening skills, sort a variety of clues, and follow directions to identify specific desert animals.
In the process, they are exposed to information within the text of the story.

"The elf owl hoots."

In the Shipwreck room, your child makes active choices, choosing colors, shapes, or letters.

"F"

In this game, students must remember where letters are located as they match uppercase and lowercase letters.
This activity strengthens their ability to organize and recall new information as it is discovered.
As in every Mars Moose Adventure, success brings an exciting reward and stronger language arts skills.
In the rainforest room, students learn new words and basic sentence structure as they follow directions to identify animals.

"The puma is a type of cat."

Students also make the connection between an animated picture and its real-life image as they view animals in their natural environment.

"Wow, Tyrannosaurus is really cool!"

Through these interactive activities, the Walkabout Adventure shows young children how exciting reading can be.

"It was fun! I can't wait to come back! Bye!"

"Bye."

\subsubsection{Liquid Books}

Fables, poems and riddles come to life in Liquid Books, where children are invited to exercise their imaginations in an engaging environment that exposes them to several distinct types of literature.
The goal in Liquid Books is to become Ranger Reader of the Year by reading all three books.
As they meet this challenge, students practice reading and comprehending written information.
In Under the Pepper Tree, students are introduced to poetry and rhyme structure.
They also learn new vocabulary, including specific animal names and book titles.

"I saw a play with a butterfly and wondered why it fluttered by."

Listening skills are developed in an enjoyable way as students hear the musical quality of the rhyme.
In Who Am I?, students learn about animals found in various habitats.
This setting helps develop a stronger natural science vocabulary.

"Yes, I am an octopus."

By solving riddles about various animals, students develop problem-solving and critical thinking skills.
In African Tales, students evaluate clues and interpret text to identify who is telling the story.

"One day, he stopped and stared. What is this?" he said. "It falls down the mountain and sounds like thunder."

Understanding the author's perspective is one of the key steps on the path to becoming a good reader.
In Liquid Books, students begin to develop an appreciation of literature.
They learn that reading is an exciting experience, full of delightful surprises.

\subsubsection{On Screen Activities - General}

"Look, I'm so glad to see that you're reading!"

Learning with the Lightspan programs can be a rewarding experience for both you and your child.
The on-screen activities provide ideas and suggestions for how you can help to enhance your child's learning as you work on the Lightspan programs together.

"Good job, alright. You want to do another one?"

"Yep."

"Alright. Let's do another one."

\subsubsection{On Screen Activities - Stay \& Play}

The life cycle of a butterfly is presented in Stay and Play.
Talk with your child about the sequence of events in nature.
Ask your child to describe sequential events occurring in his or her life.
For example, baby teeth being replaced with permanent teeth or the process of growing up.

\subsubsection{On Screen Activities - Cosmic Quest}

In Cosmic Quest, children are always eager to fly with Mars Moose through the city.
To help your child learn directional words, encourage him or her to fly Mars to the left and to the right, up and down.

\subsubsection{On Screen Activities - Walkabout}
Distinguishing animal sounds builds a foundation for oral comprehension.
Encourage your child to mimic the animal sounds in the southwest room and then to describe those sounds.

"He sounds like he's hooing!"

Have your child compare his or her description to that provided within the text of the story.

\subsubsection{On Screen Activities - Liquid Books}

Having your child retell a story in his or her own words is a good way to review your child's understanding of the story.

'I'm sorry I had to work late. How are you two doing?'

'Fine."

"Fine."

"Oh, I'm so glad to see that you're reading. This looks like an interesting story. Why don't you tell me what's happening?'

'Okay, let me tell you. A long time ago, an animal went out to see the world.'

'Then he screeched, "I'm the best! I can fly far and fast, and I have colorful feathers, and...'"

'"My bill is long!"'

'Well, I wonder who you are.'

'A bird! He's a bird!'

'Well how do you know that?'

'Because he has feathers, a long bill, and he could fly.'

'Well good, you two figured that one out. Nah, why don't you tell me what happens to this bird.

'Okay, let me start. A one day he stopped and stared. 'What is this?' he said. Then he flapped and flapped, and I forget what comes next.'

'He saw something rolling down the mountain, remember?'

'Look, I have an idea - why don't we look it again and find out what the bird discovers?'

'Okay mom?'

'Oh, here it is. Let's take a look.'

'Turn down the volume so then I could read by myself. One day he stopped and stared. 'What is this?' he said. 'It falls down the hill... no, mountain. It falls down the mountain and sounds like thunder.' Now I know, it's a waterfall!'

'That's right. See the bird finally discovered something that's as great as he is. Now why do you think he was so impressed by that waterfall?'

'Hmm... I don't know, being able to fly would be pretty cool.'

'Yes, but the waterfall is big and powerful and beautiful, and that's great too.'

'I really like listening to that story, but I really really liked hearing you two read.'

\subsubsection{Off Screen Activities - General}

The Educational Concepts introduced in Lightspan's programs can become an exciting part of your child's everyday life.
The off-screen activities provide fun, rewarding activities that you and your child can do together to practice and reinforce the skills learned on lifespan.

\subsubsection{Off Screen Activities - Reading}

The more a child reads, the better.
Set aside some time each day for reading, independently or together.

\subsubsection{Off Screen Activities - Learning directional words}

An enjoyable way to help your child learn directional words is by playing a clue game together.
Take turns creating one or more clues that lead to specific objects in your home.
For example, "I am something that is to the left of the television and on the bookshelf. Do you know what I am?"

\subsubsection{Off Screen Activities - The Wonders of Nature}

Place a potato in water and watch it change over time, or plant seeds outside and watch them grow.
The wonders of nature provide a great opportunity for your child to observe a sequence of events.
Recording the changes in a daily journal also provides practice writing.

\subsubsection{Off Screen Activities - Everyday Activities}

Everyday activities can provide a good way to practice reading skills.
While preparing a meal together, ask your child to identify the letters on a milk carton or to read the back of a cereal box.

\subsubsection{Off Screen Activities - Assisting your child with writing and spelling new words}

As your child is learning to write and spell new words, it may be helpful to draw pictures or simply write down words the way he or she thinks they might be spelled.

"Okay gang, book time!"

"We should have gone to the library today. We don't have any new books to read before bed."

"I know, why don't we make up one of our own stories?"

"New great idea! Carrie, why don't you pick out your favorite color pencil? And hey, buddy, you want to start?"

"There once was a boy who loved to eat peanut butter and pickle sandwiches."

"Yeah, yes, we're off to a good start.
Weird, very weird, but a good start. Alright, whose turn is it?
Mine? Huh, alright, here I go.
Um...
After he had made some amazing Triple Decker peanut butter and pickle sandwiches, he liked to climb under the bed and eat them.
Who's next?"

"I know what I want to say, I'm just not sure how to spell it."

"Oh, honey, it doesn't have to be perfect to start with.
Just spell it the way you think it should be spelled, or you can draw pictures if you want.
"Oh, there's a smile.
Wow, what do you got for us?"

"When he wasn't looking, his little sister liked to sneak big bugs into his sandwiches."

"And so they waved goodbye to the Snowman as the train took off from the station, and they never saw him again.
The end."

"That was a good one!"

"That was a great story.
We should put this on the refrigerator tomorrow morning.
Alright now, let's go to bed, huh?
Yes, sir, it's bedtime. Come on, let's pack it up."

---------------------------------------------------------Up to here---------------------------------------------------------
At 27:35

\subsection{Mathematics - The Secret of Googol}



\section{Lightspan Adventures: Grades 3 \& 4 audio transcription}

"Lightspan corresponds perfectly with the curriculum that we're using today.
It goes along with all of the books that we're using, all of the programs that provide us with information on prefixes and suffixes - everything.
Lightspan helps to bridge the connection between the home and the teacher and the child"

"I think Lightspan has encouraged parents and teachers to work together because there's a common thread now, both at school and at home."

"The kids can't wait to get on Lightspan, and sometimes I have to hold them back."

"I think Lightspan has been a giant step, bringing technology into the classrooms."

\section{Lightspan Adventures: Grades 5 \& 6 audio transcription}

"My students really enjoy Lightspan.
It's very appealing to them.
They like the graphics, they like the storyline.
It's very age appropriate for them.
They choose to do it when they have free time, spare time."

"When the Lightspan program stratus goes into the home, with parents and students working together, I see that there's going to be a lot of communication happening between parent and child, as well as problem solving."

"So what did you do in school today?"

"I adjusted my angle by 60°."

"The Lightspan concept of parents and families and teachers and students all working together is a very, very important one."

"You want to check out Family Activities?"

"Okay."

"If students are using the Lightspan program at home with their parents, they'll be actually reading more.
They're also learning about characters in history, they're problem solving, they're learning vocabulary.
So this can only benefit their education."