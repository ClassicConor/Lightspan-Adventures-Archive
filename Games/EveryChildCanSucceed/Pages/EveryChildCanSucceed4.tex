\chapter{Every Child Can Succeed 4}

\begin{figure}[H]
    \centering
    \includegraphics[width=\textwidth/2]{./Games/EveryChildCanSucceed/Images/EveryChildCanSucceed4CD.png}
    \caption{Every Child Can Succeed 4 CD}
\end{figure}


The fourth of the Every Child Can Succeed games published and released by The Lightspan Partnership for the PlayStation 1.

Every Child Can Succeed 4 features three video programs:

\begin{itemize}
    \item Monitoring Student Progress
    \item Effective Instructional Stategies
    \item Learning Essential Skills
\end{itemize}

\clearpage
\newpage

\section{Monitoring Student Progress}

\subsection{Audio Summary}

Monitoring Student Progress investigates how successful schools develop assessments in reading and language arts for each grade.

\subsection{Transcription}

How well are all of your students learning? Are you constantly improving your instruction? Highly successful schools ask themselves tough questions like these. How do they find the answers?

Monitoring student progress. It can be as casual as a principal visiting a classroom, or as formal as a standardized test. At the highly successful schools we visited, monitoring was used to answer important questions about the school. We're going to examine five of those questions.

"Jimmy tells me he's in the top 98\% of his class. That sounds pretty good. Hey, come back here!"

Nancy Ishinaga inherited a school with severe problems. When she became principal of Bennett Q Elementary in Englewood, California, she asked herself, 'Where are the students now', 'what is the level of academic achievement in this school?' And she quickly found out.

"When I first came here, this was in '74, and the first test scores, state test scores came out, our third graders were reading at a three percentile, which is, you really can't get lower than three percentile. And I approached the staff, and I said, 'Three percentile, you know what that means? It means either the kids are all retarded, or something's not happening here, something's wrong with the program.' Now which is it?"

Nancy Ishinaga and her staff used the assessment data to take stock of where their students were. That's the first step in any improvement effort, and it became clear to them that the school's academic program was not working.

"And of course, they all had to admit, and many of them already knew that the program really wasn't doing what it was supposed to do. So we really had very little problems, we all agreed that all kids can learn to read. Me being a Skinnerian behaviorist, I came up with the fact that, 'Hey listen, Skinner taught pigeons how to read, and if he could teach pigeons how to read, certainly you can teach every kid in the school how to read.' And that's where we started from."

You never know where Euclid Principal Esther McShane will turn up next. Breaking up fights on the playground, in the classroom watching a new teacher, playing an educational game with students, or joining in song with a group of fourth graders. She is constantly monitoring classrooms and test scores. One way she uses that data is to create standards of success for each child.

"Academic success will be that every child will show gain every year, so that a child that is on the 50th percentile this year, our goal should be the 60th percentile, that we should not be satisfied at any level. And I do have some children on the 11th percentile, that child must get the support from this school to grow up to 25th percentile, 30th percentile, because I believe that every child is capable of growing, it may take Jose three years to get to the 30th percentile, but my god, he's going to get there."

Lou Yaniw is another principal who uses assessment data to create standards for success. His school is Thorncliff Elementary in Edmonton, Alberta, Canada. Here, all students are tested at the beginning and end of each school year.

"By identifying assessment needs of students early in the school year, we can tailor our curriculum to meet the needs of students. Once we know where the students are when they begin the school year, we provide the appropriate curriculum so that by the end of the year, on an assessment profile, we can demonstrate significant growth."

At the New Suncook School in Lovell, Maine, success is measured by more than just test scores. Principal Gary McDonald believes student involvement in learning is just as important.

"If you have children sitting in a small group, sitting by themselves, and they're actively engaged in learning, that to me means that they have an investment in their learning."

To make sure her students master the school's key learning objectives, Nancy Ishinaga and her staff have created a rigorous curriculum, and they monitor student progress to see if it's working.

"In other words, we have a very well-defined curriculum, and I monitor what happens in the classrooms by periodic tests which the teachers all give, and every teacher in every grade level gives the same tests. They turn in the results to me, and we go over them together to see where strengths and weaknesses might be, where changes in the tests or changes in the curriculum need to be made."

"It doesn't mean that we control everything. It means that within a certain parameter, the teachers have all the freedom they want. So if you go into every class, every teacher teaches differently and different things go on. However, the basic framework is fair so that everybody learns what they need to learn."

What does this activity look like to you? Some people call it chaos but New Suncook Principal Gary McDonald calls it active learning.

"Well, I think from the casual observer certainly you'd begin to question what was happening. There's kids all over the place, you know, there's a lot of noise, a lot of interaction. I think when you get more involved in sitting down and being with the kids and watching them interact with the different centers, then you find that there is a tremendous amount of learning going on."

These students seem involved in their lessons, but can teachers really monitor and assess the learning going on here?

"There's a tremendous amount of assessment that goes on within a day and within the center time. The teacher is interacting with the children individually, they're looking at each of the centers and creating through observation a real assessment of where each of the children are at as they move through their center time. And that's a part that we begin to talk about as staff as to what is the appropriateness of the activity, and it comes back to the question we ask often during the day is what are we doing and why are we doing it?"

The highly successful schools we visited were determined to create success for every child, not just the top students. At Alki, Pat Sander uses classroom monitoring and testing to make sure no students fall through the cracks.

"From my individual perspective as a principal, I want to know where the children are. I try and get into the classrooms every day and monitor what's going on in there."

One way Pat Sander uses the data she collects is to organize students into learning groups, groups that will challenge them to do their best.

"We move the children as they need to be moved. Through the frequent monitoring, if we start to assess that a child's taking a growth spurt in education, then we'll move them to a group that's more appropriate to them. If we find out that maybe they've slowed down a little bit and the group that they're with is moving at a faster pace, then we'll move the child again. In the first semester of the school year, the first 90 days, we moved approximately 10 to 15 percent of the children, so it's not a tracking system once they're put into a group, but we monitor that and watch where they're going."

Alki has a site council, sort of like a board of directors that helps run the school. Four parents of Alki students are on the council, and they are closely involved in the details of student monitoring at their school. Alki school has a diverse student body. The site council wants to be sure there is no achievement gap for minority students.

"So one of the things we're doing in this, you know, in this group of parents and and teachers, is looking at test scores. How are the test scores of minority kids? Is the difference between them increasing or decreasing? So all these things that we're doing at this school, are they making a difference for kids?"

Back at New Suncook School, teacher Linda DeShane is also determined to see success for every child. Her learning centers work because she carefully monitors and knows each student.

"Children are in control of their choices, but the teacher is responsible for finding out the needs of the kids and giving them choices within a range of activities. And when you have children long enough, you can almost tell when a child is choosing something because it's easy, or because it'll get done quickly, or if he's choosing it because it's a challenge. It's been successful for us, I mean, we feel comfortable with it."

"I said your son doesn't seem to listen very well."

At Bennett Q School, Nancy Ishinaga personally tests the academic skills of each new student. She uses this assessment to get parents involved in the child's learning.

"Of course I have to do a lot of counseling with the parents to make them understand why, and generally they understand. Parents understand kids have to attain certain skill levels. In fact, many of them say 'phew, it's about time there's a school that requires certain kinds of learning before the kids are promoted.' In my other school they just passed them on, so we don't do that."

At Alki school, second and third graders are grouped together at what's called the primary level. Fourth and fifth graders at the intermediate level. This is part of the early childhood model pioneered by Louise McKinney.

"We say in our early childhood model schools that if a child doesn't learn in the way we teach him or her, then it's our responsibility to teach the way that child learns."

Diane Tompkinson is a teacher at Alki. She was looking for a way to improve her students' listening skills. A staff development expert from outside the school showed her how to use an instructional technique called Story Painter. The children paint a picture to present their vision of what a story is about.

"The more strategies and methods you have, the more children you reach, and the more interesting it becomes. And I think that's a key thing we're finding nowadays is letting children know there is more than one path to the house, and they're all acceptable and appropriate, and so that you have children taking risks in learning, you have a lot more staff taking risk, you see, there's so many things at work, and it's exciting to try out a wider variety."

At the highly successful schools we visited, the educators were constantly monitoring student progress. Monitoring helped them answer important questions about the school, such as 'How do we define success?' and 'Are any students falling through the cracks?' Successful schools are in a constant mode of self-improvement. They use monitoring to judge their efforts.

"Bottom line is student achievement, and we need to continue to work on it. We need to focus in an area, we need to grow in that area, and as we see growth in that area, we need to identify another area, so that we can continue to build on techniques, strategies, that can equal success for children."

\section{Effective Instructional Stategies}

\subsection{Audio Summary}

Effective Instructional Strategies identifies how focusing staff development on active learning techniques can create learning opportunities for students that include hands-on activities.

\subsection{Transcription}

Ineffective instructional Strategies. Number 17: A strategy for keeping students on task.

"And that happened during the reign of Louis XIV, or was it Louis the 15th? Uh, maybe it was Louis the Twelfth."

You want to see some real strategies? Strategies that work? That's what this program is about. In highly successful schools, the educators work to develop effective instructional strategies - strategies that produce results for their students in their schools. In this program, we'll examine five strategies these schools use.

Successful schools create effective ways to group students for instruction. At the Vann School in Pittsburgh, some students are placed in a program called ELS, Early Learning Skills. ELS acts as a transition period between kindergarten and first grade for students who are not ready to advance.

"If you give them a year to learn the importance of language and work on fine motor skills, listening, attention span, and you also give them a year to build up their confidence, they usually do very very well."

Doris Brevard is a strong supporter of the Early Learning Skills group. She calls it an ounce of prevention.

"We feel here if you can solve the problems, if you can get the children help at a very young age, then that will eliminate more serious problems when they get older. That's why we stress, in the beginning of the reading instruction, that the children master whatever is being taught before they move on."

Like other kids, the ELS students can't keep their hands off the school's computers. But they aren't playing games, they're using educational software that Nadine Dega developed with her husband.

"When you see something that the kids really like and it's valuable to them, and something's working and getting through to them, it makes us all excited, you know. And the extra work and the extra time, you don't mind doing it if you see results. And if you'd see these kids' faces whenever they're working at the computer, it's just exciting."

Alki Principal Pat Sander is also a strong believer in grouping to support instruction. She and her teachers monitor to find each student's individual learning style, then they create instructional groupings based on the children's needs.

"We move the children as they need to be moved. Through the frequent monitoring, if we start to assess that a child's taking a growth spurt in education, then we'll move them to a group that's more appropriate to them. If we find out that maybe they slowed down a little bit and the group that they're with is moving at a faster pace, then we'll move the child again. In the first semester of the school year, the first 90 days, we moved approximately 10 to 15 percent of the children, so it's not a tracking system once they're put into a group, but we monitor that and watch where they're going."

In some schools, children in learning groups are given labels. You don't see the labels, but they exist, and they may stay with that child through every grade. That doesn't happen at Alki.

"The basic philosophy goes back to all children can succeed, and when you place labels on children, expectations for those children accompany those labels. And what we wanted was to make sure that we weren't holding children back because there was a label that was put on them because of some bureaucratic need to rationalize how monies was being spent."

Another important trait of successful schools is appropriate pacing of instruction. The emphasis here is on accelerating the learning for all students, even those who are behind. Without acceleration, these students will never catch up.

This is Thorncliff School in Edmonton, Alberta, Canada. Monica Allman teaches a group of students pulled from regular classes because they were not doing grade level work. Her goal is clear: bring every student up to grade level in every subject.

"A really good example is in math, a number of my students, my grade sixes, started off at a low grade three level. They could barely add, subtract, multiply or divide. I told them that I had no intention of keeping them in grade three if they could prove to me they could master a grade three math curriculum. So we worked at it, and they passed the grade three, eighty percent plus mastery. So I put them in grade four, they thought this was wonderful, they did that, they mastered that. I put them in grade five, they thought that was wonderful, they mastered the grade five, and I put some of them into grade six. They were then pestering me saying, 'Would you put us into grade seven?' because a number of my students should be in grade seven, one should be in grade eight. And I said, 'Yes, I'm not here to hold you back, I'm here to encourage you and make you the best that you can be, and to bring you along as far as I can.'"

In these classes, the pacing of the instruction is geared to help all children succeed by emphasizing their strengths.

"I want them to be aware that they have weaknesses because I believe everybody that is a learner has some weaknesses somewhere, but that you'll use your strengths to overcome those weaknesses. And I'm very, very, very demanding in that way. I want them to be the very best that they can be because they're our future."

At Alki Elementary, Principal Pat Sanders' approach is to teach as though every child were gifted.

"At Alki, what we've tried to do is no longer remediate children, but accelerate them. We believe that the kinds of education that people have supported for gifted children over the last five or ten years are the kinds of instructional approaches and the kinds of education that all children should have."

Alki students who need extra help in some areas will get it, but they are still held to the same standards as other students.

"I require that teachers frequently monitor student achievement, and that that student achievement comes to the office, and that children attain an 80\% mastery in reading, language arts, and mathematics objectives. It's not to say that on the first time through that child may do that and that we'll hold them in that place, but for five months later, reteaching and through some of the other activities, I would hope that retesting takes place and that we can show that that child has been able to, in fact, move up on the mastery level."

Meet another principal who sees a potential for excellence in every student: Beacon Hill's Gary Tubbs. To find that potential, he expects teachers to use techniques that accelerate learning.

"What we have found in working with gifted educators or educators who have worked with kids who have been tested to be in the gifted range, is that these techniques work with all children. There's a lot of research to show you can actually raise intelligent quotients by that kind of teaching, so all children are gifted if we just haven't all necessarily discovered their giftedness in all areas, in the areas that they excel."

"Mrs. Crenshaw, I think this time your students have gone a bit too far."

Highly successful schools demand that students be involved in their own education. Teachers at these schools use instructional approaches that promote active and enriched learning.

At Alki School, second and third graders are grouped together at what's called the primary level. Fourth and fifth graders at the intermediate level. This is part of the early childhood model pioneered by Louise McKinney.

"We say in our early childhood model schools that if a child doesn't learn in the way we teach him or her, then it's our responsibility to teach the way that child learns. We also know that all of our children are not analytics, and that analytics might do well working with workbooks, but the majority of the children in our schools who do not learn well tend to be very global learners, and so they learn better when they have context and meaning."

To provide context and meaning for their students, Alki teachers Diane Tompkinson and Lori Eba are about to lead a field trip to the beach. The trip will complement the students' classroom lessons on sea life. The first step is getting the students ready for the trip. The next day, the students enjoy some hands-on learning. The teachers hope this active experience will motivate kids to learn more about sea life.

Successful schools use a variety of approaches and techniques. The same students who went to the beach now put together their own books about the field trip. They'll write poetry, create math story problems around sea life, and do art projects about the creatures they saw.

"It's a way to bring real life experience so that the child can then take that information back and write about thingg. They become totally engrossed in what it is that they're doing, and the motivation is there, the feeling that 'God we want to learn all we can about it', and it just seems to create a feeling in them that they can do it and they want to do it, and it has meaning to them. And I think that's the whole point of what we're doing for kids is to create meaning for them."

Teachers in successful schools are like chameleons - they adapt. They know how to customize instructional materials and methods for their classrooms. The driving force is student needs, and the adaptations help reinforce the school's instructional mission.

At the Rice School in Dallas, science teacher Georgia Beatty has a terrific imagination. She uses common materials found at home to create original and active lessons for her students.

"If you're not willing to take in a class after school or go the extra mile or search for something beyond the textbook, put the textbook on the shelf and do something creative, then this isn't the place for you. We're not a textbook-oriented school where we just simply you know, start at chapter one and proceed through chapter 14. We're very different. As long as the kids benefit and that's the end product, it's okay."

Like a chameleon, Georgia Beatty adapts to her situation. When her students need something different or when the school policy changes, she changes too.

"And even when you get to a point where you think it's polished, something changes, and then you wind up starting all over. Every time they change the textbooks or the learner standards that we get from the state, I feel like a brand new teacher again because I'm having to dig and look and try to make something fit to the particular needs of the kids."

Welcome to the New Suncook School in Lovell, Maine. Give them half a chance and every educator here will tell you about the school's mission: creating lifelong independent learners. To realize that mission, they customize their curriculum and their teaching methods.

Rhonda Poliquin is a fourth grade teacher at New Suncook. She calls herself a guide, a person who helps students learn by teaching them how to find information and solve problems. She has adapted her entire curriculum to meet that goal.

"I think we've worked towards kids becoming independent learners, so that you won't see as much a teacher standing in front of the class and lecturing to kids, because we want kids to find out how to get that information themselves, and if there's something they want to find out about, giving them the tools and the skills that they need to be able to find out that information. Really the goal is to get kids to look at problems in different ways, that you can look at it through an example as a scientist or as a historian, and that using those perspectives, that will help you to solve problems, which is part of becoming an independent learner."

"Introducing the Chi Square Marching Band!"

At successful schools, the educators are always searching for new ways to steal time from the school day for reinforcement in language and math. For example, at Rice School in Dallas, the students begin every day by reading to the accompaniment of classical music.

At Euclid, Principal Esther McShane often reads to her students during the lunch hour. And at Beacon Hill School, students practice their computer skills by writing poetry.

"Welcome aboard!"

Today's field trip at New Suncook is another good example of how schools steal time for language, and how teachers at these schools integrate language into other lessons.

Benny Miller is a song about a caterpillar who dreams of being a butterfly, but after his transformation he is captured and becomes part of a boy's collection.

"We've been studying entomology for quite a few weeks now, and one of the concepts that we really want to get across to children is the idea that nature is very fragile and we need to protect it. This is a literacy activity and we like to bring literacy experiences into the field, and it also might inspire some of their own writing when they get back to the class."

Other students on this field trip will search for insects in the field, and in the stream. After some exploring, Susan Steller's class will design their own insects, complete with camouflage. By combining entomology with language and art, the New Suncook teachers help strengthen basic communication skills. Plus, it's just plain fun.

"I think they're having a wonderful time. I think they're learning a lot and having fun doing it."

At the highly successful schools we visited, the educators develop instructional strategies that work for their students. They group for instruction, they carefully pace instruction to accelerate students, they create active and enriched lessons, they customize instruction to their students' needs, and they steal time for language and math. These strategies are designed to help every child succeed.

\section{Learning Essential Skills}

\subsection{Audio Summary}

Learning Essential Skills emphasizes why all children must learn essential skills in maths and language arts before proceeding to the next grade.

\subsection{Transcription}

"We believe that it is our job to educate the kids, to make them literate. It's not the home's job, it's our job. If the home helps, it's fine, but if they don't help, we still have to educate and make the kids literate. Kids who come to Bennett Q, if they're here for any length of time, you can be sure that they can read, they can write, they can compute, and they can speak well, because this is our job, and this is what we plan to do, and this is what we do."

At the highly effective schools we visited, we found principals and teachers who worked together toward a common goal: high levels of attainment for all students. At the Euclid School in Los Angeles, Principal Esther McShane and her staff focus on student learning.

"We're only here for one reason, and that is to educate children, and I'm responsible for the education of every single child that enters this gate. They have entered my spare of life, so I am responsible for it, and I need to provide the climate, the environment where learning can take place."

"I want my children to believe in themselves, and to believe that they can accomplish anything they choose to, and I realize that my job is to teach them strategies for accomplishing these things."

At the Bennett Q School, just outside Los Angeles, Principal Nancy Ishinaga has a similar partnership with her faculty.

"A school should be a place where kids come to learn. A school should be a place where a kid succeeds and where kid is valued, and we have a specific goal. So when a kid leaves us, we want him to have certain knowledge, certain skills. We want him to be able to go to the next step and be successful."

"You get a good principal who knows what she's talking about and knows curriculum and knows her people, then anyone can do it, and not no one should take excuses. We don't have excuses for why children can't learn."

"First you have to get their attention."

When principles and teachers focus on learning for all students, the first priority is maximizing the time available for learning to take place. Discipline is one important element in providing an environment for learning.

At the Vann School in Pittsburgh, Principal Doris Brevard feels strongly about her role.

"The principal must provide the atmosphere, and then it's the teacher's job to teach. But if that teacher can go into his or her classroom and teach 35 minutes of the 40-minute period, then something will be accomplished. But the principal must see to it that the discipline throughout the building is such that the teacher can teach the 35 minutes."

Cathy Gallagher teaches fourth grade at Vann.

"I know that when I go in my room, I will have an atmosphere in which I can teach, and I think one of the things I always remember when I first came to Vann, she said to me, 'You can teach whatever style you like, your style is your style, and as long as you're teaching and children are learning, that's all I ask of you.' And she also told me that when I go in my room to teach, I will have an atmosphere in which I can teach, and if I don't have that atmosphere, then that's her job to provide it for me, and I have, it's always been provided for me."

Another way to maximize learning time is to move swiftly from one class activity to the next, as in Dana Brooks' class at the Rice School in Dallas. Whether it's a teacher-centered school such as the Rice School in Dallas, or a more student-centered school such as New Suncook in Maine, transitions take very little time.

"If we had to yell to them or say 'okay it's time to clean up', they wouldn't hear us, so by clapping a pattern it brings their attention to what we're doing, and the hands on the head is they can't be touching anything while they're listening, so they have a better listening time when they have their hands on their head."

Maximizing learning time also means that every possible minute is devoted to teaching and learning. At the Euclid School in Los Angeles, that can even mean having the principal read stories to the students at lunch.

For a principal like Esther McShane, maximizing learning is a high priority. Effective use of time is one of the things she looks for in the classroom.

"What I look for in a classroom is to see time on task. Why is it that the child is engaged in that activity? Is that activity challenging? Does it raise another level of awareness for a child? Are children exchanging ideas with one another?"

"All I'm saying is that if you'd been a little more proficient in math, this would have never happened."

At the Vann School in Pittsburgh, their goal is for all students to master the academic content. The curriculum is structured to help students achieve that goal.

"The primary level is the foundation for the educational system. If they start out, and they're passed through, and they don't have the basics for reading, they don't have the basics for math, they never get it. It has to be achieved in the primary level, and we firmly believe that, and we work for it."

Vann Principal Doris Brevard has the same conviction.

"We feel here if you can solve the problems, if you can get the children help at a very young age, then that will eliminate the more serious problems when they get older. That's why we stress, in the beginning of the reading instruction, that the children master whatever is being taught before they move on."

At the Bennett Q School in Los Angeles, Nancy Ishinaga also stresses the value of a structured, sequenced curriculum.

"We have worked out a system where everybody goes through, everybody is taught specific things that make them a successful student, and it's what every teacher does."

Howard Rothenberg teaches math at Bennett Q.

"If you don't have the foundations from first, second, and third grade, you can't begin fourth grade where you need to start, and you have to go back and remediate, and you're losing time and the children, and that's what happens in most inner city schools, the children are years behind because every year has to go back and repeat what they did not learn previously. Well, we do not have that problem here, for the most part. We have a very structured curriculum at each grade level, and every teacher knows what he or she is responsible to teach, so that the following years' teacher can take up at that spot, and the kids are pretty much at grade level, and you're able to start there and work from that point."

"One of the very important things about our school is that our kids are held accountable for their learning. In other words, we have expectations by grade level for every kid, and the kids of the parents know from the very very beginning that if they are in kindergarten, these are the things they have to master before the year is over, and if they don't, they probably will not go on to the next grade. Parents understand, kids have to attain certain skill levels. That's not new, in fact, many of them say 'phew, it's about time there's a school that requires certain kinds of learning before the kids are promoted.' In my other school they just passed them on, so we don't do that."

At the Alki School in Seattle, the curriculum is more flexible. Principal Pat Sander is just as concerned about success for all students, but she has her own approach for reaching that goal.

"We believe that the kinds of education that people have supported for gifted children over the last five or ten years are the kinds of instructional approaches and the kinds of education that all children should have. So that instead of saying you have to crawl before you walk, some children may in fact walk and go back and learn to crawl later."

Fred Pruitt teaches at Alki. His class is studying the rainforest in science. That content is also used to teach spelling and build vocabulary, to teach punctuation, and also math. Whether a school uses a highly structured, sequenced curriculum or one that is more flexible, in highly effective schools the emphasis is always on mastering academic content.

"I have a job to do. I'm going to be with these children for one year, and my goal is to help them want to learn and to learn as much as they can in that one year, because they're not going to get that chance again."

"It's a good thing this school provides a lot of individual instruction. At least I'm falling behind at my own pace."

To make sure students are learning, highly effective schools monitor their progress and provide special help when necessary. At the Thorncliffe School in Edmonton, Alberta, Canada, Principal Lou Yaniw has goals for each student.

"We have identified for each individual a learning profile, so that over the course of the time that the student is in the school, whether it be one year or seven years, kindergarten through grade seven, we would be able to track their progress. The goal is that every child is capable of high levels of success, and that what we wish to see for each individual child on a year-to-year basis is growth from September through June, and that that growth profile continue to raise as they continue on at our school. Another one of our goal statements and belief statements is that if children can't learn the way we teach, then we must teach the way they learn. If a child in the classroom is having difficulty learning, the teachers have committed themselves to working with that child."

At Thorncliffe, helping students learn can mean teaching behavioral skills in classes such as this one taught by Pat Kostouros.

"The purpose in having the kids come into the skills program is that the teachers have checklisted certain behaviors in the classroom that they notice that they're lacking in those particular skills and so need practice in those areas. If you're not able to behave and you're distracting others and you're distracted yourself, then you're really not picking up the academic things that the teachers are offering. It fits into the basic philosophy of the whole school that learning is important, that practice is important to be the best that you can be. If you're lacking some skills, we'll offer them to you so you can better your life."

At Alki School in Seattle, Pat Sander uses a mainstream approach to ensure student success. One of her first moves as principal was to get the non-English students into regular classrooms as quickly as possible.

"We've tried to do an immersion technique, and instead of having those children pulled out where they're in a room with other children that are speaking the conglomerations of languages and dialects, we have looked at instructional placements that are appropriate to them and mainstream them in with the rest of the children. And so consequently the English language then seems to take off much more quickly with those children."

The same approach applies to children with learning disabilities or physical handicaps. They spend a small part of the day with special ed teacher Tove Andvik, or in individual instruction, but the majority of their time is spent in regular classrooms.

"They need to be in class. They will learn more by learning from their peers and from their teachers, and by learning how to use the textbooks and how to listen and how to take notes, those are all more useful skills for them as they go on in school than pulling them out and working one-on-one."

"The basic philosophy goes back to all children can succeed, and what we wanted was to make sure that we weren't holding children back because there was a label that was put on them because of some bureaucratic need to rationalize how monies was being spent."

It takes a combination of factors to make a successful school, but an emphasis on essential learning skills is one of the most important. The schools in this program maximize the time available for learning, they design a curriculum that stresses mastery of academic content, and they provide support for students who are not succeeding.

"I think that a child who doesn't feel good about himself can not feel good about the world, so we try to make every one of our kids successful at least academically, so whatever may be wrong with another part of his life, at least this part is good, and we hope that for many of them this will carry them through, and I think it will. I really do believe that it will."