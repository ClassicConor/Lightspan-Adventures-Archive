\chapter{Write Away - Overview}

Write Away is a series of education video games developed by The Lightspan Partnership starting in 1996. Each of the 10 CD's consists of 10 separate videos, with the video length combined being less than 30 minutes into total.

Similar to the 16 Tales video games, the entirety of the Write Away series is a series of videos, with the user having no control over the video content. For each disk, the content always includes an introduction video, seven to eight story videos, and a conclusion video.

The introduction video always contains the same content: the 'Write Away' theme song, and a brief introduction to the characters. The same series of actors and actresses are used in each of the videos, but they switch between one another in each of the videos. That is to say, a single actor may appear in multiple of the games, but not all of them. The seven to eight story videos are always different, and there is no pattern to the content of these videos. The conclusion video always contains the same content: the 'Write Away' theme song, and a brief conclusion to the characters.

In the conclusion videos of each of the games, the actors encourage the student watching to write their own stories and send them to:

\begin{center}
    Write Away Information\\
    The Imagination Machine\\
    (714) 771-2499
\end{center}

Lightspan claims that it actually used the stories that they were sent to them in order to create new videos. This is indicated when, viewing all of the video programs, the title of the video program is accompanied by the name of the author of the story. In addition, the image which is used to accompany each of the eight story videos usually either a picture of the author of the story, that is to say, an image of the child who wrote the story, or a picture of a scene in the short film (presumably when they couldn't get a picture of the child). Although I have no reason to not believe this to be the case, credits for many of the videos include the role of 'Writer', which indicates that these people were, in fact, the writers of the program, or perhaps they had a more limited writing capacity, such as scipting the introductions for each of the stories.

The documentation for this game was slightly more difficult that the documentation for the 16 Tales series. The main reason for this was that, in comparison to the 16 Tales series, where the content for all these videos was already online and available to view on YouTube to grab the transcripts, this was not the case for the Write Away series. The YouTube Channel \href{https://www.youtube.com/@longplayarchive}{LongplayArchive} had already uploaded the first four games of the series onto YouTube in high quality, but not the final six.

To find the remainder of the six video on YouTube for easy transcription, I had to find the much smaller YouTube channel, \href{https://www.youtube.com/@WizardmanTonight}{WizardMan} who had the remaining videos uploaded onto his YouTube channel. WizardMan seems to have uploaded not only the entire Write Away series, but also the vast majority of other Lightspan games, as well as PlayStation 1 games in general. Not all the games he's recorded seem to be complete, but for the Write Away series, as well as others, his videos include everything. All of the Write Away videos on his YouTube channel have an extremely low number of views, with, Write Away 5 having only 8 views at the time of writing this overview (31st March 2024).

Similar as well to the 16 Tales series, the degree to which the content of this series could be considered educational is questionable. The content of the videos is not particularly educational, and the only educational aspect of the videos is the encouragement to write your own stories at the conclusion of each of the videos. The main focus of these 'games' was to encourage children to write their own stories, and to be creative in their writing.

With respect to being able to play these games online, of course \href{https://vimm.net}{vimm.net} remains the best place to emulate all of the games. However, \href{https://archive.org}{archive.org} is also a good place to have a look at playing the games, hosting all of the games except Write Away 8.

These games, like 16 Tales, can be more accurately described as interactive video programs, as the user has no control over the video content. The user can only pause, rewind, and fast forward the video content, and select which video programs to watch.

I was not able to find any information regarding merchandise for the Write Away series. I suspect this is the case as a result of the Write Away series not being directly developed by The Lightspan Partnership, but rather by the children who sent in their stories.

Unlike the 16 Tales series, which I personally felt to be quite dry, I found all of the Write Away video programs to have a certain charm to them, even if I wasn't the target audience, and even if I personally wouldn't watch them were it not for documentation purposes. The fact that the actors and actress are made to perform scripts written by children, and wth the limited resources available to them, I found to be quite endearing. A nice texture to some of the Write Away videos is the 'Tales from the Script' episodes, which appear sporadically throughout the series. The 'Tales from the Script' videos, in production and performance, essentialy no different to the other videos. However, they are all horror-themed, and the videos are all introduced by Paul, one of the recurring actors in the series.

There were other interesting choices that were made for the series also. Write Away 7 has a newspaper theme, where every story was introduced as a newspaper bulletin. The final story in Write Away 10, 'The Sea Monster in Elephant Butte', was a thinly veiled parody of a Twilight Zone episode, with the stories being introduced and concluded as 'The Scary Zone'.

The development of this series, like the rest of the Lightspan games, remains, to me, one of the strangest stories of the PlayStation 1 era, and perhaps gaming in general. Nearer the end of the series, I did find myself finding fun and interesting moments, however this was not my primary feeling towards the series.

There are a few final points which I want to make on the transcriptions and documentation of the Write Away series. The first being the significant amount of time that it took me to do documentation for, in comparison to other Lightspan games. I only only had to get accurate transcripts for the videos, but also had to label each person who was talking, like a screenplay. Many times, particularly when there was narration, it was difficult to know who was doing the narration - for these instances, I simply wrote 'NARRATOR' for the speaker. In other instances, it was unclear whether actors were still playing their characters, or playing themselves. Not only this, but each Write Away game having up to 10 videos meant the time editing the text was significant.

Secondly, there were many times when the characters were speaking in which I could not accurately understand what it was that was being said. In these instances, I either provided a guess, or used [square brackets] to indicate that I was unsure of what was being said. In some cases, I used "double quotation marks", and in other cases I used 'single quotation marks', creating an inconsistent style for transcription. In this respect, although I put in much more effort to Write Away than many of the other series, I am still, overall, not completely satisfied with the quality of the documentation, and at some point in the far flung future, it may make sense for myself or someone else to correct the information written.

A number of oddities in the series also made the Write Away series on of the more unusual to document. The scrolling credits at the end of each of the videos seemed to end abruptly, with respect to the music not seeming to end on a satisfactory note. As well as this, the credits for Write Away 9 and 10 seemed to end even earlier, with key roles such as Editors not seeming to be credited. The music used in the outro for each of the conclusion videos changes from Write Away 4, which seems to use a more extended version of the Write Away theme song.

There seems to have been an oversight to many of the introduction videos for a number of the games. These videos possess no real content, and simply act as a short introduction to the characters. In the early on videos, there's a shot of all the actors appearing on the stage, before they're introduced individually. However, in a couple of later videos with this same introductory sequence, the actors which appear on the stage as a group shot, and the actors are introduced individually are not the same, an issue which is only properly fixed in Write Away 7. There's another funny mistake in Write Away 6, with the story 'Why People Don't Go to the Cemetary at Night', where the camera zooms in on Paul, who plays a zombie. As the camera zooms in, it begins to show less and less of Paul's body, which isn't moving because he has just died. However, just before the camera cuts, you can see Paul's begin to push himself off the floor, as if he's about to get up and walk away. There many have been other interesting mistakes in the series, but these are the ones which I noticed.

A final note: looking through the credits, I attempted find the names of the actors which appeared in the series. Although I wasn't able to find much for many of them, Brian Kwan, who appeared in some of the earlier series, (and who, in my opinion, was one of the better actors for the series), appears to have now found work for The Walt Disney Company as an executive director and senior producer, particularly on the Disney+ side of the business. I wonder if he ever looks back on his brief time working for Lightspan, and if he does, what his thoughts are on it.