\chapter{Story Lane Theater 3}

% \begin{figure}[H]
%     \centering
%     \includegraphics[width=\textwidth/2]{./Games/WriteAway/Images/WriteAway2CD.jpg}
%     \caption{Write Away 2 CD}
% \end{figure}

The third of the Story Lane Theatre games published and released by The Lightspan Partnership for the PlayStation 1.

Story Lane Theatre 3 features two video programs:

\begin{itemize}
    \item Koi and the Kola Nuts
    \item Mose the Fireman
\end{itemize}

\clearpage
\newpage

\section{Koi and the Kola Nuts}

\subsection{Audio Summary}

Koi and the Kola Nuts is a humorous retelling of a classic African folktale where proud Koi sets off on an adventure to find respect. The story is told by Whoopi Goldberg, with music by Herbie Hancock. Backstage Pass visits with the illustrator, Reynold Ruffins.

\subsection{Transcription}

Come on in, settle back. It's curtain time at the Story Lane Theater. You've never seen a stage like this one. Anything can happen here.

Today we're going to take a journey with the son of a chief in the African folktale called "Koi and the Cola Nuts." But before we set off, we're going to visit with the artist who did the pictures you'll see in the show. His name is Reynold Ruffins and he lives on Long Island, not far from New York City. He's been drawing almost his whole life, and today he makes art for things like books and magazines. He's going to tell us what it's like to be an artist and how he made the pictures for "Koi and the Cola Nuts."

"I've always lived in New York. I worked in Manhattan, but the last few years I have come out to Long Island. It's much quieter, and for the first time, I had my own studio arranged pretty much the way I would like it to be, and I'm surrounded by the things that I like — doodads and knickknacks and little things that I can't throw out. I love masks; they're just beautiful, beautiful art.

"The greatest thing about being an artist is, I think, being able to see light on a leaf, water in a glass, a sense of the time of day, the quality of light, the way shadows fall in the corner of a room, just somehow being observant. It enters through your eyes, gets knocked around in your head, and is screened through all the experiences you've had. But it's getting out, and down through your body, and out your hands that makes it real.

"When I was first asked to do 'Koi and the Cola Nuts,' I thought, what a wonderful idea. It takes place in an interesting setting in Africa. When I was a child, I never saw any depiction of African-American children, so I welcomed the opportunity to create images of life in Africa and to share that image with all children. Africa, being a very big and diverse place, I went to the library and got many books about African architecture, animals, foliage that you might find there, that I referred to, and then changed and designed in order to make what I thought was a more interesting visual.

"I made several sketches for Koi, and I wanted him to be both young and distinguished, a little hoity, and able to still keep his personality when times didn't go well for him. Ideas can come from a wide variety of places, and he, in particular, I think, grew out of my interest in mask, African masks. I tried to stylize his features — the large eyes, long nose, so that they would be a connection to the masks.

"There are so many things to think about when doing a large piece of work like this. You have to plan it out, you have to make sketches, and I make a drawing first. What I want to do is transfer this finished sketch to this piece of brown paper. I'll do that by using a little tracing paper so that the line will be transferred from the tissue onto the colored paper that I'm working on, in this case, it'll be a nice brown-colored paper. I'll put in the details. You know, that's not realistic, but it represents hair texture. I can add color where that's necessary. This is a pattern or design often seen in African art, as kind of black and white triangles. Let's give you a little character, Koi - eyes reveal so much.

"This is a scene of Koi talking to the ant that he's met along the way. I have his eyes focused on the ant because he's small; he's got to get close, and he has something to say. I want him to look concerned and interested. Just darken some areas to give features and expression. Look pretty good now.

"Now, what I do is cut this out very carefully, watching the line. There we are, just about done. You're coming to life, Koi.

"Now that I finished cutting it out, I added it to other elements of the story, and using these cutouts, I'm able to arrange and rearrange figures in different scenes. Now this is a night scene, so I made this the moon - a big moon, to give it scale. Palm tree and Koi, he's very concerned, talking to the ants. How about you're so sad, Koi? We'll have a few tears. That's a nice composition for that. How's that, Koi? You like that? You'll feel better later, you'll see it all comes out okay in the end. I know.

"Like many other fairy tales, 'Koi and the Cola Nuts' is a test. Someone is being tested for their ability. Koi learned that if you're helpful to others, if you're kind to others without being asked to be, if you do good because you see that it's necessary, then good will come back to you, and that being good is its own reward, and somehow if you're good, good will return to you."

And now, let's watch 'Koi and the Cola Nuts.'

Come. Sit down. I will tell you a story that comes from Africa. It is dawn in the west of Africa and the sun is a big orange-red ball slowly rising over the horizon. Feel the warm breeze that blows. The rains have just come and the earth is full of promise. Flowers bloom and banana trees bear fruit, and the yams and rice grow well in the fields. Smell the beautiful fragrances. Look! The animals gather at the river. Far in the distance, there is a village. The smoke from the cooking fires making a trail in the sky. Listen\dots You can hear the sound of the drums. The drums spread the news that the chief of a village has died. All of the people and all of the animals from miles around hear the message of the drums: Chief Sedaka is dead.

A Palavar, that's a meeting, is called, and all the elders of the village gather to decide what to do. They decide that the wisest man of the village will divide the chief's possessions among the chief's sons. So, the wise man counts out to each of the sons. So many goats, so many cows, so many tusks of ivory, and so many pieces of gold. The wise man finishes. But here comes Koi, the chief's youngest son. He's been hunting and no one fetched him when the wise man divided his father's possessions. There's nothing left for Koi, not even a tiny chicken bone.

Since this wise man is wisest when it comes to avoiding work, he does not bother to redivide the chief's possessions, no. It's too much work. The wise man looks around and he sees a small, sickly kola tree.

"Ah, Koi," says the wise man, "and for you, we have a kola tree over there."

"What do you say, man?" cries Koi. "My father, the chief dies, and you only give me this kola tree?" Koi is a proud young man. He grows angry, very angry. He cannot abide the slight of the wise man. "I will go to a land where I am treated like the son of a chief," Koi says. Koi picks all the kola nuts from the tree and wraps them into a mat. He ties the mat into a [kinja], swings the load onto his back, and leaves the village. "I will never return here," Koi tells the people of his village. "You do not know how to treat the son of a chief."

Koi walks for nine days with the [kinja] on his back. The [kinja] is a heavy load, but the jungle makes good music for him to walk by. The birds sing and the chimpanzees play hide and seek amid the elephant grass. Koi watches the zebras galloping across the veld. He walks gently past a sleeping lion and lioness, careful not to awaken them. Koi comes to the foot of a mountain. It's a big mountain with a peak hidden by clouds. He climbs the mountain with great difficulty, the [kinja] getting heavier with each step. Finally, he arrives at the top. He sees the lush valley beneath him. There's a wide, silvery river curling through the valley.

"This will be a good place," says Koi. "Perhaps the people there will know how to treat the son of a chief." Koi makes his way down the slope. He sees a large snake, but the snake does not see Koi. The snake looks left, the snake looks right, and then he slithers along.

"What are you looking for, friend snake?" says Koi.

"Oh, it's terrible," says the snake. "My mother is ill and she needs kola medicine. I must find some kola nuts to make the medicine that will make her well."

"Look no further," says Koi. "I have kola nuts you may have. Take them, make the medicine that will make your mother well again."

"Oh, you've saved my mother's life," said the snake. "Thank you so very much."

"It's nothing," says Koi. "Go quickly now. Your mother needs the kola medicine." Koi walks down the mountain, happy that he can help friend snake.

Koi sees an army of ants marching in an endless column as wide as a zebra stripe. Koi steps aside to let the ants pass. And as the column passes, he hears a tiny high-pitched voice.

"Do you know where we can find a kola tree?" Asked the leader of the ants. "We made a most unfortunate mistake. We ate the forest devil's kola nuts, a whole basket full, and we must replace them, or else the forest devil will trample us with his gigantic feet."

"How many do you need?" asked Koi.

"He says he wants as many kola nuts as he has fingers and toes. Let's see, I have six feet and four toes on each foot."

"The forest devil is a person," says Koi. "He has ten fingers and ten toes. You need twenty kola nuts."

Koi takes twenty kola nuts from his [kinja] and gives them to the ant. They put the nuts on their heads, and marched down the mountain to give the forest devil his tribute. Koi resumes his journey. At the base of the mountain, he encounters an alligator. The alligator crawls so very slowly. The alligator is crying.

"What's your trouble?" says Koi.

"I accidentally ate the rain maker's dog," says the alligator. "I am doomed. If I had only known it was the rain maker's dog."

"What difference does it make?" says Koi. "To eat any dog is bad enough, I say."

"What difference does it make? The difference is that the rain maker says he'll strike me dead with a lightning bolt unless I deliver him a [kinja] full of cola nuts by sunrise, and there are no kola nuts on this side of the mountain. It'll take me two days to find the kola nuts."

"Well, how do you know there are no kola nuts on this side of the mountain?" says Koi. Koi unwraps a corner of his mat to show the alligator his [kinja] full of kola nuts.

"Why, you have many cola nuts," says the alligator.

"Yes, and I will give you the entire [kinja] full. You need them more than I." So Koi fastens the [kinja] to the alligator's back, and the beast hurries to pay his debt to the rain maker.

That evening, Koi reaches a village. The guard asks, "Who comes to the village of the great chief [Filikily]?"

Koi stands proudly and answers, "I am Koi from the land beyond the mountain. I am the son of Chief Sadaka."

Now Koi is covered with dust and his legs were scratched from the rocks on the mountain. He does not look like a chief's son.

"I have run away from my home because the people there do not know how to treat the son of a chief," says Koi. "May I be a guest of your village?"

"Son of a chief," says the guard. "You look more like the son of a hyena."

As Koi speaks to the guard, many people from the village gather to see the visitor.

"He's very dusty," says one old woman. "He's no son of a chief. He's just an [assu], nothing but an outcast."

"Yes, he's just an [assu]," says her husband. "And he'll bring us bad luck too."

"Send him away," says another man.

"Let's eat him," says a very large and fat woman.

"Let's bury him in the anthill," says a tall man.

"No, let's feed him to the crocodile," says another man.

"I say cook him in the pot," replies the very large and fat woman.

The villagers like the fat woman's suggestion. They seize Koi and take him away to the pot, and they chant: "He's no son of a chief, he's merely a thief. There's nothing worse than an [assu] with a curse. Let's have a great big feast, or a little one at least."

Koi's body shakes with fear. "Surely this village does not know how to treat the son of a chief," he says, and a man dressed in a leopard robe emerges from a hut. It's the great Chief Filikily.

"What do we have here?" asked Chief Filikily.

"I am Koi, the son of Chief Sadaka from over the mountain," says Koi. "I ask if I may be a guest of your village, and your people take me to the pot to cook me."

The very large and very fat woman shouts, "He's no son of a chief! He's an [assu]! Let's cook him!"

The villagers agreed. They chant: "He's no son of a chief, he's merely a thief. There's nothing worse than an [assu] with a curse. Let's have a great big feast, or at least a little one."

"Wait," says Chief Filikily. "We shall test him first to see if he is truly the son of a chief."

Koi breathes a sigh of relief. Chief Filikily speaks, "If the boy chops down that palm tree so that it falls towards the forest instead of toward the village, we will let him go. If the palm falls towards the village, we cook him."

The chief's people laugh. The palm leans so sharply towards the village it looks as though the slightest wind will topple it.

Koi shudders and sweats. 'This test is impossible. You must think.'

"Please Chief Filikily. Send your people away and let me cut the tree down after dark."

"Yes," says Chief Filikily. "It is agreed. You have until morning to fell the tree."

The chief gives Koi an axe and the villagers go to their huts, dreaming of tomorrow's feast. Koi sits beneath the palm to await the night. As he waits, he weeps. He does not know how he will make the palm fall towards the forest. The cooking pot awaits him, he knows.

Koi hears the rustling of leaves and vines. "Who is it? Speak!"

"It's me, friend snake," says a voice in the night. "I have been tracking you all day. I wished to thank you again for the cola nuts you gave to me. My mother is well, and she sends you her thanks."

"That's kind of your mother," says Koy, "but my fortune has changed since last we met."

"Is that why you weep? Tell me."

"I must chop down this palm tree so that it falls towards the forest, or else I shall be cooked."

"But look at how it leans. The people of this village must drink too much palm wine."

"Exactly," says Koy. "It's a trick."

"Imagine that," says friend snake. "Well, we must play a trick on them. I will get my six uncles, the pythons."

The pythons are Africa's biggest and mightiest snakes. At moonrise, friend snake returns with his six uncles, the pythons. The great snakes wrap their tails around the leaning palm tree and wrap their necks around the nearby baobab tree. Koi chops the palm with all his strength, and when the last fiber of the palm is cut, the six uncles pull the tree over so that it falls towards the forest.

Chief Filikily wakes up the next morning and his people follow him to see whether Koi has passed the test. Koi sits on the fallen tree, drinking the milk of a coconut. The tree is in the forest.

"Look," says Chief Filikily. "The boy shall go free."

"It's magic," says a villager. "The boy used juju to push the tree into the forest."

"Cook him in the pot," says the very large fat woman. "Let's cook him tonight."

The people chant: "He's no son of a chief, he is merely a thief. There's nothing worse than an [assu] with a curse. Let's have a big feast or little one at least."

"Quiet," says Chief Filikily. "We will test the boy once more. We will scatter 10 baskets of rice in my fields. If he picks up every grain in the dark of the night, he is truly the son of a chief and I will free him. If he fails, we cook him tomorrow night."

When darkness comes, Koi goes out to the field with a basket. On his hands and knees, he feels the ground for grains of rice. He tries and tries, but he can see nothing. It's an impossible task. Koi feels hopeless. "This is no way to treat the son of a chief." A tear rolls down his cheek and falls onto the head of an ant.

A really high-pitched voice exclaims, "Rain! Rain! Rain! Everybody take cover! The floods have come! Wait, wait," says the ant. "Rain isn't salty. Look, it is a boy who is crying. Ah! Aren't you the boy who gave us the cola nuts two days ago?"

"Yes," said Koi, "but now I'm in danger. I must gather the 10 baskets of rice spread in the field by morning, or else I shall be cooked."

"Do not worry, my friend. You bring the baskets and my people will pick up the rice."

Soon the fields crawl with millions of ants, and each ant carries a grain of rice.

The next morning, the 10 baskets of rice sit before the hut of Chief Filikily. Koi sits in front of the baskets, eating a wild plum.

"You perform well," says the chief. "Now you go free."

The people of the village surround Koi and Chief Filikily. "He's a sorcerer," says one villager.

"He's the devil," says another.

"Whatever he is, he cheats us of a feast," says still another.

"Let Chief Filikily free him," whispers the very large and very fat woman to the very short man. "We will catch him outside the village and cook him ourselves."

Koi hears what the fat woman says. He speaks to the chief, "I'm afraid to go. Your people want to eat me, and they will overtake me when I'm outside the village."

Chief Filikily stands before his people and announces in a big voice, "There will be no feast\dots yet. I will test the boy one more time. I shall throw my medicine ring into the deepest part of the river, and if he brings it back to me, will you honor him as the son of a chief?"

The people of this village are pleased. In one voice they chant, "We promise, Chief Filikily. We promise, we promise."

The chief raises his arms over his head. There is silence. "It is decided," says the chief.

Chief Filikily and his people go to the river. The chief throws his medicine ring into the deepest part. Then all go back to the village, leaving Koi alone on the bank.

Koi looks at the deep, dark waters of that river. "Even Chief Filikily tries to destroy me," says Koi. "I do not even swim."

Koi wades into the river. He sees a long gray nose of an alligator gliding towards him. Koi is frightened. The alligator grins when he sees Koi.

"Don't you remember me?" asked the alligator.

"I'm not so sure," said Koi. "All alligators look the same to me."

"You gave me your [kinja] full of kola nuts and saved me from the wrath of the rain maker," says the alligator.

"Oh, I am glad that you still live," said Koi. "As for me, this is the night before I die. I must bring up the chief's medicine ring from the bottom of the river by dawn, or I shall be cooked."

"You don't say. Perhaps I can help you. I think I swim a little better than you."

The alligator dives under the water and comes up with a small clamshell. He dives again and brings up a fish. The alligator dives in and out and in and out of that water all night.

When the sun begins to rise, the alligator comes up for the last time. He has nothing. It's too late. Chief Filikily and his people will arrive in moments. Koi is certain he will be cooked.

"I thank you, alligator. You've helped me as though you were my brother, but it's no use. I will be cooked."

Then the alligator smiles. Looped on one of his biggest and sharpest teeth is Chief Filikily's medicine ring.

"That is it!" shouts Koi. "That is it!"

Koi takes the ring and dances with it to the village. When Chief Filikily sees that Koi holds the medicine ring, he smiles. He takes off his leopard skin robe and puts it on Koi's shoulders. The chief stands before his people and says, "Surely this is the son of a chief." Then the chief claps his hands and the most beautiful woman Koi had ever seen walks out of a nearby hut. "And this is my daughter. She shall be your wife."

Koi turns to Chief Filikily and says, "I have found a village where the people know how to treat the son of a chief."

Suddenly a smile comes over the face of the very large and very fat woman as she sings, "We will have a wedding! We will have a wedding tonight! Now we will have a feast! Finally, we will have a feast!"

And now the story is over.

\subsection{Credits}

Told by: Whoopi Goldberg;
Illustrator: Reynold Ruffins;
Written by: Brian Gleeson;
Music Composed by: Herbie Hancock;
Music Performed by: Herbie Hancock (keyboards), Bill Summers (African percussion);
Music Engineered and Mixed by: Will Alexander;
Technical Assistant: Darrell Smith;
Soundtrack Mixed by: Chris Nelson;
Editor: Mark Forker;
Paintbox Artist: Anna Pivarnik;
Post Production: Rebo High Definition Studio (New York NY);
Art Assistants: Lynn Ruffins Cave;
Assistant Director: John McCally;;
Associate Producer: Doris Wilhousky;
Producer: Ken Hoin;
Director: C.W. Rogers;
Executive Producers: Mike Pogue, Mark Sottnick;

\section{Mose the Fireman}

\subsection{Audio Summary}

Mose the Fireman is a humorous tale of the larger than life legendary fireman, Mose Humphrey, who fights fantastic fires, falls in love, and invents some rather amazing things along the way. The story is told by Michael Keaton, with music by John Beasley and Walter Becker. Backstage Pass visits with New York historian, Don Cannon.

\subsection{Transcription}

Come on in, settle back. It's curtain time at the Story Lane Theater. You've never seen a stage like this one. Anything can happen!

Today's story is called "Mose the Fireman." It takes place in the 1800s in New York City. New York was a much different place back then. Most of the buildings were made of wood, so in crowded neighborhoods, fires could be a big problem. And that's where Mose fits in. But before we get to Mose's story, we're going to stop off at this old firehouse, once known as Engine Company Number 30, and now known as the New York City Fire Museum. The man you see inside the museum is Don Cannon. He's a volunteer fireman, a historian, and an expert on the ways people have fought fires from the beginning of history to the present.

"I've always felt that fighting fires and studying and learning how to discipline myself and learning how to work with others and going through all the tough times in the street and in buildings is kind of like the price of admission to a community of people who you really know are good people. All firefighters are supposed to work in teams. You need each other to stay alive. You're working together towards a common objective, and you each depend on the other to complete the task together. For any firefighter, the height of his or her career would be to save a life, and they can go on with their lives and live a good life themselves.

"I was really delighted the first time that I read the story of Mose. He was such an attractive figure to me. Tough, rough and ready, heart of gold, loyal, energetic. Mose represented a stereotype of the Bowery boy, the newly arrived brash immigrant type, and they were looked down upon by a lot of people as being a crude bunch of ruffians, but he demonstrated and represented to the larger community that were good people, that they loved the city in their own way, and they would risk their lives to protect their neighborhoods.

"When fires broke out in Mose's day, the firefighters would be awakened by the deep, deep sound of the tower bells in the district where the fire originated. Mose's experience is very common: snug a bed, aroused by the bells of the fire alarm tower, jumping into his pants and boots neatly laid by the side of his bed, going out into the cold dark, running, hearing the footsteps of other volunteers running to the firehouse, pulling this thousand-pound hand-pumping engine, exhausted, racing to the fire. And then they had to find either a pond or a stream and connect into it through a hydrant or a plug with their leather hoses and start pumping, and each company tried to be the first engine in at a fire, and the honor of putting first water on a fire was a great one at that point.

"Fireman Mose would have looked at his world in much the same way that firefighters do today. I think that the people you work with or people he worked with were closely connected. They shared experiences, evoked the same kind of passionate loyalty and affection that fire companies do in Baltimore or Pittsburgh or Chicago today. I really think that whatever the time, whatever the place, I think there's a real understanding that rises above time, rises above place. There's a shared respect for life and for each other and for people. The sum total of everybody working together is quite astonishing."

Thanks for your thoughts, Don. And now, let's watch the story of Mose the Fireman.

Excuse me, excuse me. I hate to bother yous, but I just wanted to ask you if by any chance you'd ever heard of a guy by the name of Mose the Fireman. Moses Humphrey was his real name, and he was the greatest firefighter who ever lived, and this is going way back now. In fact, this goes so far back that New York City was so new it still had a price tag on it. But seriously, this is a long, long time ago. It's even before subways was invented.

Now, to begin at the beginning, Mose Humphrey was born in 1809 in a tenement on the Lower East Side of New York, the same year Abraham Lincoln was born in a log cabin on the plains of Illinois. Anyhow, that winter there was a big fire in New York. Every fireman in the city was there trying to put it out. It burned all through the night, destroying block after block of wooden houses. It was very sad. You see, they didn't have too many fire plugs at that time, so they had to pump the water right out of whatever river or pond or puddle or whatever was close by. Anyhow, just as they were putting out the last fire, there was a big explosion. Boo Boom, that blasted the top half of the building into a million pieces. Most of it landed in the icy waters of the East River, and everyone who had been trapped inside was given up for dead.

But not long after that, one of the firemen from Engine Company Number 40 heard a cry from down by the river, and right there lying in a busted hogshead among the frozen cattails, they found a little redheaded baby. Well, the firemen of Number 40 decided to adopt the little guy right then and there, and because of the way they'd found him, they decided to name him Moses, after a guy who was also found floating in a river, although not in the East River and not in New York. And so Moses it was, although practically everybody just called him Mose.

And so Mose grew up in a firehouse. He took his first bath inside a fire helmet, and they used a big coil of hose for his playpen. Now, as you already know, firefighting was a very primitive operation back in the old days. To begin with, they had only one piece of equipment called an engine, which was really just a big clunky water pump on wheels. And, because all the firemen were volunteers in those days, whenever the fire bell sounded, they would all stop whatever they was doing, run to the firehouse, and haul the old engine out, and the they'd drag it to the fire themselves with ropes, which was no piece of cake. You see, it wasn't until many years later that they figured out they could get horses to do it. These guys were not exactly valedictorians.

Pretty soon they'd let Mose ride to the fire with them, and he'd watch as the fire marshal shouted orders to everybody through a silver speaking trumpet.

"Man the brakes!" he'd say. The brakes were the big handles that pumped the water. "Man the brakes! Watch that hose there! Step lively, boys! Can the mongoose!"

And the men would pump up and down on the brakes as fast as they could, and this would suck the water out of the wooden pipes that ran along most of the main roads and shoot it out the other end. It was hard work.

Well, Mose grew up fast, and pretty soon he was right in there with them, hauling the engine. And not long after that, he was so big and strong that they'd just let him pull it all by himself. Now the engine at Company Number 40 was called Lady Washington, that's George's wife, and there wasn't a more beautiful piece of equipment anywhere. And whenever the fire bell sounded, Mose was always the first one ready to haul her off. You see, Mose was never one to fool around with the stairs. He'd just throw his boots out the window and take a flying leap right after them, putting them on as he fell and landing without a hitch. It got to where he could jump out the window, shave, comb his hair, brush his teeth, eat some oatmeal, read three chapters of a good book, drink some coffee without spilling a drop. Mose was some kind of fireman.

Now, I wouldn't want to give you the idea that Mose and the Bowery Boys of Company Number 40 were the only ones fighting fires in New York. There were fire companies all over this city, and every company thought their engine was the best, so I guess you could say they had a little rivalry going on between them. Whenever a fire alarm went off, three or four companies would all race to the fire at once, hoping to be the one to tap the fire plug and have the honor of holding the hose and putting out the fire. One time, two companies got into a brawl over a plug while the very fire they should have been putting out continued to burn. Now, Mose and Sykesy, who was Mose's best pal, couldn't believe what they were seeing. They had to do something.

"Who needs a fire plug anyway?" Mose shouted. "Come on, Sykesy, let's go." So Mose grabbed a pickaxe and just made a hole in the pipe itself.

"Sykesy, take the butt," he shouted. Sykesy grabbed the nozzle and pointed it at the fire, and then Mose stuck the other end right into the hole he made in the water pipe and began pumping the engine brakes all by himself. The water shot out with such force that the fire was out in no time, and Sykesy had to hose down the brawling companies just so as he could tell them to go home. That cooled them off.

Now as you probably already figured out, the Bowery was kind of a wacko place. There were live pigs running wild through the streets, rooting around to eat whatever scraps of food they could find. And whenever you turned, there were chimney sweeps and vendors of all kinds selling onions and whatnot, and there were guys with carts that sold steaming hot yams and fresh apples and baked pears dripping with syrup. I'm getting hungry just thinking about it. Somebody run down and get me a canoli! And each of these characters would have a song that they would sing, like the guy who sold clams would sing: "My clams I want to sell today, the best clams from a rock away, hey!"

But the most popular street food of them all was the boiled ears of corn that were carried in cedar buckets by the white corn gals of the Bowery.

Well, one day as Mose was swaggering down the Bowery with Sykesy, he spotted a white corn gal named Lize. And it might be a corny way of putting it, but when she sang, her voice was as sweet as the corn she sold.

"Hot corn, hot corn, here's your lily white corn. All you that's got money, poor me that's got none. Buy my lily white corn and let me go home." What an attitude, but hey, this is New York we're talking about. Besides, [sewed] corn, am I right?

Well, what can I say? Mose was gaga, head over heels in love. Now, the natural thing for a Bowery boy to do when he was in love was to ask his girl to go to the fireman's ball with him. But somehow, Mose just couldn't get up the nerve. You see, without some big scary fire roaring around them, Mose was as shy as could be. And so he'd just buy some corn and be on his way.

Well, pretty soon the fireman's ball is right around the corner, and Mose still hasn't asked her. But lucky for Mose, there was a big fire over on Delancey Street. That was some blaze, all right. The flames shot way up over the top of a four-story building, which was the tallest building there was in New York at that time, and it presented some real problems for the firemen because they didn't have ladders that could get up that high. But Mose was on the scene in a flash with the Bowery boys at number 40 right behind them.

The building was covered in flames, and it looked like it was going to collapse at any minute. Then all of a sudden, they hear this voice calling for help. And then Mose saw her on the fourth floor, right through the burning flames. It was Lize, his white corn gal.

Without even stopping to think, he ran over to a ship builder's joint and grabbed the tallest ship's mast he could find. Then, like the lunatic that he was, he sprinted toward the burning building and pole vaulted right up to the fourth floor. Well, what with all those flames roaring around him, Mose felt like his usual brave self. So then and there on the window sill, with the building about to collapse and the fire so hot that they were both sweating bullets, Mose asked Lize if she'd go to the fireman's ball with him.

Well, what's she going to say? "I'd love to," she said, and that was that. That's all Mose needed to hear. He was so happy he couldn't contain himself. He grabbed Lize and was just about to jump out the window with her, but then he realized he should probably do something a little more civilized, what with her being a real lady and all. So holding on to Lize with his right arm and the ship's mast with the other, he slid all the way down to the ground. It was a straight shot, no traffic.

Oh, by the way, that's how the fire pole was invented. I bet you didn't know that, did you? You probably thought it was Ben Franklin or Tom Edison who invented it, didn't you? Well, it was Mose who did it, and if photography had been popular then, I'd show you pictures to prove it.

Well, it wasn't long before the fireman's ball rolled around. Mose got all gussied up for it, although no Bowery boy ever went anywhere without wearing his red flannel fire shirt. You see, you could never tell when there was going to be a fire, and you had to be prepared. But because of his love for Lize, Mose put on a white dickey over the top of it. Very classy. Of course, Lize didn't look so bad herself. She was a sight for sore eyes. The two of them danced all through the night. It was Mose's dream come true.

Well, Mose wasn't one to waste time, and after a couple of hours of dancing with Lize, he was all set to propose marriage. That's all there was to it. So he wiped his mouth on his sleeve, and he cleared his throat, and he said, "Lize, will you\dots" But just as he was getting to the point, Mose heard a distant sound - a bell coming from the direction of City Hall. Being the remarkable fireman that he was, he could hear it over all the music that the orchestra was playing.

"Fire!" Mose shouted with all his might. "Fire! Turn out, boys, turn out!"

With that, every man in the place dropped whatever he was doing and ran for the nearest exit. But by the time they got to the fire, it had already burned several blocks. It was the worst fire New York had ever seen. Every single company in the city was there. Mose was up and down ladders rescuing people all through the night. But it was a losing battle. There just wasn't enough water, and the fire continued to burn out of control.

After a while, it began to look completely hopeless. Mose sighed the saddest sigh ever sighed. Gotham would never be the same. From where he was, Mose could see the Hudson River at the edge of the island. It was all so sad. All that water and no way to get it to the fire. But then suddenly, Mose, lunatic that he was, got a crazy idea. A cockeyed, nutso, wacko idea. Although somehow, the more he thought about it, the less nuts it seemed.

Without a moment to waste, Mose jumped down to the ground and grabbed the biggest shovel he could find. Then he rolled up his sleeves and started digging a hole in the general direction of New Jersey. Sykesy just looked at him. "Hey, this ain't no time for chasing groundhogs," he said. "For crying out loud, what's the matter with you? The whole town's burning down!"

But by that time, Mose was already underground and headed due west. "Hey, Mose!" Sykesy shouted. "Hey, where you going?"

"What do you think?" he shouted back. "To get water!"

Well, that's all Mose had to say. He always kept his word, and Sykesy knew Mose would be back with water just like he said. In a little while, practically everybody in town was standing there looking into the hole and just waiting for Mose to reappear. Then suddenly there was a loud pop, followed by a sound like a tidal wave might make, and then kaboosh! Up shot Mose like a spooked fish with the Hudson River right behind him.

Then without wasting a second, Mose picked up Lady Washington all by himself and wedged her into the hole as tight as possible. "All right, everybody," he said. "We're going to have to work together! Hook every hose up to Lady Washington! Don't nobody crowd around, there's enough H2O for everyone one of yous bums! And hurry it up, 'cause there ain't much time."

Well, every company did as they were told, and the minute all the hoses were hooked up, Mose started pumping the brakes. He was moving so fast it looked like a blur on top of a blur on top of a hallucination. But then all of a sudden, it happened. Water began to shoot out of all those hoses at once, and the pressure in each hose was so unbelievable that it took every single man in each company just to keep them under control.

In fact, there was so much water coming out of those hoses that the fire didn't know which way to turn. It just freaked out. I guess you could say that the fire had met its match. Who got it? Well, Mose continued pumping the brakes so fast that in no time at all, every single flame and every last ember was extinguished. Bada bing. Just like that, it was over. Mose had saved New York.

The whole city erupted with cheers. "Hooray for Mose!" everyone shouted. "Hooray for the king of the Bowery! Mose has done it again!"

Now, of course, this was true, but Mose was a very humble guy. He couldn't stand being the center of attention.

"I'm just doing my duty," he said.

You see, when you're as much of a hero as Mose was, you do stuff like drain the Hudson River without blinking. It's just what you do.

Well, people kept on cheering, and the noise was so loud that Mose had to scream his marriage proposal to Lize, who screamed back her answer, which was "Absolutely, yeah, definitely, no question about it, yes!" And since everybody was already dressed for the occasion, they went back up to the firehouse for the wedding, and that was that. Bada bing, married.

Oh, by the way, after Mose had gotten through with the Hudson, it took so long to fill back up that for weeks you could just walk right over to Jersey, which is not the brightest idea in the world, but hey, it's a free country. You want to go to Jersey, go to Jersey.

And as for the tunnel he had dug, it just kind of sat there for a while. But after a couple of years, somebody got an idea about putting train tracks down there on account of it would be nice and quiet and out of the way. And that's how the subways came into being. Tch, go figure. So I guess you could say Mose even had a hand in inventing the subways. But hey, nobody's perfect, am I right? Of course I am.

\subsection{Credits}

Told by: Michael Keaton;
Illustrator: Everett Peck;
Written by: Eric Metaxas;
Music Composed by: Walter Becker, John Beasley;
Music Performed by: Walter Becker (bass + banjo + mandolin), John Beasley (keyboards + piano), Steve Tavaglione (all horns), Paul Marchetti (drums + percussion);
Sound Effects by: Walter Becker, John Beasley, Paul Marchetti, Roger Nichols;
Music Engineered by: Roger Nicols, Dave Russell
Music Produced by: Walter Becker;
Music Coordinator: Bobbi Marcus;
Narration Recorded by: Tim Claman;
Narration Edited and Sountrack Mixed by: Chris Nelson, Sarah W. Neil;
Editor: David Russell;
Paintbox Graphics: Don Novak;
Post Production: Palace Production Center (South Norwalk CT);
Series Logo Music: Jay Ungar;
Art Production Director: Susan C. Anderson;
Art Director: Paul Elliott;
Associate Producer: Doris Wilhousky;
Producer: Ken Hoin;
Director: C.W. Rogers;
Executive Producers: Mike Pogue, Mark Sottnick;

\clearpage
\newpage

\section{Screenshots}

\begin{figure}[H]
    \centering
    \begin{subfigure}{0.45\textwidth}
        \centering
        \includegraphics[width=\linewidth]{Games/StoryLaneTheater/Images/StoryLaneTheater3Image1.png}
        \caption{Story Lane Theater 3 - Screenshot 1}
    \end{subfigure}
    \begin{subfigure}{0.45\textwidth}
        \includegraphics[width=\linewidth]{Games/StoryLaneTheater/Images/StoryLaneTheater3Image2.png}
        \caption{Story Lane Theater 3 - Screenshot 2}
    \end{subfigure}

    \begin{subfigure}{0.45\textwidth}
        \centering
        \includegraphics[width=\linewidth]{Games/StoryLaneTheater/Images/StoryLaneTheater3Image3.png}
        \caption{Story Lane Theater 3 - Screenshot 3}
    \end{subfigure}
    \begin{subfigure}{0.45\textwidth}
        \centering
        \includegraphics[width=\linewidth]{Games/StoryLaneTheater/Images/StoryLaneTheater3Image4.png}
        \caption{Story Lane Theater 3 - Screenshot 4}
    \end{subfigure}
    \caption{Screenshots from Story Lane Theater 3}
\end{figure}