\chapter{Story Lane Theater 1}

% \begin{figure}[H]
%     \centering
%     \includegraphics[width=\textwidth/2]{./Games/WriteAway/Images/WriteAway2CD.jpg}
%     \caption{Write Away 2 CD}
% \end{figure}

The first of the Story Lane Theatre games published and released by The Lightspan Partnership for the PlayStation 1.

Story Lane Theatre 1 features two video programs:

\begin{itemize}
    \item Anansi
    \item Brer Rabbit and Boss Lion
\end{itemize}

\clearpage
\newpage

\section{Anansi}

\subsection{Audio Summary}

Anansi contains two captivating traditional Jamaican folktales about the tail-spinning spider Anansi. The story is told by Denzel Washington, and is accompanied by the music of UB40. Backstage Pass visits the illustrator, Steven Guarnaccia.

\subsection{Transcription}

Come on in, settle back. It's curtain time at the Story Lane Theater. You've never seen a stage like this one. Anything can happen here.

In a few minutes, we'll hear why all stories belong to Anansi, the trickster spider from Africa. But first, let's go to New York City for a backstage visit with Steven Guarnaccia, the artist who drew the pictures for the show.

"This studio, the place where I work, is the place I feel the happiest working. This is a place that I go to to just be with my imagination really, and therefore it's a place that I find a really really fun place to make drawings. I just sit here at this desk, and I just have to look around a little bit and I start to have ideas for pictures. When I was in second and third grade, I was sick with rheumatic fever and I was bedridden for a long time. So instead of going out, and I didn't really have an opportunity to ride my bicycle or to play ball, I spent a lot of time looking and reading; watching. So books and animated cartoons, all sorts of drawn things, became the things that I lived with, and I never stopped being interested in those things.

"I carry a sketchbook with me everywhere. So the very first drawings of Anansi were made in this sketchbook, but they didn't even really look like Anansi, who's kind of part person, part spider. His hair is a little more realistic, he's wearing eyeglasses that look like the kind of eyeglasses people really wear. None of those things seemed interesting or special enough. So the next thing that I did was to work on the shapes of the character. I decided that instead of those realistic eyeglasses, he needed round circles. They looked a little bit more like eyes at the same time that they looked like eyeglasses. And I started to play with his beard and his hair, and I changed the soft floppy hat to a top hat which seemed more like a trickster's hat.

"Anansi was a perfect chance to get everything I felt about bugs into a whole story. Anansi he's a bug, he's a spider, but he's also a character. I wanted him basically to be a character that everybody, kids or grown-ups, thought was kind of cool. He doesn't always do the right thing, he doesn't always learn his lesson, but I think that his personality is really great. He's got a great sense of humor, he's always trying to succeed, and the fact that he doesn't succeed all the time I think just makes him all the more endearing, all the more kind of charming. The scene that I'm drawing right now is right before the hat-shaking dance that Anansi performs, and of course he performs it because he's got a hat full of hot, hot beans that he's just plumped down on his head, and the beans are hot, hotter than anything he's ever experienced, and his hair is acting like bolts of lightning shooting off of his head. His eyes are bulging out of his head about to pop, and his teeth are clenched, and in the next scene, his eyes are going to do a kind of spiral like this. That's when things get really hot and he won't be able to even see straight.

"If I had to come up with one important thing that I told kids, it would be to do the things that you love to do the most, whether that's taking care of animals or playing sports or drawing or reading. Whatever it is, I think you do the things best that you love doing, and I've spent my life doing the one thing that I love to do more than anything, which is to draw, make pictures, and I don't think you can go wrong if you do what you like."

Thanks, Stephen. And now let's head for Jamaica to hear two stories from Anansi, the owner of all stories.

Long, long time ago there was nothing in this world. Then there were fishes in the sea. Then there were animals in the bush. And in the beginning, there was one place: Africa was that place. Then there was people. Then people in the beginning when this world was new, they told stories: Africa stories. Them tell stories about tings they see, them tell stories about things they think, them tell stories about things they dream. Yeah man, them people tell stories all the time.

Then one day a terrible thing happened. Many, many of those people get taken away from them home in Africa. Men make them slaves, men bring them in ships and take them to Jamaica to work. Ah, sad, sad ting. Them work hard in Jamaica on the sugar plantations. It's tough going there, but them remember them old stories. You see, them bring the old stories with them. Them people in Jamaica still tell them stories. Them tell about Anansi the spider.

Now Anansi, him teeny teeny, but him smart. How else him get all stories name for him? You see, Anansi own all stories. When your mother tell you a story or your grandfather tell you a story, them borrow Anansi story. All stories belong to Anansi.

Now stories not always belong to Anansi. In the beginning, all stories Tiger stories. Tiger, him own all stories because him biggest and strongest animal in the bush. Tiger, him own everything in the bush. Anansi the tiniest in the bush, him own not'ing. When him whisper, no one hear. When him holler, others in the bush, they just all laugh at him.

One day Anansi decide to go ask for something to be named for him. So Anansi go to home of Tiger. When him come to Tiger, Anansi bow so low him forehead touch the ground. Him say, "Everyting name for you Tiger," Anansi say, "and how come nothing name for me? We give your name to everyt'ing: Tiger Lily, Tiger Moss, Tiger stories, Tiger this, Tiger that. But nothing bears me name."

"Mmm, ah that's true," Tiger say. "You want something named for you? Is that what you say?"

"Yeah man," Anansi say.

"Huh, what you want?" Tiger say.

"I want all stories be name for me," Anansi say.

So Tiger, him think to himself, "mmm. I trick this boy. I make a joke on Anansi."

"That cool." Tiger say, "That cool, man. I name stories for you Anansi. But first, there's a little ting I ask for." Tiger say, "Bring to me Snake. Him that live by river, huh. Bring him to me alive on Saturday. Then you have the stories."

See now, all the animals in the bush, they all laugh now, because Tiger him make fool out of Anansi. Snake is big big big big and Anansi the spider him teeny teeny teeny. Cling cling bird laugh big belly laugh. Frog him laugh too, and Monkey him jump backwards over and over him laugh so hard.

Anansi no care. "Snake, him too small," Anansi him say. "That the deal, man. I bring you Snake Saturday, them all stories name for me."

So Anansi leave home. Him hear laugher in the bush wherever him go. All creatures big and small just laugh at him. Tiger trick Anansi bad.

Now this was Monday. On Tuesday, Anansi get big idea. Him t'ink it a marvelous idea. Anansi decide to build a [caliban] to catch Snake. So him take strong strong vine and make a noose. Him hide vine in grass, him set some berries that Snake love best in noose, then Anansi wait and wait, and Snake come slither down through the bush and him see the berries.

"Mmm\dots Ooh, them berries do be my dinner," Snake say. So Snake lay across the vine and eat them berries.

Now Anansi pulls hard on the noose, but Snake too heavy. Anansi pull and then pull and pull, but it no good. [Caliban] fell. Snake eat all them berries and Snake go home.

Wednesday, Anansi get big big idea. "This one work," Anansi say. "Today me catch Snake." So Anansi dig deep deep hole by side of road, and then put grease on side and bottom of hole and make it nasty and slippery. On the bottom him put six banana. Now Snake, him love banana.

Anansi tink "Snake go in slippery hole for banana and him can't get out 'cause him side too slick. This do the trick, this do the trick." So Anansi him hide in the bush beside the road to wait.

And Snake come slithering down the road - him hungry for dinner. Snake see them bananas at bottom of hole, but him know sides slip slide from grease, so him wrap tail around trunk of tree. Then Snake reach down in a hole and get bananas. When him finish eating, him pull himself out of hole with him tail. Snake him just crawl away, him belly full with Anansi's bananas. Anansi no had no bananas. Anansi no have Snake.

Thursday. Anansi get big big big biggest idea of all. Him decide to make a fly-up and catch Snake. "Yeah, fly-up catch a snake," Anansi say. "Fly up catch a snake. Snake put him head in trap and go up, catch him good." So Anansi builds fly-up. Inside trap, Anansi put one egg for bait. "Oh, would the Snake feel happy when him see egg."

Snake just lift him head and stand up. Him bend over and snatch egg real easy right out-a trap and not even touch the trap. Not even fly-up could catch Snake.

Friday morning come and Anansi worried. Snake him take all of Anansi food and Anansi no catch Snake too. "Rotten luck," Anansi say. So Anansi him think all day Friday. Him sit on pebble and tink hard how him catch Snake, but not'ing come to him. Anansi going to lose deal with Tiger.

Saturday morning come and this day Tiger want Snake so stories be named for Anansi. But Anansi have no way to catch Snake. This look bad for Anansi. Yeah man.

Anansi feel like walk down by river. Him go there and him think today. Him come to Snake hole. Snake watchin' sun come up. Snake him head out of hole but him body in the hole.

"Ah, Anansi! Me very angry with you, man," Snake say. "All week you try and catch me. First the [caliban], then the slippery hole, and yesterday the fly-up. Me have a good mind to eat you up," Snake say.

"Yeah, but me just [try and prove me] longest thing in the world," Anansi say.

"Yeah but everybody know me longest ting in the world," Snake say.

Anansi know him got plan now and just keep on talking, tell him own story. "Uh, Snake," Anansi say, "you longer than bamboo tree over there? Me tink not."

Now Snake got him big big pride. Him say, "I bigger than biggest bamboo." So Anansi him go chop bamboo tree with him teeny teeny machete. He put it near Snake hole.

"Me don't tink so Snake," Anansi say. "Why don't you come out of that hole and see yourself?"

Now Snake him know him long. Him move quick to prove it to Anansi. Snake him lay on top of bamboo.

"No, I don't think so," Anansi say. "Bamboo longer. Look at Snake."

And Snake see that bamboo a little longer because him tail curl up at end so it look like bamboo longer.

"Hmm\dots Tie me tail to bamboo, Anansi. It curl at the end. It need to be straight. Tie it tight."

So Anansi do what Snake say. Him tie Snake's tail to bamboo. Now Cling Cling Bird him see what go on and him call the other animals to watch. Frog come, Monkey come, even Tiger come.

"Just provin' me point," Anansi say to them animal. Him give eye wink to him all though. "Snake, you're long but you curl up even at your belly," Anansi say.

"Then tie me there too and tie it tight. Me show you me longer than them bamboo."

So Anansi do what Snake say. Him tie Snake belly to bamboo. Them animals all watching, no say peep. Them not believe them eyes what them see.

"Snake, you're still shorter them bamboo," Anansi say. "Not much though, about six inch."

"Augh, what me do now," Snake begged Anansi.

"I think you can do it," Anansi say. "You got to stretch hard man. You got to stretch so hard that your skin pull up and your eyes close. Make yourself longer man.

"Stretch long." Snake say.

All them animals, "stretch long!"

And Snake stretched long. Him keep stretching and them animal call to him, "You winning, Snake! You're almost longer than the bamboo!"

Snake stretch so hard him eyes closed like Anansi said. When him close eyes, Anansi tie Snake head to bamboo. Anansi done it. Him catch Snake. Snake tied up on bamboo ready for Tiger.

Tiger and them animal all be silent. Anansi, this teeny teeny ting, catch big big Snake his self. So now Tiger come close to Anansi. Him tail move back and forth. Snake on bamboo beside them. Tiger him say not'ing. Him just keep walking in [another] bush. Tiger him look back at Anansi from over his shoulder. Him nod in the direction of Anansi. Him say nothing but we all know what look mean. With him eyes, Tiger say, "You win, boy. All stories now belong to you Anansi."

And after that, nobody dare call them Tiger stories no more. Anansi win the stories. They belong to Anansi, 'cause him catch Snake. All stories belong to Anansi. Even this one. Yeah man.

Now one time, story get Anansi into trouble. This story so big it make Anansi the spider bald. It's true, man. Look at Anansi now. Him got no hair on him head. Him bald like mango. This how it happened:

Anansi mother-in-law died. Sad time for him wife. So Anansi sent her to funeral without him. Anansi say to him wife, "What good, wifey? Me soon come to mother-in-law funeral." Anansi got to t'ink hard, man. Him big big man. Him stay home to figure out how him going to be special at funeral.

So after wife go, Anansi t'ink how him going to show him big sadness at mother-in-law funeral. "Everyone eat at funeral. Me be different. Ah ha! Me no eat for seven days," Anansi say. "This would show me have most sorrow at mother-in-law funeral. Make me big big man."

But first before he go, he meet everything in him house. Him eat for 3 hours. Mango, plantain, yam, everything him eat. Anansi eat so much, he get big tummy ache. Him belly full man.

Then Anansi go to him mother-in-law funeral. After Anansi mother-in-law buried, them animals say, "Eat, Anansi, eat. You have a long trip. You must eat."

"What kind of man eat after him mother-in-law buried?" Anansi say. "Me no eat to show me sorrow. Me eat on the eighth day."

So Anansi no eat. Them animals think him sad. Nobody before starve himself because mother-in-law buried. Animals t'ink him good man to show him respect.

Second day come and Anansi eat not'ing. All the animals say, "eat, Anansi. You must eat."

"What kind of man eat after him mother-in-law buried?" Anansi say. "Me eat on the eighth day."

Now them other animals eat feast, man. And Anansi him just watch. Him get prouder second by second.

Third day come and Anansi eat not'ing. Them man say, "Eat, Anansi! You don't need to stop."

"What kind of man eat after him mother-in-law buried?" Anansi say. "Me eat on the eighth day."

Ho ho ho! But this getting hard for Anansi. Three days a long time to go without no food. Him starve to death. Him belly hurt so hungry. Him watch animals eat them meal. Him him him watch Monkey eat rice and Anansi can taste it in his own mouth. Anansi so hungry him dream of food at night when he sleep.

"Eat Anansi, eat on the eighth day. Eat, eat on the eighth day, Anansi. Eat\dots"

Fourth day come and Anansi head spinning around and round him so hungry. All the animals still sleep, so Anansi go for a walk to take him mind off food. Anansi walking, him see pot of beans cookin' on fire real hot. He can't control himself. He go right for them beans. He gonna eat. There ain't nobody here. For better for worse, he gonna eat now.

So Anansi stick the spoon in the pot and snatch beans and stuff 'em in his mouth. He eat fast fast fast fast fast. Him so hungry. Anansi stick spoon in a pot again. Ah, this time he think better. Him think can take beans to the bush and eat them in peace and quiet so nobody see him ea'ing. But when he lift the spoon out of the pot, he hear the other ones coming. Pig, Rabbit, Dog, Cling Cling Bird, and Monkey come walking toward them beans and Anansi.

Anansi think quick. He poured them hot hot beans into him hat. Then Anansi put that hat with them hot beans on him head. The man will say, "Eat, Anansi. You must eat." But Anansi still go on with him story. "What kind of man eat after him mother-in-law buried? Me eat on the eighth day." But this time is different. Them beans burning him head bad, man. Heat so strong, Anansi's eyes popping out of him head.

So Anansi take his hat and move it forward and backward, side to side. But heat getting worse on Anansi's head. Him jump and shake [up], gettin' real lively. Him do this so much it funny.

"What you doing with your hat dere man?" Dog say. "You got honey bee in there?"

Anansi him got another story him tell. "Ah, today is a day of hot shaking dance in village where I grew up," Anansi say. "It's great dance. We do it like I do." Anansi shaking and dancing all over, just jiggling in him hat all about. He can't stand still, the beans so hot he burning up under his hat.

"Um, me me must go to village and do the hat shaking dance with me people," Anansi say. "See, festival too much fun to miss. They need me there."

So Anansi pushed through all them animals, shaking and jiggling him hat. Him moving real quick to get out of there. Smoke coming out of him hat, him head burn so bad.

"But you must eat, Anansi," them animals say. And Anansi take off in a big hurry. "Before your journey, you must eat. What kind of man eat after him mother-in-law buried?" Anansi say.

Anansi run quick like mongoose, but them animals still behind them. They still beg them to eat. "Eat, Anansi," them animal say. "You must eat."

Hot beans hurt Anansi's head bad. Anansi just t'ink about getting that hat off him head. Him can't stand it no more, so Anansi take him hat off and them hot hot beans come tumbling down all over him head and face.

Them animals seen beans dripping on Anansi's head. Pig's seen it, Rabbit's seen it, Dog's seen it, Cling Cling Bird's seen it, and Monkey's seen it. Them stop right there in their tracks. They know Anansi ain't no big man. Him say him no eat out of respect for him mother-in-law, but him eat all along. Him telling stories again.

"Such a big man," Monkey say.

"Oh no, no, no, me can't eat," said Cling Cling Bird in a mocking voice like Anansi. "What kind of man eat after him mother-in-law buried?"

Well, them animals all laugh big at Anansi. They scorn him and jeer him. Anansi him run away from them. Him ashamed that him tell story and get caught. Him run in bush to clean him head, and he rub all them beans off him head.

Now when Anansi rub beans from his head, him feel no hair there no more. Yeah man, top of him head shiny. Them beans burn him hair right off him head. Now this make Anansi be more ashamed. Him go back home and t'ink some more. Him sit in him web and think long long time about him mistake.

To this day, Anansi the spider still bald. Yeah man. Bald like mango. So if you tell Anansi story, you tell the good story, not the bad one. You no want your head to be bald like Anansi if you tell the bad one. Tell the good Anansi story all the time. Yeah man.

\subsection{Credits}

Told by: Denzel Washington;
Illustrator: Steven Guarnaccia;
Written by: Brian Gleeson;
Music by: UB40;
Music Engineered by: Delroy McLean;
Music Mixed by: Matt Lane;
Narration Recorded by: Jeff Sheridan;
Soundtrack Mixed by: Chris Nelson;
Editor: David F. Russell;
Paintbox: Don Novak;
Post Production: Palace Production Center (So. Norwalk CT);
Art Director: Tin Raglin;
Art Production Manager: Kenna Kay;
Editorial Director: Eric Metaxas;
Associate Producer: Doris Wilhousky;
Producer: Ken Hoin;
Director: C.W. Rogers;
Executive Producers: Mike Pogue, Mark Sottnick;

\section{Brer Rabbit and Boss Lion}

\subsection{Audio Summary}

Br'er Rabbit and Boss Lion is a witty southern folktale about mean Boss Lion, who threatens the peaceful folk of Br'er village, and Br'er Rabbit, who's forced to teach the carnivorous troublemaker a lesson he'll never forget. The story is told by Danny Glover, with music by Dr John. Backstage Pass visits the illustrator of the story, Bill Myer.

\subsection{Transcription}

Come on in, settle back. It's curtain time at the Story Lane Theater. You've never seen a stage like this one; anything can happen here.

Today we're going to hear the story of Br'er Rabbit and Boss Lion. The story was first written down about 100 years ago by a man named Joel Chandler Harris, who lived not too far from this house outside Atlanta, Georgia. Mr. Harris wrote down all kinds of stories he heard people telling throughout the South, and he called them the Uncle Remus Stories.

The two people you see sitting out on the porch are Bill Mayer and his wife Lee. Bill is the person who made the pictures you'll see in the show.

"I love doing art. It's just a wonderful thing and a marvelous thing to make a living at. It's something that always has been a part of me from the time I was very, very young, and it's impossible for me to escape, you know. If there's a party and some kids are drawing on the floor, I'd much rather sit down on the floor and draw with the kids than sit around and talk to adults. I mean for me, drawing is the most wonderful way to express yourself, and most of my drawings have a lot of humor in them, but it's because that's really the way I view life. I see humor in everything.

"Most of the time when I sit down to draw little characters or animals, it's more like a stream of consciousness. When I started on the Br'er Rabbit video, since it was a little town on the river, that's how I sort of pictured it. I took my canoe down to the Chattahoochee River, and I got in the canoe and Lee brought her book, and she was reading her book and I just started doing little thumbnail sketches of how I pictured the little river town to be. It's funny, you know, when you're floating down the river going past little decaying barns and, you know, trees, and the way that trees are foggy in the morning and misty, and all of that stuff just sort of seeps its way in. You absorb the things that you're around, and when they come out, they come out in more of a personal way.

"My idea of how I saw the images on the screen, the village and the characters, were very different from the way they've been interpreted before. The little town, I had an idea to picture it in the way that I always remembered the South. The little houses were leaning and sort of patch tin roofs. The way people dress in overalls, what they wear, and the way they act are just things that I've always seen in the country when I take drives, you know, and I tend to just remember them and try and use them.

"The little characters like the Buzzard happens to look a lot like my neighbor down the street. In fact, his kids kind of recognized him one day and went, 'Oh, it's Dad! Look!' Even the little pigs and geese that are being swallowed by the lion, they do tend to resemble my nieces and nephews.

"These are some cutouts that I did of Br'er Rabbit. I always saw Br'er Rabbit as not a cute character, but one that had a lot of personality. He's a rascal, he's very intelligent, a bragger, a little bit full of himself. And the way we use these cutouts is we have a background and we place them on the background by overlaying the figures over the top. The rabbit sort of acted out the scene where he's saying goodbye to all the townspeople. That way it allowed us to show a lot of expression and a lot of movement as it went through.

"I think the feeling I'd love for people to have when they finish with the story is that they really had a great time listening to it. Uncle Remus for me is something I remember my grandmother reading me, and it has real fond memories because they were wonderful stories. The stories always had a way of telling morals or explanations for things that happen in our lives. The moral for Boss Lion is that if you use your intelligence, you can pretty much outsmart anyone. You can overcome obstacles in your life."

Thanks Bill! And now, let's watch the story of Br'er Rabbit and Boss Lion.

Way down in the deepest South, along the river they now call the Mississippi, there once lived an assembly of animals who were all the best of friends. Now this was a good long time ago, so long ago that most folks forget, but some still remember how the animals in Br'er Village got along just as kindly as any a critter could. They helped each other out of fixes and fixed each other out of hard times. Things were peaceable and pleasant, generally speaking that is. But one day, all that changed when Boss Lion, the king of the forest, as he liked to call himself, came along and set himself down in an old cave on the outskirts of Br'er Village.

Now, as everyone knows, or ought to know, Boss Lion was not a fellow to be messing with. He was the meanest, baddest, biggest, smelliest, proudest, fattest cat in the forest, and he was not so nice besides. He had a mouth full of scissors-sharp teeth, he had claws that could cut clear through a cottonwood tree, and a mane of hair which he kept slicked and oiled at all times. And this here Boss Lion, if it isn't clear already, was certainly no vegetarian.

And sure as a hog's a pig and vice versa, Boss Lion lived up to his ferocious reputation. He gobbled up pawsful of gooselings, he swallowed a six-pack of piglets, he laid about on the outskirts of town fat and mean, doing not a lick of work, eating up whatever animal came along, and being just generally antisociable in his behavior.

After a few days, them Br'er folks had enough of Boss Lion and held a meeting to see what they could do about him. They gathered in the center of town and argued back and forth this way and that, then that way and this, and a few other ways besides. And finally, they decided that someone would have to go to Boss Lion and explain to him the score - that if he kept up the way he was doing, he would destroy the village and there wouldn't be no more animals left to live there. Things were getting so bad they'd decided to make a deal with Boss Lion: if he promised to stay inside of his cave, they would come every day to feed him.

Then Br'er Fox stepped forward and cleared his throat.

"Well, folks," he said. "Seeing that we got that matter settled, all we got to do now is figure out who's gonna talk to Boss Lion."

At that, all them folks fell suddenly silent. After a couple minutes, Br'er Pig snorted and spoke up. "I wouldn't mind one bit telling that old lion what we decided here, but I got some rooting to do before the rains come."

And then Br'er Bear came forward, "I'd love to give Boss Lion a piece of my mind," he said, "but I'm behind on my berry harvest and I got to make fruit cocktail for the cubs."

Then everyone started talking all at once. Br'er Coon said he had to get his [hay] in from the field, Br'er Goose said he had to take care of his wife while she lay an eggs, Br'er Gobble had a headache and needed to buy some aspirin, and Br'er Frog had to get a haircut and then needed to [paint his pad]. It looked as if no one was going to talk to Boss Lion.

And just as the meeting was about to break up and everyone go off all discouraged and defeated, Br'er Rabbit stepped into the center of the circle.

"I swear by my white whiskers," he said, "y'all might be too tender-footed to tell that mean [inaudible] lion what we think of him, but I'm not."

Now Br'er Rabbit was not a very big fella. In fact, even as hares go, he was a puny sort of guy, always getting himself into all kinds of unaccountable trouble on account of that big mouth of his. So Br'er Gobble shouted out, "Br'er Rabbit, you can't go to Boss Lion! He's ten times bigger than you!"

"Yeah," Br'er Fox added, chuckling to himself. "That lion will eat you up as an appetizer and make a keychain out of your foot!"

"Now, now, now, listen folks," Br'er Rabbit said. "I'm going to tell you now and y'all listen good. Br'er Rabbit ain't afraid of nobody, certainly not some scurvy-breath, greasy-headed, flea-infested, boil-faced Boss Lion. Appetizer? I'm off right now to pulverise his paw, kick him in the jaw, and maybe even mispronounce his mane if I feel like it. We'll see who makes a keychain out of who!"

And with that, Br'er Rabbit hitched up his pants and pulled his [duck-clogged] cap over one eye. He removed a cigar from his breast pocket, put it in his mouth, and sauntered off to the outskirts of the village, whistling the entire time as if he was heading to the fish hole to hook him a bullheaded catfish.

Yet the minute Br'er Rabbit turned the corner and was out of view from the rest of the villagers, he stopped whistling and stubbed out his cigar. He fixed his cap straight on his head, and the closer he came to Boss Lion's cave, the more he started sweating and shaking, until his legs were wobbling so bad he could hardly walk straight. When he got to Boss Lion's cave, he took his cap off and knocked meekly on the door: tap, tap, tap.

The door flew open and Boss Lion stood in the threshold, breathing straight down Br'er Rabbit's neck.

"Who are you and what do you want?"

"Uh, it's me, Br'er Rabbit," he stuttered.

"What do you want?" Boss Lion demanded.

"The\dots the folks here had a, a little meeting. They asked me to tell you, since you are the big boss, it's not right that you should have to get your own food. They says that if you stay at home all the time, they'd be just plum honored to bring you all the food you ever wanted."

"I want fresh meat three times a day," Boss Lion growled at Br'er Rabbit. His breath was something awful, ho. "And if they don't bring it to me, I'll eat them all up and destroy their village!"

"Yes sir, yes sir, Mr. Boss Lion," Br'er Rabbit stammered. "They will bring you meat three times a day, sir, and good eating meat at that, sir. You can count on me, 'cause I'll make sure it personably!"

With that, Br'er Rabbit lit off like buckshot, as fast as his furry feet could take him. But as soon as he drew near the village, he stopped to catch his breath. He fished out his old cigar, cocked his cap back over one eye, and ambled into town just as calmly as if he was on his way back from the fish hole. When they saw him coming, they all ran out to greet him.

"Did you see Boss Lion?" they asked. "What did he say? Did he eat you up as an appetizer? Weren't you scared?"

Br'er Rabbit silenced them with a wave of his paw. "Lord Almighty, I told you time and again that Br'er Rabbit ain't afraid of no fat cat lion. You asked me to tell him your mind, and I did."

Them folks was considerably curious and demanded of Br'er Rabbit a full accountin' of what had happened.

"Well," he said, "I went straight up to that cave, invited myself in, and sat down on the chair crossing Boss Lion. And I put it to him directly, saying, 'Look here, Mr. Lion, you're disturbing the peace of our village, not to say eating us all up. Now, none of us folks like to see another starve, so we'll feed you ourselves. So you got to stay in your cave at all times and got to eat whatever we give you. Otherwise, I'll personally kick the living kidneys out of you and lock you up in the hospital.' Well, that old cat growled and barked a bit, but he saw I meant some serious business and that I weren't scared of him one whit. So he turned it over a piece and then agreed."

"But one thing," Br'er Rabbit added. "We got to feed him three times a day."

At this, a big roar of praise rose up from the crowd, and everyone talked about how brave Br'er Rabbit was. But after a minute, the hubbub died down, and a few folks started wondering aloud where the meat would come from and who was going to feed Boss Lion first. Well, wouldn't you know it, they all started yammering and quarreling and pointing and screaming at each other, "You first! You first!"

So Br'er Rabbit held up his paw again and suggested they draw straws. Br'er Rabbit just happened to have a few straws in his back pocket, so he held them in his fist and all the folks drew straws. Now it came to pass that Br'er Goose picked the short one, and when he saw it, he hemmed and hawed and honked out like, "Nope, nope, nope!" And Br'er Rabbit says, "Yep, yep, yep," and pointed in the direction of Boss Lion's cave on the outskirts of town.

So Br'er Goose hung his beak low and headed out to Boss Lion's cave, stopping at home first to cook an omelet for Boss Lion. And when he got to Boss Lion's cave, it was just about twilight, which is to say supper time, and Boss Lion gobbled up Br'er Goose and ate the omelet for dessert.

The next morning, Br'er Rabbit held the straws again, and this time Br'er Pig drew the short one. And when he saw it, he squealed and screamed and oinked out, "Oh no, no, no!"

And Br'er Rabbit said, "Yep, yep, yep," and pointed in the direction of Boss Lion's cave on the outskirts of town. So Br'er Pig hung his head low and made his way to Boss Lion's cave, stopping first at home to get a milkshake for Boss Lion. And when he got to Boss Lion's cave, it was breakfast time, and Boss Lion ate Br'er Pig right up and used the milkshake to shampoo his hair.

And so it went on like this at each feeding time for a couple of days, with Br'er folks going out to feed Boss Lion and never coming back again. And on the third day at lunchtime, Br'er Fox up and made an announcement. "It's my turn now to hold the straws, he he he\dots" He peered over at Br'er Rabbit, and a thin smile of satisfaction spread across his face. He had been watching Br'er Rabbit and realized that as long as Br'er Rabbit held the straws, he would send everybody else to feed Boss Lion except himself.

So when Br'er Fox held the straws that afternoon, sure enough, Br'er Rabbit picked the short one. When Br'er Rabbit saw his straw, he didn't hee-haw or honk. Instead, he grew pale and pensive and said out loud to no one in particular, "Well, if that ain't the cat's tail itself, I'm pickled bunny in the barrel."

And then he hopped up on an old tree trunk, cleared his throat, and addressed his fellow critters.

"Critters," he said, "we sure had some good times together. Some laughs, some cries, some close calls. But now it looks like my time is up, and I suppose I'm aiming to be an appetizer for that old lion. So pray for me, folks, and perhaps soon we'll all meet in that big fish hole in the sky. Goodbye, ladies. Goodbye, gentlemen. And goodbye, Br'er Fox."

And he hung his head mighty low and walked out of the village into the leafy cool of the afternoon. And all those folks were so moved that tears began rolling down their faces.

Now Br'er Rabbit was in no considerable rush to get to Boss Lion's cave, so he decided to make a detour and see his farm for the very last time. He took off through the woods, and when he arrived there, he looked over his carrot patch.

"Goodbye, carrot patch," he said, waving to the green shoots of carrot in the ground. He saw a shovel and said goodbye to it. "I'm shoving off myself, Mr. Shovel. Nice to have worked with you." Then he walked to his well to bid it goodbye and take a last drink of sweet water. "Farewell, trusty well," he said, and saw his reflection in the bottom of the well.

And when he saw his reflection, he got such a good idea that he yelled out loud, "Yes sir, you are a farewell 'cause you always looking out for my welfare!" And at that, he scampered off tippity-top like the wind itself to Boss Lion's cave.

By the time he got to the cave, it was getting on toward evening. The sun was setting behind the cottonwoods, and the evening swallows were in the air. Lunch had long passed, and Boss Lion was mighty angry at that. "Where is my meat, rabbit? You are late with my meat!"

"I, I, I tried to get here soon as I could. It's the Lord's truth. You see, I had so much meat for you that I couldn't carry it all by myself, so I stashed it away. And if you'll follow me, I can take you to it." Boss Lion looked as if he were about to pounce on Br'er Rabbit and gobble him up right there, so Br'er Rabbit had to do some mighty lickety-split fast talking.

"Look, look, look here, look here, Boss Lion," he pleaded. "I'm just a teeny rabbit. I'd only be an appetizer. Big lion like you, why, I wouldn't be an appetizer. I'd be just a bite of fur and bones, you couldn't even put me on a cracker, I'm so small. But if you want a whole lot of meat, a great big pile of fresh, juicy steaks, I promise I'll show you. I promise!"

"Take me to the meat, and it better be enough, or I'll eat you right now!" And so they took off together through the woods to Br'er Rabbit's farm. When they reached Br'er Rabbit's well, he opened the door, looked in, and fell backwards as if he was stung by a honeybee.

"Lord Almighty!" he yelled to Boss Lion. "There's some big critter in there, and he's eating your meat!" Boss Lion pushed Br'er Rabbit aside and stuck his head in the well.

"Who are you?" he growled down into the deep well, and a few seconds later his echo came back up: "Who are you?"

Now old Boss Lion was not used to being talked back to, seeing he was the king of the forest and all, and when he heard the voice boiling back up from the well, he grew awfully angry and roared back down, "Who am I? I says who are you!" And once again the voice came back, "Who am I? I says who are you!"

Boss Lion was now turning strawberry red with rage, huffing and snorting and growling. And just then, Br'er Rabbit poked him in the side. "You hear him sass you?" he said. "That lion down there is eating your meat and making a mockery out of you. You going to take that from him? Why, curse his pile of fresh meat-stealing soul! I'd whoop that lion down there myself if he has the guts to get up here!"

Boss Lion looked into the well again and saw a lion staring right back up at him, a mean and ugly lion, and he yelled at it, "I'm going to get you!" And it yelled directly back at him, "I'm going to get you!" And this was just about all Boss Lion could handle.

"Step back, rabbit! That cat's dead meat!" And with that, Boss Lion took a flying leap right into that well. There was a great crash deep inside. Cuplunge! And as soon as Br'er Rabbit heard it, he slammed the lid shut and locked it.

After a few seconds of silence, he took out his cap, cocked it on his head, and pulled out his old cigar. And he sauntered off into town at a leisurely pace while the last [wisps] of sunlight were softening over the forest.

When he got near the village, all the folks was standing around debating who was going to feed Boss Lion next. And when they saw Br'er Rabbit coming down the road in the [gloaming], they thought for sure he was a ghost and started running to save their souls. But Br'er Rabbit stopped them and told them he was no ghost, and as soon as they realized it was so, they asked him all excited, "Did, did, did\dots did Boss Lion eat you up?"

"Eat who up?" Br'er Rabbit asked back. "Me? No, ain't no confounded lion come near to these [naps] of mine!"

"What, what did he say?" the folks asked. "What, what did he do? Won't he come eat us all up now?"

And at that, Br'er Rabbit laughed and explained to them what he had done. Well, he changed the story around a little and said that when Boss Lion started getting rough with him, he simply beat the living kidneys out of him as he had promised he would and threw him into the well. And though most folks didn't believe him at first, he later showed them the well and the drowned troublemaker at the bottom, so they had to believe him.

And for many, many years afterwards, Br'er Rabbit was a hero in those parts along the Mississippi River. People told stories about him and sang his praises for acres around. And things were pretty peaceable once more in Br'er Village, except for a few small scrapes here and there and maybe a minor spat now and then. But as long as anyone can remember, no Boss Lions ever came by again to bother the folks of Br'er Village.

\subsection{Credits}

Told by: Danny Glover;
Illustrator: Bill Mayer;
Written by: Brad Kessler;
Music by: Dr. John;
Music Performed by: Dr. John (piano + guitar), Richard Crooks (drums), Zev Katz (electric bass), Ronnie Cuber (flute + saxophone + claronet);
Music Recorded and Mixed by: Ted Spencer;
Music Coordinator: BB.;
Narration Recorded by: Dough Botnick;
Soundtrack Mixed by: Chris Nelson;
Animation Camera/Editor: Mark Fokker;
Assistant Editor: Stephen Marlin;
Post Production: Rebo High Definition Studio (New York NY);
Assistant Illustrator: Lee Mayer;
Art Assistants: Patrick Vickery, Jason Mayer;
Photo Stats: Pete Brown, David Inman;
Associate Producer: Doris Wilhousky;
Editorial Director: Eric Metaxas;
Illustrator: Bill Mayer;
Written by: Brad Kessler;
Art Director: Paul Elliott;
Producer: Ken Hoin;
Director: C.W. Rogers;
Executive Producers: Mike Pogue, Mark Sottnick;

\clearpage
\newpage

\section{Screenshots}

\begin{figure}[H]
    \centering
    \begin{subfigure}{0.45\textwidth}
        \centering
        \includegraphics[width=\linewidth]{Games/StoryLaneTheater/Images/StoryLaneTheater1Image1.png}
        \caption{Story Lane Theater 1 - Screenshot 1}
    \end{subfigure}
    \begin{subfigure}{0.45\textwidth}
        \includegraphics[width=\linewidth]{Games/StoryLaneTheater/Images/StoryLaneTheater1Image2.png}
        \caption{Story Lane Theater 1 - Screenshot 2}
    \end{subfigure}

    \begin{subfigure}{0.45\textwidth}
        \centering
        \includegraphics[width=\linewidth]{Games/StoryLaneTheater/Images/StoryLaneTheater1Image3.png}
        \caption{Story Lane Theater 1 - Screenshot 3}
    \end{subfigure}
    \begin{subfigure}{0.45\textwidth}
        \centering
        \includegraphics[width=\linewidth]{Games/StoryLaneTheater/Images/StoryLaneTheater1Image4.png}
        \caption{Story Lane Theater 1 - Screenshot 4}
    \end{subfigure}
    \caption{Screenshots from Story Lane Theater 1}
\end{figure}