\chapter{Story Lane Theater 3}

% \begin{figure}[H]
%     \centering
%     \includegraphics[width=\textwidth/2]{./Games/WriteAway/Images/WriteAway2CD.jpg}
%     \caption{Write Away 2 CD}
% \end{figure}

The third of the Story Lane Theatre games published and released by The Lightspan Partnership for the PlayStation 1.

Story Lane Theatre 3 features two video programs:

\begin{itemize}
    \item Koi and the Kola Nuts
    \item Mose the Fireman
\end{itemize}

\clearpage
\newpage

\section{Koi and the Kola Nuts}

\subsection{Audio Summar}

\subsection{Transcription}

Come on in, settle back. It's curtain time at the Story Lane Theater. You've never seen a stage like this one. Anything can happen here. Today we're going to take a journey with the son of a chief in the African folktale called "Koi and the Cola Nuts." But before we set off, we're going to visit with the artist who did the pictures you'll see in the show. His name is Reynold Ruffins and he lives on Long Island, not far from New York City. He's been drawing almost his whole life, and today he makes art for things like books and magazines. He's going to tell us what it's like to be an artist and how he made the pictures for "Koi and the Cola Nuts."

"I've always lived in New York. I worked in Manhattan, but the last few years I have come out to Long Island. It's much quieter, and for the first time, I have my own studio arranged pretty much the way I would like it to be. I'm surrounded by the things that I like—doodads and knickknacks and little things that I can't throw out. I love masks; they're just beautiful, beautiful art.

The greatest thing about being an artist is, I think, being able to see light on a leaf, water in a glass, a sense of the time of day, the quality of light, the way shadows fall in the corner of a room—just somehow being observant. It enters through your eyes, gets knocked around in your head, and is screened through all the experiences you've had. But it's getting out, down through your body, and out your hands that makes it real.

When I was first asked to do 'Koi and the Cola Nuts,' I thought, what a wonderful idea. It takes place in an interesting setting in Africa. When I was a child, I never saw any depiction of African-American children, so I welcomed the opportunity to create images of life in Africa and to share that image with all children. Africa, being a very big and diverse place, I went to the library and got many books about African architecture, animals, and foliage that you might find there. I referred to them and then changed and designed in order to make what I thought was a more interesting visual.

I made several sketches for Koi. I wanted him to be both young and distinguished, a little hoity, and able to still keep his personality when times didn't go well for him. Ideas can come from a wide variety of places. He, in particular, I think, grew out of my interest in masks. I tried to stylize his features—the large eyes, long nose—so that they would be a connection to the masks.

There are so many things to think about when doing a large piece of work like this. You have to plan it out, make sketches, and I make a drawing first. What I want to do is transfer this finished sketch to this piece of brown paper. I'll do that by using a little tracing paper so the line will be transferred from the tissue onto the colored paper that I'm working on. In this case, it'll be a nice brown-colored paper. I'll put in the details. You know, that's not realistic, but it represents hair texture. I can add color where that's necessary. This is a pattern or design often seen in African art, as kind of black and white triangles.

Let's give you a little character. Koi's eyes reveal so much. This is a scene of Koi talking to the ant that he's met along the way. I have his eyes focused on the ant because he's small; he's got to get close, and he has something to say. I want him to look concerned and interested. Just darken some areas to give features and expression. Look pretty good.

Now, what I do is cut this out very carefully, watching the line. There we are, just about done. You're coming to life, Koi.

Now that I finished cutting it out, I added it to other elements of the story. Using these cutouts, I'm able to arrange and rearrange figures in different scenes. This is a night scene, so I made the moon a big moon to give it scale. Palm tree and Koi, he's very concerned, talking to the ants. How about you're so sad, Koi? We'll have a few tears. That's a nice composition for that. How's that, Koi? You like that? You'll feel better later; you'll see it all comes out okay in the end.

Like many other fairy tales, 'Koi and the Cola Nuts' is a test. Someone is being tested for their ability. Koi learned that if you're helpful to others, if you're kind to others without being asked to be, if you do good because you see that it's necessary, then good will come back to you. Being good is its own reward, and somehow if you're good, good will return to you.

And now, let's watch 'Koi and the Cola Nuts.'

Come sit down. I will tell you a story that comes from Africa. It is dawn in the west of Africa and the sun is a big orange-red ball slowly rising over the horizon. Feel the warm breeze that blows. The rains have just come and the earth is full of promise. Flowers bloom and banana trees bear fruit, and the yams and rice grow well in the fields. Smell the beautiful fragrances. Look, the animals gather at the river. Far in the distance, there is a village with smoke from the cooking fires making a trail in the sky. Listen, you can hear the sound of the drums. The drums spread the news that the chief of a village has died. All of the people and all of the animals from miles around hear the message of the drums: Chief Sedaka is dead.

A meeting is called, and all the elders of the village gather to decide what to do. They decide that the wisest man of the village will divide the chief's possessions among the chief's sons. So the wise man counts out to each of the sons so many goats, so many cows, so many tusks of ivory, and so many pieces of gold. The wise man finishes. But here comes Koi, the chief's youngest son. He's been hunting and no one fetched him when the wise man divided his father's possessions. There's nothing left for Koi, not even a tiny chicken bone.

Since this wise man is wisest when it comes to avoiding work, he does not bother to redivide the chief's possessions. No, it's too much work. The wise man looks around and he sees a small, sickly kola tree. "Ah, Koi," says the wise man, "and for you, we have a kola tree over there."

"What do you say, man?" cries Koi. "My father the chief dies, and you only give me this kola tree?" Koi is a proud young man. He grows angry, very angry. He cannot abide the slight of the wise man. "I will go to a land where I am treated like the son of a chief," Koi says. Koi picks all the kola nuts from the tree and wraps them into a mat. He ties the mat into a kente, swings the load onto his back, and leaves the village. "I will never return here," Koi tells the people of his village. "You do not know how to treat the son of a chief."

Koi walks for nine days with the kente on his back. The kente is a heavy load, but the jungle makes good music for him to walk by. The birds sing and the chimpanzees play hide and seek amid the elephant grass. Koi watches the zebras galloping across the veld. He walks gently past a sleeping lion and lioness, careful not to awaken them. Koi comes to the foot of a mountain. It's a big mountain with a peak hidden by clouds. He climbs the mountain with great difficulty, the kente getting heavier with each step. Finally, he arrives at the top. He sees the lush valley beneath him. There's a wide, silvery river curling through the valley.

"This will be a good place," says Koi. "Perhaps the people there will know how to treat the son of a chief." Koi makes his way down the slope. He sees a large snake, but the snake does not see Koi. The snake looks left, the snake looks right, and then he slithers along.

"What are you looking for, friend snake?" says Koi.

"Oh, it's terrible," says the snake. "My mother is ill and she needs kola medicine. I must find some kola to make the medicine that will make her well."

"Look no further," says Koi. "I have kola nuts. You may have them. Take them, make the medicine that will make your mother well again."

"Oh, you've saved my mother's life," said the snake. "Thank you so very much."

"It's nothing," says Koi. "Go quickly now. Your mother needs the kola medicine." Koi walks down the mountain, happy that he can help friend snake.

Koi sees an army of ants marching in an endless column as wide as a zebra stripe. Koi steps aside to let the ants pass. As the column passes, he hears a tiny high-pitched voice.

"Do you know where we can find a kola tree?" asks the leader of the ants. "We made a most unfortunate mistake. We ate the forest devil's kola nuts, a whole basket full, and we must replace them, or else the forest devil will trample us with his gigantic feet."

"How many do you need?" asks Koi.

"He says he wants as many kola nuts as he has fingers and toes. Let's see, I have six feet and four toes on each foot."

"The forest devil is a person," says Koi. "He has ten fingers and ten toes. You need twenty kola nuts."

Koi takes twenty kola nuts

from his kente and gives them to the ant.

"Thank you, thank you," says the ant. "You have saved us from being trampled by the forest devil."

Koi continues down the mountain. He walks along the edge of the silvery river and he sees a crocodile. The crocodile's large jaws are gaping wide. Koi steps quickly around the crocodile, but the crocodile does not move. Koi watches the crocodile from a distance, but the crocodile does not move.

"What is wrong, friend crocodile?" asks Koi. "You seem in great distress."

"Oh, it's terrible," says the crocodile. "I have a bone in my throat. It's stuck sideways. If I try to eat, the bone goes up and down, up and down, but it will not go down. If I drink, the bone stays stuck in my throat. I am dying of thirst and hunger. Oh, my, my."

Koi goes to the crocodile and looks inside the crocodile's mouth. He sees the bone stuck sideways. Koi reaches into the crocodile's mouth and carefully takes out the bone. Then he takes a kola nut and gives it to the crocodile.

"The kola nut will make your throat feel better," says Koi.

"You have saved my life," says the crocodile. "Thank you, thank you."

Koi walks along the bank of the silvery river and suddenly, he hears the sound of people shouting. He moves through the high grass and sees some people at the water's edge. They are all yelling and waving their hands.

"Why are you so upset?" asks Koi.

"Don't you see," says a man pointing to the river. "Our queen has fallen into the river. The river is full of crocodiles. We cannot swim out and rescue her because the crocodiles will eat us. Oh, our poor queen."

"Don't worry," says Koi. "I will call my friend the crocodile. He will help."

Koi calls the crocodile, and the crocodile swims to him. He tells the crocodile to swim out and rescue the queen. The crocodile swims to the middle of the river. He nudges the queen with his large nose, and the queen grabs hold of the crocodile's nose. The crocodile swims to the bank with the queen holding tight to his nose. The queen gets out of the water and thanks Koi.

"You have saved my life," she says. "Come to my palace and I will reward you."

The queen orders her ministers to make a great feast for Koi. The musicians play music and the people dance and sing. The queen takes Koi to her treasure house and opens the door. The treasure house is filled with gold and ivory and precious stones. The queen tells Koi to take as much as he wants.

"Thank you," says Koi, "but all I really want is a kola tree to replace the one that I gave away."

The queen sends her ministers to find the finest kola tree in the kingdom and they plant the tree in Koi's garden. The queen and Koi become friends, and Koi decides to stay in the queen's village. The queen treats Koi like the son of a chief, and Koi is happy. Koi's kindness is returned to him many times over, and he lives a long and happy life in the land where he is treated like the son of a chief.

\subsection{Credits}

\section{Mose the Fireman}

\subsection{Audio Summary}

\subsection{Transcription}

Come on in, settle back. It's curtain time at the Story Lane Theater. You've never seen a stage like this one. Anything can happen!

[Applause]

Today's story is called "Mose the Fireman." It takes place in the 1800s in New York City. New York was a much different place back then. Most of the buildings were made of wood, so in crowded neighborhoods, fires could be a big problem. And that's where Mose fits in. But before we get to Mose's story, we're going to stop off at this old firehouse, once known as Engine Company Number 30, and now known as the New York City Fire Museum. The man you see inside the museum is Don Cannon. He's a volunteer fireman, a historian, and an expert on the ways people have fought fires from the beginning of history to the present.

"I've always felt that fighting fires and studying and learning how to discipline myself and learning how to work with others and going through all the tough times in the street and in buildings is kind of like the price of admission to a community of people who you really know are good people. All firefighters are supposed to work in teams. You need each other to stay alive. You're working together towards a common objective, and you each depend on the other to complete the task together. For any firefighter, the height of his or her career would be to save a life, and they can go on with their lives and live a good life themselves."

[Music]

"I was really delighted the first time that I read the story of Mose. He was such an attractive figure to me, tough, rough, and ready, with a heart of gold, loyal and energetic. Mose represented a stereotype of the Bowery boy, the newly arrived brash immigrant type. They were looked down upon by a lot of people as being a crude bunch of ruffians, but he demonstrated and represented to the larger community that these are good people, that they loved the city in their own way, and they would risk their lives to protect their neighborhoods."

"When fires broke out in Mose's day, the firefighters would be awakened by the deep, deep sound of the tower bells in the district where the fire originated. Mose's experience was very common: snug a bed, aroused by the bells of the fire alarm tower, jumping into his pants and boots neatly laid by the side of his bed, going out into the cold dark, running, hearing the footsteps of other volunteers running to the firehouse, pulling this thousand-pound hand-pumping engine, exhausted, racing to the fire. And then they had to find either a pond or a stream and connect into it through a hydrant or a plug with their leather hoses and start pumping. Each company tried to be the first engine in at a fire, and the honor of putting first water on a fire was a great one at that point."

"Fireman Mose would have looked at his world in much the same way that firefighters do today. I think that the people you work with or people he worked with were closely connected. They shared experiences, evoked the same kind of passionate loyalty and affection that fire companies do in Baltimore or Pittsburgh or Chicago today. I really think that whatever the time, whatever the place, there's a real understanding that rises above time, rises above place. There's a shared respect for life and for each other and for people. The sum total of everybody working together is quite astonishing."

"Thanks for your thoughts, Don. And now, let's watch the story of Mose the Fireman."

[Music]
[Applause]
[Music]

Excuse me, excuse me. I hate to bother you, but I just wanted to ask you if by any chance you'd ever heard of a guy by the name of Mose the Fireman. Moses Humph was his real name, and he was the greatest firefighter who ever lived. This is going way back now. In fact, this goes so far back that New York City was so new it still had a price tag on it. But seriously, this is a long, long time ago. It's even before subways were invented. Now, to begin at the beginning...

Mose Humph was born in 1809 in a tenement on the Lower East Side of New York, the same year Abraham Lincoln was born in a log cabin on the plains of Illinois. Anyhow, that winter there was a big fire in New York. Every fireman in the city was there trying to put it out. It burned all through the night, destroying block after block of wooden houses. It was very sad. You see, they didn't have too many fire plugs at that time, so they had to pump the water right out of whatever river or pond or puddle or whatever was close by. Anyhow, just as they were putting out the last fire, there was a big explosion. Boom! That blasted the top half of the building into a million pieces. Most of it landed in the icy waters of the East River, and everyone who had been trapped inside was given up for dead. But not long after that, one of the firemen from Engine Company Number 40 heard a cry from down by the river, and right there lying in a busted hogshead among the frozen cattails, they found a little redheaded baby. Well, the firemen of Number 40 decided to adopt the little guy right then and there, and because of the way they'd found him, they decided to name him Moses, after a guy who was also found floating in a river, although not in the East River and not in New York. And so Moses it was, although practically everybody just called him Mose.

So Mose grew up in a firehouse. He took his first bath inside a fire helmet, and they used a big coil of hose for his playpen. Now, as you already know, firefighting was a very primitive operation back in the old days. To begin with, they had only one piece of equipment called an engine, which was really just a big clunky water pump on wheels. Because all the firemen were volunteers in those days, whenever the fire bell sounded, they would all stop whatever they were doing, run to the firehouse, haul the old engine out, and then drag it to the fire themselves with ropes, which was no piece of cake. You see, it wasn't until many years later that they figured out they could get horses to do it. These guys were not exactly valedictorians. Pretty soon they'd let Mose ride to the fire with them, and he'd watch as the fire marshal shouted orders to everybody through a silver speaking trumpet. "Man the brakes!" he'd say. The brakes were the big handles that pumped the water. "Man the brakes! Watch that hose there! Step lively, boys! Can the mongoose!" And the men would pump up and down on the brakes as fast as they could, and this would suck the water out of the wooden pipes that ran along most of the main roads and shoot it out the other end. It was hard work.

Well, Mose grew up fast, and pretty soon he was right in there with them, hauling the engine. And not long after that, he was so big and strong that they'd just let him pull it all by himself. The engine at Company Number 40 was called Lady Washington, after George's wife, and there wasn't a more beautiful piece of equipment anywhere. Whenever the fire bell sounded, Mose was always the first one ready to haul her off. You see, Mose was never one to fool around with the stairs. He'd just throw his boots out the window and take a flying leap right after them, putting them on as he fell and landing without a hitch. It got to where he could jump out the window, shave, comb his hair, brush his teeth, eat some oatmeal, read three chapters of a good book, drink some coffee without spilling a drop. Mose was some kind of fireman.

Now, I wouldn't want to give you the idea that Mose and the Bowery Boys of Company Number 40 were the only ones fighting fires in New York. There were fire companies all over the city, and every company thought their engine was the best, so I guess you could say they had a little rivalry going on between them. Whenever a fire alarm went off, three or four companies would all race to the fire at once, hoping to be the one to tap the fire plug and have the honor of holding the hose and putting out the fire. One time, two companies got into a brawl over a plug while the very fire they should have been putting out continued to burn. Now, Mose and Syy, who was Mose's best pal, couldn't believe what they were seeing. They had to do something.

"Who needs a fire plug anyway?" Mose shouted. "Come on, Syy, let's go." So Mose grabbed a pickaxe and just made a hole in the pipe itself.

"Syy, take the butt," he shouted. Syy grabbed the nozzle and pointed it at the fire, and then Mose stuck the other end right into the hole he made in the water pipe and began pumping the engine brakes all by himself. The water shot out with such force that the fire was out in no time, and Syy had to hose down the brawling companies just to cool them off.

    [Music]

Now, as you probably already figured out, the Bowery was kind of a wacko place. There were live pigs running wild through the streets, rooting around to eat whatever scraps of food they could find. And whenever you turned, there were chimney sweeps and vendors of all kinds selling onions and whatnot, and there were guys with carts that sold steaming hot yams and fresh apples and baked pears dripping with syrup. I'm getting hungry just thinking about it. Somebody run down and get me a canoli! And each of these characters would have a song that they would sing, like the guy who sold clams would sing: "My clams I want to sell, my clams I want to sell, twenty-five cents a hundred for good clams." That kind of thing.

But the Bowery Boys loved the Bowery. That's because they were Bowery Boys, and they stuck together and took care of each other, and that's just how it was.

\subsection{Credits}