\chapter{Every Child Can Succeed 2}

\begin{figure}[H]
    \centering
    \includegraphics[width=\textwidth/2]{./Games/EveryChildCanSucceed/Images/EveryChildCanSucceed2CD.png}
    \caption{Every Child Can Succeed 2 CD}
\end{figure}

The second of the Every Child Can Succeed games published and released by The Lightspan Partnership for the PlayStation 1.

Every Child Can Succeed 2 features three video programs:

\begin{itemize}
    \item Staff Development
    \item Instructional Leadership
    \item School Leadership
\end{itemize}

\clearpage
\newpage

\section{Staff Development}

\subsection{Audio Summary}

Staff Development - This segment illustrates how effective staff development activities are practical, and concerned with instruction and conditions in the participants' classrooms.

\subsection{Transcription}

Teaching can be a lonely profession. Isolated in their own classrooms, teachers may have few chances for new learning, for growth, for working with colleagues.

"I think the only way that educators can truly make a difference in the lives of all their children is if within the building they have a sense of support from one another. If you're out there on your own you might do really great for a while, but you're eventually going to burn out because you can't do it on your own."

At the highly effective schools we visited, teachers and administrators learn from each other through staff development. It's a process that can take many forms - a principal demonstrating an effective lesson for a new teacher, an informal discussion around the lunch table, or a session with an expert from outside the school.

This program won't give you a formula for doing staff development in your school: each school is unique. But we are going to show you the characteristics of staff development at some highly successful schools.

Welcome to Bennett Q Elementary, Inglewood, California. You won't hear just educational theory from Principal Nancy Ishinaga. Her school practices staff development that is focused, specific and practical. At this lunchtime meeting, Nancy Ishinaga, and some of her first-year teachers, talk about improving student writing.

Informal staff development meetings like this one at Bennett Q are one way for teachers to exchange practical advice and solve problems.

"You know, I give the teachers a chance to figure out problems. They they come to me with problems and generally with solutions. I can't do it all by myself. I don't have all the answers, so I use my staff to train the others as well. I just kind of create the situation in which this can happen."

Here's another form of staff development. As Melanie Crawford teaches a lesson to her class, three new teachers watch. The purpose: for the new teachers to sharpen their observation skills and begin to model good teaching. After the lesson, Melanie Crawford meets with the new teachers to find out what they observed in her class.

Principal Esther McShane attends as many of these sessions as she can. She also makes it a priority to spend time with individual teachers, observing them, giving them feedback on what she's seen.

"The bottom line is student achievement, and we need to continue to work on it. We need to focus in an area, we need to grow in that area, and as we see growth in that area, we need to identify another area so that we can continue to build on techniques, strategies that can equal success for children."

What you're watching now is the end result of previous staff development. Teacher Diane Thompkinson was looking for active listening techniques. A staff development expert from outside the school showed her how to use an instructional technique called "story painter". The children paint a picture to present their vision of what a story is about.

"You get a lot better listening skills when they try to imagine what is going on in the story, and particularly when they know they're going to have an opportunity to paint their favorite part, and that we are going to work with a story and describe that chapter."

Because story painter works so well in Diane Thompkinson's class, Joan Doerr of the Academic Achievement Office has come to Alki school to share the process with others. In workshops like this, teachers are learning new strategies for getting students involved.

For Diane Thompkinson, these opportunities to share teaching methods are what staff development is all about.

"The more strategies and methods you have, the more children you reach, and the more interesting it becomes. And I think that's a key thing we're finding nowadays - is letting children know there is more than one path to the house, and they're all acceptable and appropriate. And so that you have children taking risks in learning. You have you have a lot more staff taking risks, you see, so there's so many things at work and it's exciting to try out a wider variety."

Gary Tubbs is a hands-on principal. At Beacon Hill School, Tubbs spends a large part of every day on staff development. He visits classrooms, gets involved with his teachers and students, and even teaches an occasional class. Here he demonstrates a lesson featuring integrated learning to some new teachers. As the teachers watch, Tubbs leads the class through analysis of their lunch menu.

After lunch, the students analyzed the assignment: 'The Spicy Bar'.

"I like to get back into teaching because it feels like the right thing to do. It keeps my skills sharp so that I can be a better supervisor for the teachers on my staff. It raises my level of credibility with the staff because I'm willing to get up and teach at a moment's notice and work with their kids and get right down on the carpet with the kids, and I'm able to then go in and evaluate a difficult situation because I've been there myself. I can help teachers learn to become self-evaluators because teaching is close enough to me, I've done it recently enough that I'm able to talk at a level that I don't feel like I'm talking down to them."

At the schools we've seen so far, teachers are encouraged to grow professionally. They are supported by their colleagues, and the principal gets involved in staff development by recognizing what teachers need and providing opportunities for growth.

It's an average day for these teachers at New Suncook School in Lovell, Maine - up to their elbows in mud, searching carefully through the muck to see who can find the most interesting bugs. Well, okay, they don't do this every day. Today, Lori Kinsey, a local conservationist, is leading the teachers and their principal on a tour of a bog near the school. She's helping the teachers prepare a unit on wetlands.

"The teachers really, contractually are obligated only to be within the school from 7:30 to 2:30, but certainly a school like New Suncook, and I think in a lot of schools, where there's really committed people, these people are out on their own, they're involved in their staff development activity, but they can relate back to their classroom and can relate to trying to make better things for kids in classrooms. So it's a situation beyond what you see - as far as the people here is the reality that they're going back tonight and still having to prepare for tomorrow. And this happens all the time."

What do you think these teachers leraned about each other, and about themselves today?

A day in the bog - it's no picnic, but these teachers enjoyed it and grew from the experience.

"It just makes the year go better for the kids and also for me, so anything I can do to help my year go better is certainly worthwhile also."

Staff development is a means to an end - the end result being a successful school with successful students. In the schools we've just seen, staff development is an integral part of the school's mission and culture. It is practical, ongoing and full of opportunities for teacher growth. But most of all, it's about improving the instruction for every student in every class.

"I really have a chance to grow professionally. I had a really rich experience in this building because of that. And the more I learn, the more I try, I demonstrate, I share, the more other people share with me, the richer the whole school gets. So that's been really nice."

\section{Instructional Leadership}

\subsection{Audio Summary}

Instructional Leadership investigates how the principal and staff of a school can work together to develop a clear mission statement that focuses on the academic success of every child.

\subsection{Transcription}

"We're only here for one reason and that is to educate children. And I'm responsible for the education of every single child that enters. They have entered my sphere of life, so I am responsible for it, and I need to provide the climate, the environment where learning can take place."

At each highly successful school we visited, outstanding leadership is critical to the school's success. In this program, we'll examine instructional leadership. Another program will look at other aspects of leadership. In high achieving schools, the principal is usually the instructional leader. We wanted to know how do these principals improve instruction, how do they motivate teachers and students, and how do they create a culture of success in their schools?

"Thou shalt educate every child?"

Imagine a school called Euclid Avenue Elementary in East Los Angeles. It is dirty, dangerous, one of the lowest achieving schools in the city. Now imagine that someone calls and says you are to leave your successful school and become the new principal at Euclid. It happened to Esther McShane.

"I actually thought it was a joke. I resisted, I said I'm not coming, and I was told I'll meet you there at 10:30 the next morning. And it certainly was a shock. I didn't want to be here, I wasn't looking for another challenge."

Ruth Johnson of the California Achievement Council understood the problems at Euclid:

"There was racial tension, there was tension of teachers that were not talking to each other in the halls, there was no focus on looking at kids and achievement. It was a fractured school, very fragmented."

Esther McShane developed a clear mission for her new school: academic success for every child. That meant cleaning up the campus, improving instruction, and increasing student achievement.

"I have made a difference because I have empowered people. They didn't believe that they were excellent teachers, they didn't believe that they were capable of raising children's scores, they didn't believe that they could have law and order in a school."

"The culture of the school has changed. The school is now focused on student achievement, there's a curriculum focus, teachers are practicing and implementing new ways of teaching and learning, and that to me is critical if the school is going to move on."

"I wouldn't trade this school for a mint now, not for a mint."

Each of our principals started with the belief that all students could succeed. That belief became the focus of the school's mission - a mission these principals communicated clearly and relentlessly.

"I think that if someone has the belief that these kids can absolutely learn and are capable of doing everyting that you want them to do, then you set about doing that and if you continuously monitor it, continuously talk about it, continuously get your philosophy out, with parents, with communicity, with teachers and with students, then everything will be fine."

"And that sums up the lesson for today."

Managing instruction is one of the effective principal's most important jobs. We've come to Bennett Q Elementary in Inglewood, California to meet a principal with a knack for designing and implementing curriculum. Her name is Nancy Ishinaga.

"We have worked out a system where everybody goes through, everybody is taught specific things that make them a successful student, and it's what every teacher does."

The curriculum at Bennett Q is sequenced, beginning in kindergarten up through the sixth grade. Every child progresses from blocks to abstract concepts like diameters and fractions. Under this system, each grade builds on the success of previous grades.

Nancy Ishinaga believes in her sequenced curriculum. She carefully monitors to make sure it's being taught and being learned.

"In other words, we have a very well-defined curriculum, and I monitor what happens in the classrooms by periodic tests which the teachers all give, and every teacher in every grade level gives the same tests. They turn in the results to me and we go over them together to see where strengths and weaknesses might be, where changes in the tests or changes in the curriculum need to be made."

"Our test scores she cares about. She'd call you in and say 'have you thought about working in this area', 'watch someone else teach vocabulary and you'll do a lot better'. When I first started, my vocabulary scores were low, and she said, 'what do you want to do', and I said okay, I'll really make an emphasis next year, I'll work extra hard on vocabulary with my kids, and so it works."

"My job is to make sure that the teacher is given the wherewithal to be successful, and the teacher's job is to give everything to her kids to make them become successful, so there is this internal consistency, and I would like to reiterate the fact that we do have a system so that any school which wants to succeed can do this. It's not a matter of personalities, it's a matter of having a system by which we educate our kids."

"I'd like to overwhelm them through instructional excellence, but I'm not above winning through intimidation."

Effective school leaders create environments that encourage learning and inspire teachers. When Louise Smith was principal of Rice School in Dallas, the sheer force of her personality created a world inside these walls that left no avenue but achievement.

"I have very high standards for myself first, and for everyone else around me, and I will not lower those standards for anyone."

"Well, she is dogged in her pursuit of excellence. She lays the cards on the table when she interviews you. If you cannot meet her expectations, there are no hard feelings, but you'll have to find another place to go because the children are her priority, even though you may develop a wonderful working relationship with her, and be her friend, if you're not serving the children to the best of what she knows your capabilities to be, then you don't belong here."

"I'm very uncomfortable with mediocrity. If I find people who are satisfied with mediocrity, then I stay away from them because that's my discomfort zone, when I see people who are just satisfied with just status quo. I'm not satisfied with status quo. It has to be way above and beyond status quo to satisfy me."

Effective instructional leaders are highly visible around their schools. Nancy Ishinaga at Bennett Q plays many different roles - crossing guard, cafeteria worker, hallway monitor, and always, always instructional leader of the school.

"She sets the tone for the whole school. She's a hard worker from crossing guard to cafeteria duty, to, you name it, Nancy does it.So she sets the tone by herself working extremely hard. And again, she expects for her teachers to do well. We're held accountable for what happens in our classroom."

"My job is not to be the big boss or the big honcho who orders everybody around. Granted I am a little bossy and I'm not that patient, but I think I have a very good idea and a feeling of what a school should be like, and I try to do that."

Besides setting a good example, effective leaders also motivate their students and teachers to succeed. This is not a pep-sessiona at Vann School, this "Spirit Day" - a celebration of teaching and learning. Doris Brevard conceives "Spirit Day" to motivate and honor those students and teachers who live the school's mission: academic success for all students.

Across the country from Vann is Beacon Hill School in Seattle. Like other effective leaders, Principal Gary Tubbs supports professional growth for his staff. Here he uses a hands-on approach by demonstrating a lesson for a new teacher.

"It raises my level of credibility with the staff because I'm willing to get up and teach at a moment's notice and work with their kids and get right down on the carpet with the kids. And I'm able to then go in and evaluate a difficult situation because I've been there myself. I can help teachers learn to become self-evaluators because teaching is close enough to me, I've done it recently enough that I'm able to talk at a level that I don't feel like I'm talking down to them."

Gretchen Jacobson teaches at Beacon Hill. She admires Gary Tubbs for his leadership and his support.

"He's incredibly supportive. Really knows how to motivate. If I was having a problem in my classroom I'd come and say, 'Okay Gary, this is going on,' and he would ask me questions and get me to reflect on what I'm doing right, what I think I need work on, and then how can we solve that? And he sometimes give solutions too that I would go back to my classroom and try. So he was not only encouraging, he was a good problem solver too."

"You find the principal really believes in teacher support."

Like other instructional leaders, Vann Principal Doris Brevard supports her teachers in many ways. For example, she has enforced strict discipline at her school for over 20 years. The policy is designed to maximize the amount of time teachers have for instruction.

"The principal must provide the atmosphere, and then it's the teacher's job to teach. But if that teacher can go into his or her classroom and teach 35 minutes of the 40-minute period, then something will be accomplished. But the principal must see to it that the discipline throughout the building is such that the teacher can teach the 35 minutes."

"I know that when I go in my room I will have an atmosphere in which I can teach, and I think one of the one of the things I always remember when I first came to Van, she said to me, 'You can teach whatever style you like, your style is your style, and as long as you're teaching and children are learning, that's all I ask of you.' And she also told me that when I go in my room to teach, I will have an atmosphere in which I can teach, and if I don't have that atmosphere then that's her job to provide it for me. And I have, it's always been provided for me."

Besides creating orderly environments for their teachers, effective leaders build cohesion and cooperation. Alki principal Pat Sander encourages teamwork among her teachers. She creates staff development opportunities to get everyone talking and sharing their knowledge.

"I think in the traditional school the principal is the boss, is the person who okays what you want to do, the person you must see, you must clear things with. At Alki, the principal is a team player - a person who works collaboratively with you and other staff members to see that what you want to do with kids occurs, facilitates, fosters, supports. That happens here every day."

At the New Suncook School in Lovell, Maine, parents are meeting with Principal Gary McDonald. Their children aren't even in school yet, but McDonald believes it's crucial to get parents involved early. He wants them to understand and support the school's mission.

"Parents to me, and to the entire staff here, are seen as being tremendous value. They're welcome at all times, and we try to provide them with the knowledge and the opportunities that they can really feel an integral part of the building, and they raise excellent questions that cause us to, again, focus and sometimes give a perspective that's extremely valuable to us."

Valerie Wilfong has a son at New Suncook School, and she volunteers as a teacher's aide. She got involved with the school after meeting Principal Gary McDonald.

"I mean right from the very beginning we were welcomed in before Christian started kindergarten, and very open policy, 'We want you to be a part of what's happening,' really a partnership happening here, and that was wonderful for us."

Strong instructional leadership is one key to a highly successful school. Instructional leaders are willing to do anything it takes to achieve academic success. They develop and communicate a mission, promote quality instruction, and create supportive environments for teaching and learning. Above all else they are single-minded in their devotion to success for every child.

"The passion comes from the light that comes on in the little boy's eyes when you finally get something, or the little girl who eventually learns that 'I'm just as smart as anybody else.' It sounds philosophical and maybe it sounds corny to some people, but we're here for a reason, and I don't want life to come to an end and feel like I haven't made a difference in the lives of the children."

\section{School Leadership}

\subsection{Audio Summary}

School Leadership explores how to create a positive school climate that permits the principal to conduct systematic monitoring without causing teachers undue stress.

\subsection{Transcription}

If you have a few seconds we'll show you everything we learned about school leadership. Oh you want the long version? Stay tuned.

At each highly successful school we visited, outstanding leadership is crucial to the school's success. In this program, we'll examine school leadership. Another program will look at instructional leadership. In these schools, the principal was the school leader. We wanted to know what do these principals do that makes them highly effective school leaders? What are their priorities and what characteristics do they share?

"Mr. Housener, the principal would like to see you after class."

In highly effective schools, principals are closely involved in selecting teachers and replacing them if necessary. Louise Smith is the former principal of Rice School in Dallas.

"I tried to interview every teacher who entered this building because I needed to let them know what the philosophy was of this school and what we were about, and if they wanted to buy into that philosophy, and if they wanted to buy into that energy level that we were going to be involved in here because it takes an awful high energy level in order to do this job. You see, we're consistently working. There's no time to just relax."

Dana Brooks was one of the teachers Louise Smith hired after an extensive interview.

"It's the toughest interviewing process that I have gone through, and I've been in three school systems now. If you have to write an essay, you go through extensive interviews, but it's worth it, it's worth it."

Any teacher who interviews with Louise Smith should listen carefully - they will hear about her high expectations. And if teachers don't meet those expectations, they'll be hearing from Louise Smith again.

"I call that teacher in and I just tell them, when we interviewed, I told you the kind of place that you were coming to. Now, if you want to change your mind now, then fine, we'll talk about this, but my expectations are still the same for you and everyone else around here. We've got to get the kids where they need to be, and you've got to carry your weight just as everyone else is around here, and that that is something that I will not veer from at all."

And what happens if teachers still can't meet Louise Smith's tough standards?

"I never actually had to get rid of a teacher, but the way we were able to get this school on such a high standard, some teachers who came saw that they couldn't keep up, so they asked to leave. You see, this is a very, very busy place, so, you know, some people can't keep up the pace, and you can see that very quickly. If you're here, you can't slough off around here - you have to work. And if you're not willing and able to do that, then most of them asked to leave."

After 23 years at Vann's School, Doris Brevard has a reputation as a maverick and risk taker. She is an unwavering champion of her students and teachers.

"From the very beginning, she has indeed been a risk taker. That is, that she puts the children at Vann School above the system all the time. She is constantly thinking and aware of what's good for the kids at Vann."

Here's an example: at one time, Vann was a pilot school to test the effectiveness of the Lippincott reading series, which was based on phonics.

"And we found that at the end of the first year, the children in the Lippincott series could read just about anything that they picked up, so that gave us the idea that this was something that we wanted to use."

Unfortunately for Vann, the school board later dropped the Lippincott series and told Doris Brevard she could not use it. In most schools, that would be the end of the story, but Vann is not like most schools. It has Doris Brevard.

She organized parents, and with their support, took her argument to the board. The board reversed its decision. Doris Brevard does what is best for her school, regardless of the consequences. Like most effective school leaders, she's a rebel at heart.

Think of a school as an electrical circuit. Effective school leaders are plugged into that circuit. They get involved and monitor what's happening. Classes from Alki School in Seattle are on a beach trip today, and principal Pat Sander has come with them. She didn't have to. She has plenty to do back at school, but her priority is spending time with students and teachers.

You'll also find Pat Sander at staff development meetings. She wants to see and hear what her teachers are learning. And sometimes you'll even find her in her office. Here she is, busy monitoring her students' test scores and entering them into her computer.

"And I do that for a couple of reasons. One, so that I can be a support and a resource to the teachers as we start to look at the instructional approaches and strategies. And two, so that if a parent walks in and asks a question about their child, that I can be knowledgeable and be able to give them some information back also."

Pat Sander is plugged into the circuit. She is a watcher and a listener. All of this monitoring brings her closer to teachers, and most important, to students.

"And I think that every child has a strength and what we have to do is find that strength, capitalize on that, and then start to bring about the educational achievements in other areas that may not be the strengths of that child."

Effective school leaders are highly visible around their schools. You won't find Esther McShane in her office doing paperwork during school hours. She hates paperwork. She wants to be close to the action.

"I do not spend my time at all during the day doing paperwork because I don't need to do paperwork. The action is in the classroom, the action is in the yard, the action's in the eating area. I have high needs to be with my students and my staff."

"She's constantly out here talking to children, you know. You don't see principals doing that. She's on the playground before school, during recess, at lunchtime. She's picking up trash, and when I see her doing this, well by golly, I'm going to pick up trash too and I'm going to question children and I'm going to expect for my children to excel."

"I think she must stay here 12 hours a day at least, and she is here and she's out there working. In the past, most administrators that I saw, that I have worked with, virtually locked himself in their office and had no contact with children. She's out there and she's hustling, and she cares about the kids, and she tries to get everything available at all possible for our kids."

It takes an enormous amount of energy to do this job. It should say so right on the application. Esther McShane is driven. She believes there's no other way to achieve excellence.

"But if you have energy and you have high expectation for yourself and for children, it just comes - it formulates itself, and needs to be displayed -  that behavior needs to be every day. Every minute. I mean we're on stage, and we're on stage for children."

Effective school leaders support the front line troops: their teachers. This is Beacon Hill School in Seattle. Principal Gary Tubbs understands what can happen if teachers don't get any support.

"I think that that is the only way that educators can truly make a difference in the lives of all their children, is if within the building they have a sense of support from one another. If you're out there on your own, you might do really great for a while but you're eventually going to burn out because you can't do it on your own."

When a student has a behavior problem, Gary Tubbs might call a problem-solving meeting. He invites teachers, the school's counselor, and a family support worker to brainstorm solutions with him. Out of this meeting comes a plan to solve the child's problem - a plan that Beacon Hills educators agree on, and can implement together.

Effective school leaders also understand the importance of using outside resources to improve their schools. One of the most important resources is people, especially parents. At Alki Site Council meetings, principal Pat Sander listens carefully to what parents have to offer.

"Not one of us can do it alone. The parents, know, certainly, the most about their children. They know more than we do. And they have a different perspective on what's beneficial for their kids. The staff brings that educational background and professional experience, and hopefully I bring the administrative background that can maybe pull it all together."

Site Council Chairman Bill Staltzer is also an Alki parent. Together, Staltzer and Pat Sander worked to reorganize the school's computer lab, which was rarely used until last year.

"I would say probably of the 330 something children in this school last year, 250 to 300 of them, were on the computer every week, at least once or twice, okay? And that happened because of parents, because of my helping get it started, and there were parents down here supervising. We don't have staff to do it, you know, staff doesn't have extra time, we couldn't get a specialist, so it was done with parents."

Here's another characteristic that we saw in all of our principals: highly effective school leaders support the personal and professional growth of their teachers. At the new Suncook School in Levell, Maine, principal Gary McDonald is working with children in one of the multi-age classes. It was the school's teachers who first conceived the idea of organizing multi-age classes. Gary McDonald supported their efforts, and his teachers know they can count on him whenever they need help.

"I think the best thing is that his office door is always open. Nothing's too small for him, and, wow that's I think real unusual. I mean it can be just something that you think is really petty but he'll take time, and he has a good way of questioning you when you have a new idea, of taking the other side, but it doesn't make you defensive, he's a good devil's advocate. But he is, he's great."

The best school leaders take the extra step to support their teachers. Euclid teacher Geraldine Allen remembers a time when Esther McShane made an extra effort for her.

"When she first came to the school, I was working on a master's degree in education. She was always giving, constantly giving words of encouragement and breaks whenever possible. I was coming into the classroom, doing lessons with the students, knowing that I was doing things to enhance my career. She shared of herself, and I'll never forget that."

Strong school leadership is one key to a highly successful school. Effective school leaders personally select teachers. They take risks when necessary. They monitor their students and staff. They work hard, and they support their teachers. These principals tackle problems with courage and skill, and though they may have different styles and personalities, they all agree their job is to help every child succeed.

"My philosophy is that I think these children deserve a wonderful education just by showing up, and I intended to give it to each and every one of them, and I look at my product in return. What kind of students do I produce? And if I produce high achieving, happy youngsters who have the stamina to go out here and take advantage of any opportunity that's offered to them, then I have done my job."