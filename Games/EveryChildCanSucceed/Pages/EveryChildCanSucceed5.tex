\chapter{Every Child Can Succeed 5}

\begin{figure}[H]
    \centering
    \includegraphics[width=\textwidth/2]{./Games/EveryChildCanSucceed/Images/EveryChildCanSucceed5CD.png}
    \caption{Every Child Can Succeed 5 CD}
\end{figure}

The fifth of the Every Child Can Succeed games published and released by The Lightspan Partnership for the PlayStation 1.

Every Child Can Succeed 5 features three video programs:

\begin{itemize}
    \item The Vannn School: A History of Success in Pittsburgh
    \item New Suncook School: Achievement in the Woods of Maine
\end{itemize}

\clearpage
\newpage

\section{The Vannn School: A History of Success in Pittsburgh}

\subsection{Audio Summary}

The Vannn School: A History of Success in Pittsburgh explores why an inner city school has students with scores on standardized tests in mathematics and reading that rank the school at, or near the top of Pittsburgh's public schools.

\subsection{Transcription}

Good morning boys and girls. I want to welcome you to Vann Spirit Day. You've so often heard me say that I'm very proud to be the principal of Vann School, because to me Vann School is very special. We have three special ingredients - we have tremendous students, terrific teachers and outstanding parents. Therefore, with those three ingredients Vann School really is the greatest school on earth.

Vann School is part of the Pittsburgh public school system. You can almost see it from here located just beyond those buildings in a part of town known as the Hill District. The Hill District is a densely populated urban area. Like many inner city neighborhoods in America, the Hill District is mostly African-American and mostly impoverished, with a reputation for crime-ridden neighborhoods, substandard housing, and dysfunctional schools. Situated less than a quarter mile down the hill from Vann is another elementary school. It's the kind of school we expect to find in surroundings like these - a school with low achievement scores.

The Robert L. Vann Elementary School, however, has a much different record. Year after year for more than 20 years running, Vann's math and reading scores have placed it at or near the top of Pittsburgh's public schools. Demographic differences don't account for the difference in achievement between the schools. They both serve the same student population drawn from virtually the same projects and neighborhoods. Nor can we dismiss Vann's history of success as an anomaly. Just a short distance away, Madison Elementary School has consciously modeled itself on Vann, and in recent years has regularly matched or surpassed its sister school on measures of academic achievement.

Why should one school consistently succeed, where another serving the same population in the same geographic locale consistently fail? According to the parents, teachers and students of Vann's School, the formula for success is simple: good teaching in an atmosphere conducive to learning.

For Doris Brevard, Principal at Vann School, that formula begins with a clear routine and with discipline.

"If you are disciplined, then you're here to learn; you want to learn and it's so very easy to be taught if you have discipline. The majority of the children in the Vann area come from a single parent home. 90\% of our children are on free lunch, which means the household receives assistance. We try to make the atmosphere and the school conducive to learning, however we don't want it to be oppressive. We develop a routine for them because outside of school, there's no routine, and we feel that a routine is important for them to establish within the school building. The single parent home of course is so interested in survival they can't devote too much time to the home situation or to the education of the child. Therefore, we have to meet those needs when they come here within the building. I expect the children that come into Vann to respect the adults, but I also want them to respect themselves. I expect the children to do their very best in every written assignment, and every test, and every recitation. I expect the very best. A successful school is a school within which a child is totally educated, and every child meets his or her potential. Not all children are going to be A students or B students even, but if you're a good C student, then your accomplishments should be at the C level. The teacher keeps a constant check on the progress of the children, and I rely on his or her definition of whether or not the child has succeeded. We feel here if you can solve the problems, if you can get the children help at a very young age, then that will eliminate more serious problems when they get older. That's why we stress, in the beginning of the reading instruction, that the children master whatever is being taught before they move on. For so many years you heard constantly that the African-American child could not learn, and it was our aim here to prove that the African-American child could achieve just as well as any other child. We don't have any hidden secrets, we're just using the curriculum that's provided by the board, but our secret is we meet the needs of our children."

The Vann primary team teaches reading using the open court phonetic method. Their enthusiasm for the method stems from the fact that it seems to meet the particular needs of Vann School students. First grade teacher Gloria Mathis explains.

"The open court is phonetic, and I love it because they're not just learning words, they're not just memorizing, they're learning sounds. Everyone works together. If you have someone who's a little bit quicker than someone else, kind of pulls the group along and if someone's not quite catching on, it gives them a chance to see what's going on through one of their peers."

Teachers on the primary team work closely together to determine what works and what doesn't. Their common goal is to lay the foundations for academic success. They consciously strive to preserve a structured, disciplined classroom that nonetheless reflects warmth, friendliness and enthusiasm.

"I love teaching. I love being in the classroom. This is my heart, I wouldn't be anywhere else and I enjoy working with the younger children because you get the opportunity to reach them when they're just starting out. Their minds are open and they're willing to just accept and you can watch them grow."

The primary team's purpose is to ensure success, even if that means holding back students until they're ready to move on. Nadine Digger teaches the Early Learning Skills Program, ELS. Mrs. Digger has taught at Vann for 23 years.

"The ELS program stands for Early Learning Skills, and it's really a transition period between the kindergarten and the more formalized first grade. Most of the children who come through the Early Learning Skills program have been in kindergarten for a year, and for various reasons they were not ready to go on to the formal first grade. Sometimes it's immaturity, sometimes it's a minor learning disability that with one extra year getting the basic skills, they can go on then to be a successful student in the first grade. There are some skills that we have to repeat a lot, so I'm always looking for different ways to work on the same skills that the motivational level of the children stays high."

On her own time and with her husband's help, Mrs. Digger has created a series of computer-based lessons to enhance the learning experience for her ELS pupils. The programs provide a creative way to teach repetitive material.

"Academically we think it's working well, but probably the most interesting part has been the self-confidence that it's building in the children because for some reason, being able to work a computer makes them feel like they are really special people."

Mrs. Brevard regularly monitors classroom performance to ensure that achievement and learning remains the primary goal of all classroom activities.

"You have to be interested in children. You have to want to teach sound. You're not interested in just the paycheck. You're interested in achievement. The principal must provide the atmosphere, and then it's the teacher's job to teach. But if that teacher can go into his or her classroom and teach 35 minutes of the 40-minute period, then something will be accomplished."

"I find the most successful way that I can monitor teachers is very informal by going into the classroom participating in the class. I'm interested in the response of the children. If they're doing written work, I check the written work to be sure that it is correct, that it is the best that they can do."

"Mrs. Brevard sets an outstanding tone for the building. She establishes her expectations and we know that when we come here there are high expectations for both students and teachers, and we rise to the occasion."

Teacher Kathy Gallagher is also the Vann School librarian. Ms. Gallagher's enthusiasm reinforces learning and draws students like these first graders into the magic of reading.

"My main goal would be to teach them to love to read, to make them lifetime readers."

Marita Maloney continues to accentuate the basics, providing students with the tools they will need to succeed in the higher grades.

Children who leave the primary grades advan leave with a sense of achievement, success and self-confidence. In every class at Vann School, there is a conscious effort to reinforce a positive self image. Achievement, doing one's best is stress throughout the curriculum.

"So often the African-American child has a defeatist attitude. Therefore we stress that they are important, that they can succeed, that they are somebody and they can be somebody, and that's something that you have to teach every day in every lesson."

Performance is what puts Vann students on top on the primary grades, but how is this high level of achievement sustained in grades 3, 4 and 5? Mrs. Lipsman's fifth grade reading class shows how Vann's students rise to the challenge.

"I really believe that it's a question of high expectations more than anything else. We have teachers who are actively involved with the students, who have students actively involved with the learning process. I think one of the things that makes us very special is that we are consistent in terms of what we expect from our children, consistent from grades kindergarten through grades five."

"I always stress the fact that the success that Vann has achieved is really because of the teaching, because if we didn't have successful teaching of course we would not have the good scores and the good accomplishments."

In maths, Van's students consistently rank at the top of the Pittsburgh public school system. What puts Vann's math scores on top is good solid teaching at the hands of experienced professionals like Eileen Fabric in this fifth grade math class.

"I think the high scores are attributed to the children's motivation. They're highly motivated in math, they want to do well, they know they can and they want to succeed."

"I hope I'm laying a strong foundation for them in later life, hoping to prepare them for middle school, eventually for a secondary school, and ultimately for a lifetime."

Math lessons are reinforced by practical experience. Here, Ms. Fabric helps the children learn how to maintain their own savings accounts at a neighborhood bank. This type of involvement is part of Ms. Fabric's teaching style. Ms. Gallagher's emphasis on active dialogue with her students like these fourth graders is the key element in her teaching style.

"There should be some myster to it, and I think there should be some fun to it, and I think that children learn best when they're comfortable with you. I think you have to establish trust and I think you need humor in the classroom, and I try to incorporate all this into my teaching. I want them to be able to communicate with the written word, and in order for them to do that I think they have to experience many different types of writing. It gives them a chance to read their story orally, it gives them a chance to hear each other's stories, and it gives them a chance to understand what someone else has put into their story."

"Every child can learn, every child can succeed, and I think at Vann every child does."

Spirit Day at Vann School. At this assembly success is the only item on the agenda, with awards to students for academic excellence, for developing instructional computer programs on their own time, a rose to Mrs. Digger and her husband. And special congratulations to a student and his family on the child's improved behavior and achievement.

Fifth grader Dayron Deadrich is a good example of success at Vann School. Deyron is a good student who's also emerged as a school leader this year. Every day Deyron, his younger brother Kevin and their friend head for home through the streets of the Hill District. Home for Deyron is the projects, where retaining the values learned at Vann requires a conscious choice. It helps that Deyron and Kevin have strong parental support.

"With both of them I'm very, very, very pleased. I think that Dayron being an older brother is setting a really good example for Kevin to follow. I think that Vann is a good school. They do a good job with the children as far as discipline and if you don't have those type of teachers that they have to push the kids to excel and succeed and it would never happen."

Despite the fact that many Vann parents are fully occupied in the struggle for survival, the school has an active and supportive PTO. Mrs. Jenkins is this year's PTO President. Her daughter Leilani is in the second grade.

"PTO is there so we can help the kids. Without the PTO, they wouldn't have certain things that they could get like field trip money or maybe there's a field trip going to the museum or money funded for them just to get some books or something like that."

PTO activities like this treasure chest which rewards weekly achievement help make Vann a success. But parents don't hesitate to identify the real key to Vann's success.

"Mrs. Brevard, she's what makes it work. I see her as being consistent, which is a key when you're dealing with children, and she goes for no nonsense. You know she is in charge of her school and no one else. Mrs. Brevard and the wonderful staff that she has is what makes Vann work."

"I've known Mrs. Brevard for the entire 23 years that she's been a principal, and from the very beginning she has indeed been a risk taker. She is constantly thinking and aware of what's good for the kids at Vann. If central office comes out with a new initiative, or a new program, and she does not feel that it fits within our philosophy or our framework for these particular youngsters, she'll dismiss it and she won't allow it to be part of our curriculum or our program."

"Well I really think there's been too much emphasis that the principal is the most important element, whereas it is the teachers because if the teachers really aren't concerned about the school, if they really aren't dedicated, then regardless of what I do, whether I monitor them, I don't care what kind of atmosphere I produce, if they're not interested they're not going to take or play their part in the development of the school. My job is more or less to keep the teachers interested, to keep them involved, and to let them know that I appreciate what they've done, and I tell them that daily."

\section{New Suncook School: Achievement in the Woods of Maine}

\subsection{Audio Summary}

New Suncook School: Achievement in the Woods of Maine investigates how a small, rural school isolated from any significant sources of educational or cultural enrichment has adopted a number of innovative instructional techniques.

\subsection{Transcription}

Walk into a classroom at the New Suncook School, you're never sure what you might find - students in large groups, in small groups, working alone. The word that comes to mind is chaos.

"The initial impression is that it's [inaudible]. That there's a lot of noise, and what is going on, and it doesn't look like school like when I went to school."

Then there are the hallways, used for reading, instruction and even singing. But despite the noise, despite the apparent chaos, this is a school with one mission - to enable all students to become lifelong learners.

"I think that some people get hung up on the idea that the structure is the important part of what's happening and to me, what's most important is the interaction between teacher and child."

The atmosphere is unusual, so are the results. The New Suncook School is recognized as a model of teaching and a model of learning.

Dana Bean gives his daily weather forecast from this farmhouse right across the street from one of the highest achieving elementary schools in Maine. The New Suncook School is located in the town of Lovell, Maine, population 888. It's in the rural western part of the state, an area of chronic unemployment. About one-third of the students are eligible for free or reduced price lunch. Despite its isolation and the economic conditions of the area it serves, New Suncook School has earned a national reputation for excellence.

It's a reputation that started with the arrival of principal Gary McDonald nine years ago.

"We think we can make a difference, we think the teachers can make a difference and that we think that we can enable and create environments that help students to be confident and competent in their learning, and that learning is seen as being not just for students, it's part of the entire school culture."

That school culture is striking to a first time visitor. Instead of teachers delivering lessons, students seem to be on their own.

"When one comes in and they see kids in the halls and they see kids in small groups and in large groups, the next step is for somebody to really investigate what are those kids doing? You have children sitting in a small group, sitting by themselves, and they're actively engaged in learning. That to me means that they have an investment in their learning and the result is going to be a great deal more successful than if somebody is telling them that they have to sit there and have to do a certain event because it's time to do that and the curriculum calls for that."

One word heard often around New Suncook is choices, and nowhere are choices more apparent than in the multi-age classrooms. Multi-age classes are one feature of New Suncook. Kindergarten, first grade and second grade are all grouped together, and for one hour each day there's something called centering. The children are free to choose from a wide range of learning center activities - creative writing, art, perhaps just drawing a birthday card. And every activity, no matter how unimportant it looks, has a purpose.

"They get to see how rice will pour out easily out of some containers and not out of others, and they just discovering on their own what they can do with these [inaudible] seed. In years to come they'll be studying volume, and they'll be studying measuring, and they'll be able to see, you know, they'll have already played played with the containers and things."

Linda Deschenes has taught at New Suncook for 12 years. She and Lauren Potter were instrumental in starting the multi-age classes four years ago. Their primary consideration was children and how they learn.

"I think we really work hard at trying to think like a child, think about how children put the world together, and then we try to take the information that we're trying to give to them and bring it to them in a way that makes sense to them, and not isolated and separated in little pieces, and expect them to make the connections."

The various activities may look haphazard but everything is planned in intricate detail. For example, Linda Dushane and Karen Johnson meet every morning to develop a lesson plan for the day.

The plan is followed when their classes are combined for the one hour of centering time - two teachers and fifty young children who choose their own activities.

"The children are in control of their choices, but the teacher is responsible for finding out the needs of the kids and giving them choices within a range of activities. And when you have children long enough you can almost tell when a child is choosing something because it's easy, or because he'll get it done quickly, or if he's choosing it because it's a challenge."

There's no doubt the teachers are in control and they're expected to produce measurable results.

"There's a tremendous amount of assessment that goes on within a day and within the center time, a real assessment of where each of the children are at as they move through their center time and that's a part that we begin to talk about as staff as to what is the appropriateness of the activity, and it comes back to the question we ask often during the day is what are we doing and why are we doing it? We feel that the activities in here are beneficial to children, that there's learning taking place and that they fit in with the context of the entire day."

"Kids are involved in what they're learning and if you're going to be a lifelong learner, you have to want to learn and if we're excited about learning the kids are excited about learning."

Each spring Gary McDonald meets with parents whose children will be enrolling at New Suncook the following year. Many have reservations about the multi-age classrooms until they're reassured by other parents.

In fact, the building is filled with teachers who continually search for new ways to encourage learning. Linda Bradley's third graders are practicing with fractions but not from a book.

Meanwhile, Rhonda Poliquin's 4th grade class is studying meal worms. For the most part they work on their own in small groups. The class is an example of Gary McDonald's favorite saying - students shouldn't be consumers of knowledge, they should be producers of knowledge.

"What it means is that in lots of situations, students are sitting in their classroom being receivers of information that somebody is speaking to them. What we'd rather think that we're doing is putting kids into situations where they're actually taking that and applying it, and creating, again, situations for the kids where they need to problem solve and be critical thinkers, and that they're the ones that are producing those pieces of knowledge instead of simply sitting there and listening."

Even when a class is conducted in the traditional way, teacher up front, students in their seats, there's an attempt to make learning an active process.

New Suncook students are encouraged to be creative, so are the teachers. They're free to try anything to reach children as long as it works.

"I'm not comfortable with sitting in my office and saying that everyone needs to teach in a certain way. I trust the professional, I trust the professional in the classroom to make decisions based on the needs of the group and the individuals in that group to meet the outcomes. The accountability aspect comes in when we check the outcomes and to be sure that those outcomes are being met."

Those outcomes are judged in various ways. Every child at New Suncook is continually monitored to determine strengths and weaknesses, and there are standardized tests in reading, writing and math. New Suncook scores are in the top 20 percent of all Maine schools. This academic success is partly due to the atmosphere that's been created at New Suncook.

"Rosie White's class is reading a novel called 'Tuck Everlasting'. In the novel one character can drink from a stream and live forever."

These are fifth graders, but the atmosphere is more like that of a college classroom.

"I think the more you expect of children, the higher they reach. I mean you're not going to set goals that you know they can't attain but kids really like to talk, and if they really like to have the feeling that their view is really respected and I think a lot of people in this building besides myself really believe in doing that."

Respect for children is evident throughout the school. There's also a widespread respect for parents. They're invited to spend as much time as possible in the classroom. Valerie Wilfong has two children at New Suncook, she volunteers two mornings a week.

"Well it's great for parents, for one thing, to be able to be part of what's happening and see what's happening in the classroom and it's great where all the children are working at their own pace to be able to have an extra pair of hands to listen to one group, all the teachers working more intensively with someone else, so I think it is a great, great benefit for everybody."

"I have several parent volunteers and I really depend on them when they're here because it's like an extra hand and they really want to know what's going on in the school and it's great to have them here for their help but also for their enthusiasm, support they can lend to the school, so they're not in the way at all."

The school tries to expose students to the world outside of rural Maine. That's appreciated by parents like Brenda Thibodeaux. She owns a large farm that grows trees and potatoes. Three of the Thibodeau children have attended New Suncook. They've come away with more than just a sound education."

New Suncook provides the rural children an opportunity to cultural enrichment, to open their doors to see what's outside of this small area - what's in the cities, what's, you know, what it's like to be elsewhere, you know to bring that in, they've tried real hard to pull that into the school."

Part of that enrichment comes through field trips. On this day a few classes are off on a one-hour bus ride. Destination: the Maine Conservation School. Each year New Suncook has a theme, this year it's air and space. The students have been studying ornithology and entomology for eight months.

Now it's time to get some real life experience.

"We try to have kids for the most part have hands on, and a lot of choice, and so here what you're seeing is children moving around, experiencing what they're trying to learn as opposed to just sitting down and soaking it up with someone telling them about it, they're experiencing it firsthand."

The field trip is a time of intensive learning - learning that's integrated around the theme of air and space. For language arts there's a story about a caterpillar. Then there's art combined with entomology. The children are asked to make their own insect, one that can camouflage itself.

All of this is obviously a great deal of fun. But it's more than just fun, a fact that isn't lost on these children.

"By learning that there's different kinds of insects, and that people are like cutting down trees where butterflies migrate, so the butterflies die 'cus they don't have no place to go in the winter."

Back at school the integrated learning continues. Students are encouraged to write about the insects they saw. Once again the knowledge comes from the children.

Even in music class where students are getting ready for a school performance, the songs are about flowers and insects. This week includes another field trip, this one for teachers. New Suncook's theme next year will be water and the teachers have to learn before they can teach. They'll be bringing their classes to this bog and the children will be asked to search for bugs and worms. This field trip is voluntary but it's attended by nearly every teacher in the school.

"Certainly in a school like New Suncook and I think a lot of schools where it's really committed people, these people are out on their own, they're involved in their staff development activity, really they can relate back to the classroom and can relate to trying to make better things for kids in classrooms."

"Part of the framework of what your job involves when you're actively involving kids is you have to become involved yourself, you can't just learn it from a book."

"And I was just saying on the way up here just, you know, walking, I've learned so much just the stuff that we've tried to present to kids and have them experience, you know, stuff that I certainly didn't know."

The teachers at New Suncook seemed to work as a genuine team. In many ways, so do the students. Their learning is usually done in groups, meaning they have to get along. That idea of cooperative education is stressed throughout the school, even in physical education.

Phys ed specialist Robert Davis emphasizes cooperation not competition. In this game, the object is to knock a tennis ball off a pylon. Students quickly learn that the best way to do it is by passing to their teammates.

"If they can't get along with each other they're not going to be able to focus on psychomotor skills, and that's my philosophy of physical education, that's probably the most important part, is that they learn how to get along with each other. [inaudible] This effective development in children I think is very important and this is a great medium to focus on that."

Meanwhile this cooperative exercise is meant to focus on creative writing. Once a week students of all ages gather in small groups. They take turns reading stories they've written and asking for help.

The effort to encourage cooperation extends throughout New Suncook. Each day older students volunteer to pass some of their knowledge on to younger. The purpose isn't just to have kids teaching one another facts, it's also a perfect example of modeling.

"Modeling is children looking up to other children and doing what they're doing. It's not just the teacher who's going to model something for another child, it's kids can model for other children, teaching them how to write. A young child's like, 'oh you're writing', and those kids will just start copying them."

In order to foster even greater cooperation, Rosie White uses something called the magic circle. For one hour each week her students are free to talk or complain about anything they want. A student is in charge and Rosie White is just another participant.

"It's not really group therapy, it's part of my philosophy of making kids feel good about themselves, and learning that they can say what they want and their opinion counts, and they can solve problems or injustices, maybe is a better word, vocally rather than keeping them inside themselves or physically dealing with them.

The magic circle might seem like pop psychology, but again it's intended to encourage learning.

"For students to be able to learn, they need to feel [comfortable] about themselves, and to get the most out of what you can teach them, to feel they're respected, their, you know, their opinion is valued and they're worth something, and so I think the mood in a room is so important."

One thing is very noticeable around New Suncook - there's very little emphasis on discipline. According to Gary McDonald, that's a side effect of the school environment.

"Someone might see this as being a laid-back situation. On the other hand, it's structured to such an extent in many ways that discipline does not become a problem because the kids are always engaged, and they're always involved, and they're invested in their learning and they're raising questions and when children are in those situations, I think the discipline becomes less of a factor."

"If children talking to children is inappropriate behavior, then I guess we have no discipline. But we feel that children learning to talk to each other, cooperate, work together is preparing them for their life way down the line and working in team projects, so that isn't an issue."

New Suncook has the most extensive special education program in the district. It's the receiving school for children with severe emotional and learning disabilities. And those children are taught much like all the other children. Alice Holt is one of three full-time special ed teachers. She spends part of her day on intensive individual instruction. But most of her time is spent in regular classrooms. She's there to help all the students, not just those who are labeled special ed.

"What we're trying to do is have the kids become really members of the classroom instead of separate. There's so many times where I'd come to a classroom and it might be, you know 'Bobby, Mrs. Holt's here to get you', you know and now there's none of that, it's really you come in the classroom and you're sort of freebie for anybody that might want you, and there's lots of kids in the middle who just want a question answered or, you know, a little bit of help or some attention for something good that they did and they love it to have somebody else that they can talk to."

Cooperative learning, multi-age grouping, integrated curriculum, a positive environment.

These elements and many others have played a role in making New Suncook an exemplary school. It's the culmination of a process that began nine years ago, when Gary McDonald and his staff got together to write down their vision, then turned it into a reality.

"My vision, I think the vision of the school and a lot of the activities we do is that kids are lifelong learners and that learning doesn't stop. Learning is a lifelong activity, it isn't only in school, it's not only in class, it happens in many, many different activities, many different avenues and that hopefully people recognize that and that schooling is not the only place that it can happen, but all those put together create a context which hopefully maximize an individual's capacity to do what they would like to do with their life and to be a contributing member of society."