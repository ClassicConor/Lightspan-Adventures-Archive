\chapter{Every Child Can Succeed 3}

\begin{figure}[H]
    \centering
    \includegraphics[width=\textwidth/2]{./Games/EveryChildCanSucceed/Images/EveryChildCanSucceed3CD.png}
    \caption{Every Child Can Succeed 3 CD}
\end{figure}

The third of the Every Child Can Succeed games published and released by The Lightspan Partnership for the PlayStation 1.

Every Child Can Succeed 3 features three video programs:

\begin{itemize}
    \item High Expectations
    \item Parent Involvement
    \item Productive Climate and Culture
\end{itemize}

\clearpage
\newpage

\section{High Expectations}

\subsection{Audio Summary}

High Expectations demostrates how to encourage high achievement by all students, designing a sequential curriculum that ensures every child will master each step before moving on.

\subsection{Transcription}

Expect excellence.

"One of the things that I know as a parent and as a teacher is that no one rises to low expectations."

Expect excellence every day. Expect excellence from every student.

"We demand the very best, we demand that they achieve, and they do."

Comments overheard in the teacher's lounge.

"You're having trouble with Alan? Well, I'm not surprised, I mean he is Mike's brother."

"Higher order thinking skills? Forget it, these kids need to learn the basics."

Look around. Most people don't expect much from the kids who live here. Fortunately, educators at highly successful schools see things differently.

"The perception of a school in South Dallas, or in any inner city area, in an urban school district, is that inner city kids cannot learn, that they are deficient in some way, and I'm here to make sure that that is a myth. That it is not true, absolutely not true."

"My kids do as well or better than kids just about anywhere in this country. I can put up my top kids against the top kids just about anywhere and I know that they'll do well."

"Some teachers will say, you know, where someone's always doing the best they can, and when a student realizes that you don't expect much out of them, they won't give you anything. But if you expect a lot of out of them and help them and guide them, then they'll give you exactly what you want."

"I am not an underachiever, you're an over expector."

At the schools we visited, high expectations were an essential part of the school's culture. Everything the educators do is infused with their belief that all students will succeed. But how do they communicate these high expectations, and how do they help children meet them?

One way to convey high expectations is by monitoring student progress. Monitoring helps communicate the school's high expectations and insistence on academic excellence. At the Vann School in Pittsburgh, principal Doris Brevard is a constant presence in the classroom.

"I expect the children to do their very best in every written assignment, in every test and every recitation, I expect the very best."

Another way to communicate and achieve high expectations is to create a supportive learning environment.
At Euclid Avenue Elementary in Los Angeles, principal Esther McShane inherited a school that was unruly, unsafe, and covered with graffiti.

"You need to model these things, you need to model the expectations. I needed to change some of the staff members and the custodial staff to bring change, and my model to them, as it is for the secretarial staff or any support staff, is that we must provide an environment where teachers will want to come, where kids want to come, where parents want to come, because it's clean, it's neat, it's orderly, somebody truly cares, and if somebody shows you that they truly care, then people begin to feel they're important, I'm important, I can live up to whatever she thinks I'm capable of doing."

A positive learning environment is also a priority at Bennett Q School in Inglewood, California. Principal Nancy Ishinaga makes this a safe haven for students.

"First of all, we believe that the school must be a safe place for all kids. In other words, the streets may be dangerous, but the school must be safe, and so we have some strict rules, for example, there is to be no fighting. If a kid gets involved in a fight, there are serious consequences, so we really don't have very many fights."

Firm discipline is also a given at Rice School in South Dallas. The school's motto says it all: excellent academics, excellent behavior.

"I firmly believe that behavior comes first. In order for kids to learn, they must know how to behave in school, they must know how to come in school, sit, listen to the teacher in order to learn. No child can learn if they're talking when the teacher is talking, they miss out on too much. So the first thing we that we work on in these schools is behavior, and we work on that right along with academics."

Effective school leaders create school policies and practices that help them accomplish their high expectations. When Louise Smith became principal at Rice, one of the first things she did was demand more from the school's kindergarten.

"Usually in schools you find kindergartens, they play and they sleep all day. When I got here, kindergartens were taking naps for maybe two or three hours a day, and the first thing that I said, 'Get them up, please, and have them learning, and you teach them'."

At highly successful schools, the leaders demand that students achieve. Nancy Ishinaga of Bennett Q makes her expectations clear to both students and parents.

"In other words, we have expectations by grade level for every kid, and the kids know from the very very beginning that if they are in kindergarten, these are the things they have to master before the year is over, and if they don't, they probably will not go on to the next grade, and our kids understand this, and parents understand this very well."

Pat Sander is another leader who creates school policies that help realize her high expectations. At Alki Elementary, her approach is to teach as though every child were gifted.

"At Alki, what we've tried to do is no longer remediate children but accelerate them. We believe that the kinds of education that people have supported for gifted children over the last five or ten years are the kinds of instructional approaches and the kinds of education that all children should have."

Like Pat Sander, principal Gary Tubbs at Beacon Hill expects excellence from his students, and Tubbs applies the same standard to his teachers too.

"I have no tolerance for a teacher that tends to give up on a child, because I don't believe it. I've never seen a child, this is my 17th year, I've never seen a child who can't learn, and I've worked with special education children, I've worked with children of all poverty and affluent levels, children from the worst, just incredibly difficult histories in their family and in their personal lives. Those children can learn too, we just haven't, in some cases, found the right combination to enable them to learn."

"Yes, some adults do paint like that, but we expect more from children."

In this classroom at Rice, math teacher Victor Washington works with students to develop their thinking and problem-solving skills. Rice students are known throughout the district for their excellence in math. They are a major force at Math Olympiad competitions.

"I expect them to learn math, they don't have any other choice, it's mandatory, they're there and they can do it. I know that a lot of people are afraid, minorities as well as other students are afraid of math, they run away from that, but here, if you're in my room, I expect you to learn. I'll help you any way possible if you want my help, but I expect you to do it, and you have no other choice."

Rice's dedication to excellence in math is enhanced by its involvement in Project SEED. SEED is an intensive class for all Rice math students in grades four through six. But this isn't just about math, it's about building a positive self-concept.

"We use mathematics as a tool, and the tool is to raise the self-confidence and self-esteem. You can raise their self-esteem by teaching them how to play chess, but when you teach them the mathematics and get them excited about their learning, education, because education is the key for a lot of these kids to be successful. If we get them excited about learning, and believing in themselves that they can do anything that they want to do, as long as they try and give it their all, then that's what we are for."

Glee's high expectations for his math students are being realized. These students are doing work many grade levels above their actual standing.

"In many occasions, I've actually come into the classroom and brought my calculus text and said 'pick one of those problems out there' and they would pick it, and then when they solved they said 'Man, we solved a problem from a college algebra book' and they get so excited, and so that right there just boosts their their self-image and their self-confidence that 'Hey, if I can do this, I can do anything on sixth grade level or seventh or eighth grade level'."

William Glee has felt the sting of low expectations himself. Despite good grades, his high school counselor suggested that he wasn't college material.

"So I was told that, and I believed in myself and said 'Well, I'm not going that route, I'm going to college', so I said, if I was told that then, some of these kids are being told that, not only by time to get to high school, they're being told that third grade, fourth grade, fifth grade - by the time they get to the seventh eighth grade, they're starting to believe what they hear, they don't believe in themselves, so I believe in and I have confidence that every single kid can learn if they're motivated."

"Name the elements."

"I think I can, I think I can"

On a Seattle beach, these first graders from Alki School are getting some hands-on learning today. The teachers hope the experience will motivate kids to learn more about sea life.

"It's a way to bring real life experience so that the child can then take that information back and write about things, they become totally engrossed in what it is that they're doing, and the motivation is there, that the feeling that 'God, we want to learn all we can about it', and it just seems to create a feeling in them that they can do it and they want to do it and it has meaning to them, and I think that's the whole point of what we're doing for kids."

Later in the week, the same students will put together their own books about the field trip. They'll write poetry, create math story problems around sea life, and do art projects about the creatures they saw.

In another Alki classroom, Colleen Dumas works to integrate the curriculum and improve her students' critical thinking skills. In an activity called 'Private Eye', the students look at common objects through a jeweler's loop and draw what they see. Then they make analogies based on their drawings. Later the students will explore the analogies even further in their writing.

"Analogies showing relationship is an important skill. I'm finding that the analogies, and their ability to see relationship transfers. It transfers into their science, into their social studies, it integrates the curriculum more. I think it's a great program."

High expectations - you've seen how students can rise to them. The schools that we visited created supportive learning environments, set rigorous academic standards, and developed instruction that promotes high order thinking skills. These schools set high expectations, and found ways to help students meet them.

"There's a tendency for educators to say 'I can't teach this child when they come from this kind of home', 'I can't teach this child when they don't come with any breakfast', 'we can't teach this child, they're just too disruptive', and so the finger pointing is that the family, or the finger pointing's at society, or the finger finger-pointing's at the teacher next door, or at the principal, and the the concept or the philosophy in this school was 'let's get past that, that that doesn't no good, that we need to maybe bend our finger around to ourselves, each of us to ourselves, and say how are we accountable, and do we believe that we as a team can truly make a difference in the lives of all the children?'"

\section{Parent Involvement}

\subsection{Audio Summary}

Parent Involvement examines why all parents should communicate with the school about their own children's progress, discipline and achievement standards.

\subsection{Transcription}

"What we're really used to looking at is that traditional PTA where everybody comes once a month and you kind of have tea and cookies and you discuss the PTA budget and that may not be what parents are interested in in this particular community. I mean parents may be interested in just those things that affect their child, so you almost have to begin to kind of rethink PTAs and those kind of organizations. I mean what are they here for?"

Parental involvement can be a potent force in helping to improve schools. In the highly effective schools we visited, parents do play an active role in their children's educations. We found that relevant parental involvement, the kind that has a direct positive impact on learning, can take many different forms. We'll look at four of them.

"Hey it's just a report on my progress, not an indictment of my upbringing."

At the Vann School in Pittsburgh, most of the students come from single parent families. Principal Doris Brevard knows the pressures on the parents in her community.

"And that's not true, especially in this area where you have parents that are just trying to survive and that takes all of their energy and their time. But all I ask is the parent have a positive approach to the school, that the parent sends their child to school with a positive attitude, that the parent feels a part of the school. They know they can come to us if they have a problem, they can discuss it with us. If they need help in rearing their child, helping their child with their homework etc, that they can come here for assistance. If we need them and call them, we want them to come, but we don't expect to see them every day volunteering their time, that's not necessary."

By getting kids to school on time with a positive attitude toward learning, parents support the day-to-day activities of the school. Another way they can help is to keep track of their child's progress. At Seattle's Beacon Hill School, Principal Gary Tubbs also faces a parent community with precious little time for traditional parent involvement. He instituted a folder system to help parents monitor their children's progress.

"We give homework and we believe that everybody should be held accountable for completing that and returning it to school. Then once a week, flyers, newsletters, notes, all sorts of things will go in this folder and then these will go home just once a week. And again the parents and the students are responsible for making sure that the parent signs here and then that's returned to the teacher and the teacher knows that the materials got home to the parent. That keeps our communication lines open, it also involves the parents in a meaningful way in the education of the children."

"He doesn't listen to a thing we say, he's very noisy and he's always getting into trouble. I think he's ready to start school."

Educators agree learning can only take place in a safe, secure environment and in order to have one, discipline is essential. Parents can play a major role in supporting school discipline. At the Euclid School in Los Angeles, Principal Esther McShane first works on discipline with the individual child.

"I need to help the child understand their particular emotion at that time where there was a violation of another human being. My philosophy and the discipline of the child, if the child comes to my office on the third time, it's mandatory parent involvement. We make we make a session, we meet together. I don't talk about the problem, the child talks about the problem. Parents need to know that we're here for them, we're here to support the child and to support their role."

"Dad, parents day was yesterday, it's my turn now."

When parents do have time, they can contribute to instruction in the school. At the New Suncook School in rural Maine, Principal Gary McDonald knows that parents can make a difference.

"Parents to me and to the entire staff here are seen as being tremendous value. They're welcome at all times and we try to provide them with the knowledge and the opportunities that they can really feel an integral part of the building."

"I mean right from the very beginning we were welcomed in before Christian started kindergarten and very open policy - we want you to be a part of what's happening, really a partnership happening here and that was wonderful for us."

Valerie Wilfong is one of the many parents who help at New Suncook.

"I've worked in schools before I had children, I volunteered in classrooms where teachers were not used to having volunteers or other people and they kind of felt kind of nervous but all of these teachers are set up to feel real comfortable about other people in their classroom, and they're not threatened by them. It's great for parents, for one thing, to be able to be part of what's happening and see what's happening in the classroom and it's great where all the children are working at their own pace to be able to have an extra pair of hands to listen to one group all the teachers working more intensively with someone else, so I think it is a great, great benefit for everybody."

"I have several parent volunteers and I really depend on them when they're here because it's like an extra hand and they really want to know what's going on in the school and it's great to have them here for their help but also for their enthusiasm, support they can lend to the school, so they're not in the way at all."

"Everyone knows me, you come down the hall at the end of the day and 'hi Mrs Wilfong, look what I did in school today' and it's a great kind of feeling."

At the Bennett Q School just outside Los Angeles, Principal Nancy Ishinaga expects parents to help their children master certain skills at specific grade levels. Here she talks with the parent of two new students.

"I had a fourth and fifth grader who came in from a school in Los Angeles. I checked them on their writing, on the reading, and their math and I found out that both of them were a little bit below grade level. Now the parent said 'well you, know at the other school, my daughter was an A student, was on honor roll, and was doing very well' and I said yeah I think she's very bright but there's some skills that she hasn't been taught. Generally they understand. Parents understand kids have to attain certain skill levels, that's not new. In fact, many of them say, 'phew, it's about time there's a school that requires certain kinds of learning before the kids are promoted. In my other school they just passed them on. So we don't do that."

Asking, and expecting a parent to give a child some help in learning is one way to increase parental involvement. Of course the traditional fundraising functions of parent groups have long served to enrich the academic experience for students. These bingo players are helping to raise money for the students of the Thorncliffe School in Edmonton, Canada. These students from Euclid School in Los Angeles are on a field trip in Valley Forge, Pennsylvania with money raised partly by parent groups. And at the Charles Rice School in Dallas, the choir prepares to perform on a road trip funded by parents.

"We've been active in a lot of trips out of town, we've got parents booster clubs who support the choir, or the choir's doing an east coast tour this spring, you know and people outside this community would say, 'well how are you going to do that, how the world are those parents going to afford to send their kids on an east coast tour?' Parents do it because they know that it's important for the kids, and I think we can never underestimate what our parents will do when their kids' lives are at stake."

"So, uh, tell me: how do you like the parent-council's new idea for checking attendance?"

At the Alki School in Seattle, parents are involved in all the ways we've talked about and in one other - school management.

At Alki, a site management council, similar to a board of directors, makes important policy decisions concerning the school. The council includes the principal, staff members, and four parents. One of those parents, Bill Stahlser, is the chairman.

"Why I do it is there is more than one perspective when you're talking about the education of kids, and the one perspective, I think even of parents, we've abdicated, is to say 'okay well teachers you do it' okay, 'the principle and the teachers, here, here's my kid, you know you teach them, and I'll go away', or except to complain maybe about, you know, you're not doing enough. And so this says, now wait a second, that the parents perspective, and the perspective of somebody in a business in this community, in this neighborhood, they all have a perspective that affects education of kids, and should be taken into account."

Louise McKinney is also interested in the site council and its efforts to improve student achievement.

"All of it is focused, to the end that children will learn, and anytime you're doing something, and you can't see the connection between it, and achievement for youngsters that's something that you discard. Different faculty groups, different parent groups have different reasons for starting site councils. Some schools look at it as that's the way to get rid of teachers, or that's the way to get rid of the principal, and if that's the reason why a site council is established, then that's the wrong reason."

"It's much more, I guess in a supportive role saying, okay, 'can I help marshal the resources', and second to evaluate from an outsider's point of view. How's the school doing, or how are kids doing academically, how's the principal doing, okay, and not sort of any hostile way but say, you know we need to ask ourselves how are we doing?"

Another site council member is Ngy Hul, a native of Cambodia who came to America in 1975.

"We are raised in a very poor country, we didn't have that good opportunity like this country, so when we are here in this country, have plenty of opportunity, why not take it? So because of that commitment, I would like to see our refugees kid grow the way it is supposed to be. That's what my commitment to be part of this site council. I think this school here look like a parent and teacher are partner and work together. It's very important that all of us must work together, be open minded and try to understand each other. So when I'm involve in this one, not just only site counsel, I try, you know, tell them about my culture and all the people culture so then we can try to understand each other and can work well together."

"Not one of us can do it alone. The parents know certainly the most about their children, they know more than we do and they have a different perspective on what's beneficial for their kids. The staff brings that educational background and professional experience and hopefully I bring the administrative background that that can maybe pull it all together."

Understanding each other and working well together. When teachers, principals, and parents can do these things, schools prosper. Parental involvement helps. The four kinds of involvement we've looked at can make a major difference in student learning and success. Whether it's supporting day-to-day activities by getting kids to school on time ready to learn, or keeping track of student progress and homework completion, supporting the teacher when it comes to discipline, volunteering time to help in the instructional program itself, or even to help manage the school, it all helps.

"No one person or group of people is responsible for the education of our children. We believe in parents specifically being involved. We believe in teachers coming together with that community, and those parents, to make sure that everybody understands what the process is, everybody understands what the vision is, and everybody is able to find his or her place in that process toward the implementation of that vision."

\section{Productive Climate and Culture}

\subsection{Audio Summary}

Productive Climate and Culture discusses how a clear mission statement emphasizes academic achievement for all children.

\subsection{Transcription}

What is a successful school like? Is it noisy? Is it quiet? Is it traditionally organized, or is it more experimental? You can't always tell by looking. Each successful school is unique. It is shaped by its mission, a set of beliefs shared by everyone at the school. Beliefs that guide the way to success.

Each successful school we visited was once a troubled school with poor test scores and low staff morale. Each school turned things around. How? They created and then implementing a mission focused on improving achievement for every child.

"A school should be a place where kids come to learn. We believe that it is our job to educate the kids, to make them literate. It's not the home's job, it's our job."

"We feel that here at New Suncook that all kids can be successful. We feel that it's our responsibility to create environments so that we try to maximize, for every child, their potential."

"I firmly believe that all children can learn. I think that Hank Levin says that parents send us the best children they've got, and they, in fact, do. And I think that every child has a strength, and what we have to do is find that strength."

"All of the folks here share a common vision, they have a very similar philosophy, and that's every one of them will teach and every child in this building is going to learn."

"Hello, I'm Billy Bernard, part of the mess you inherited from the previous administration."

When Pat Sander became principal of Alki School in Seattle, she was confronted by a staff and parents who lacked a common vision of what the school should be.

"Basically, it was a school that staff didn't talk to one another; they were isolates within their own classrooms, and the staff thought that the parents ran the building. The parents thought that the staff ran the building, so there was no communication between parents and staff. And there was just a real power struggle going on, even amongst staff, as to who was going to be in charge.

At the end of Pat Sanders' first month on the job, a newspaper reported that Alki was one of the 20 worst schools in the city.

"Well, probably the best thing that happened for me was that the scores had been falling, and the newspaper article came out because it gave the parents, the staff, and myself a place to rally around that was an objective place, that we needed to make a difference and do something different for children."

Out of this experience came the beginnings of a shared mission. Parents and the Alki staff began working together to improve student achievement. Across the bay from Alki is Seattle's Beacon Hill School. Here, the school's mission has become a living document; it's as vital to Beacon Hill as a constitution is to a country.
Principal Gary Tubbs and his staff wrote the school's mission together.

"So this is a way of clarifying and putting down, making a snapshot picture, of where we see ourselves, where we'd like us to go, and trying to celebrate the success."

Beacon Hill's mission is to develop their students into responsible, literate adults. That's easy to say, but Gary Tubbs knows the hard part is making his school's vision a practical reality.

"That, of course, is a vision that needs to be constantly addressed and revisited, and be kept alive in our heads but also brought down to practical, down-to-earth terms on how does that look in the classroom? How does that turn into instruction? How does that turn into curriculum? How do we make learning meaningful so that it actually gets children a step closer to that vision?"

"I'm your teacher, Mrs. Gridley. Learn to read, write, and do arithmetic, and nobody will get hurt."

Once a school has created a mission, next comes the day-by-day efforts to implement that mission. When Esther McShane became principal at Euclid Avenue Elementary in Los Angeles, it was the challenge of her life. The school's environment was the worst she had seen.

"Professionally or academically, it was a school that was almost a joke. It was also a very filthy school. There was no pride. You know when you look at a situation where there's no pride, you need to start there."

McShane began to implement the school's mission for success by creating a more supportive environment for teaching and learning. Her actions showed the staff and students that their new principal cared about them.

"And if somebody shows you that they truly care, then people begin to feel they're important. I'm important. I can live up to whatever she thinks I'm capable of doing."

Of course, a clean environment is only the beginning. At many successful schools we visited, the discipline policy is also tied to the school's mission of academic success.

"Behavior comes first. In order for kids to learn, they must know how to behave in school. They must know how to come in school, sit, listen to the teacher in order to learn. No child can learn if they're talking when the teacher is talking; they miss out on too much."

At the Vann School in Pittsburgh, Principal Doris Brevard has implemented a strict discipline policy for more than 20 years. She uses discipline to maximize the amount of time her teachers have to teach.

"The principal must provide the atmosphere, and then it's the teacher's job to teach. But if that teacher can go into his or her classroom and teach 35 minutes of the 40-minute period, then something will be accomplished.
But the principal must see to it that the discipline throughout the building is such that the teacher can teach the 35 minutes."

"She told me that she expects discipline, and that I'm here to teach, and that if there are any problems, I just let her know about them, and she will handle them, and she's done that. She's very supportive."

At Vann, it's every teacher's responsibility to help implement the school's mission. Doris Brevard offers encouragement and advice, but the teachers are on the front line. The same thing is true at New Suncook School in Level, Maine, where teachers work to turn the school's mission into a practical reality. Rhonda Poliquin teaches fourth grade at New Suncook. She takes the school's mission to heart when she plans class activities.

"We've worked towards kids becoming independent learners so that you won't see as much a teacher standing in front of the class and lecturing to kids because we want kids to find out how to get that information themselves. and if there's something they want to find out about, giving them the tools and the skills that they need to be able to find out that information."

At the successful schools we visited, the school's mission is implemented through a carefully designed instructional program. Nancy Ishinaga sees it as a three-step process.

"Number one is that we do believe that every child can learn to read, to write, and to be a successful student. And this belief is throughout the school. Everybody believes that, we just have to figure out how to do it. And the how to do it is the next thing."

To implement her mission. Nancy Ishinaga personally tests the academic skills of each new student, and she meets with the child's parents. Then the students enter a sequenced and rigorous academic program. Nancy Ishinaga visits each classroom periodically and meets with teachers to assess how the students are doing. This is part two of her process.

"And maybe the third thing is that we try to build an environment in which everybody feels that they are important and the fact that what everybody does contributes to the whole school, so that any school which wants to succeed can do this. It's not a matter of personalities, it's a matter of having a system by which we educate our kids."

How do schools know if their mission of academic success is being achieved? One technique we saw in use was constant monitoring. Doris Brevard visits every class in her school at least once a week to find out what children are learning. Principal Esther McShane at Euclid spends every possible moment in the classrooms, on the playground, or just roving through the corridors of her school.

Ruth Johnson of the California Achievement Council believes that constant monitoring and improvement are two keys to success.

"Successful schools, I also find they self-assess. They don't wait for an outsider to come in and tell them how well they're doing, they're constantly in a mode of assessing 'how are we doing', 'what do we need to do better? There's constant staff development, there's peer coaching going on where teachers are looking at each other and constantly trying to enhance what they're doing in the classroom, so it's a constant mode of self improvement going on all the time."

At Beacon Hill, principal Gary Tubbs is also a strong believer in monitoring to assess how well his school is achieving its mission. And when Tubbs sees that a student is having a problem, often it's time for a problem-solving session.

"It's Freddy's parents, they want you to tell them again how gifted he is."

Another way for schools to reinforce their mission and create a positive culture and climate is to publicly recognize those students and teachers who succeed. Whether it's for hard work, improved grades or behavior, good citizenship or good attendance, Spirit Day at Vann School is a celebration of those students and teachers who live the school's mission.

At Thorncliffe Community School in Edmonton, Alberta, Canada, principal Lou Yaniw also believes in recognizing student achievement. Every student feels they're a part of the school's mission. The super student award is given weekly to students nominated by their teachers. At Thorncliffe, every student feels they're a part of the school's mission.

"The vision that we came to Thorncliff School with was to make this the best school that it can be. It is part of the heart and soul of the school that in a student-centered school, the belief, the generic belief, is that every child can succeed."

In the schools we visited, the turnaround from failure to success began with a clear mission, a vision shared by everyone at the school. They implemented their vision with discipline, teacher involvement in decision making, monitoring, problem solving, constant improvement, and by recognizing both student and teacher achievement.

"The principal by him or herself cannot orchestrate all that unless they have the support of their teachers behind that whole vision and philosophy, and it's not just putting the vision down on a piece of paper and saying 'now we have a vision', it's truly believing in it and truly having it in your heart as well as in your head that you live it every day and it becomes where all of your energy goes, you continually work to channel a greater percentage of your energy into that vision."