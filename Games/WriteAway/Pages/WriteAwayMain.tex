\chapter{Write Away - Overview}

Write Away is a series of educaction video games developed by The Lightspan Partnership starting in 1996.
Each of the 10 CD's consists of 10 separate videos, with the video length combined being less than 30 minutes into total.

The volumes, and the cultures which are represented with the 16 Tales games are as follows:

Similar to the 16 Tales video games, the entirety of the Write Away series is a series of videos, with the user having no control over the video content.
For each disk, the content always includes an introduction video, seven to eight story videos, and a conclusion video.

The introduction video always contains the same content: the 'Write Away' theme song, and a brief introduction to the characters.
The same series of actors and actresses are used in each of the videos, but they switch between one another in each of the videos.
That is to say, a single actor may appear in multiple of the games, but not all of them.
The seven to eight story videos are always different, and there is no pattern to the content of these videos.
The conclusion video always contains the same content: the 'Write Away' theme song, and a brief conclusion to the characters.

In the conclusion videos of each of the games, the actors encourage the student watching to write their own stories and send them to:

\begin{center}
    Write Away Information\\
    The Imagination Machine\\
    (714) 771-2499
\end{center}

Lightspan claims that it actually used the stories that they were sent to them in order to create new videos.
This is indicated when, viewing all of the video programs, the title of the video program is accompanied by the name of the author of the story.
In addition, the image which is used to accompany each of the eight story videos usually either a picture of the author of the story, that is to say, an image of the child who wrote the story, or a picture of a scene in the short film (presumably when they couldn't get a picture of the child).

The documentation for this game was slightly more difficult that the documentation for the 16 Tales series.
The main reason for this was that, in comparison to the 16 Tales series, where the content for all these videos was already online and available to view on YouTube to grab the transcripts, this was not the case for the Write Away series.
The YouTube Channel \href{https://www.youtube.com/@longplayarchive}{LongplayArchive} had already uploaded the first four games of the series onto YouTube in high quality, but not the final six.

To find the remainder of the six video on YouTube for easy transcription, I had to find the much smaller YouTube channel, \href{https://www.youtube.com/@WizardmanTonight}{WizardMan} who had the remaining videos uploaded onto his YouTube channel.
All of the Write Away videos on his YouTube channel have an extremely low number of views, with, Write Away 5 having only 8 views at the time of writing this overview (31st March 2024).

Similar as well to the 16 Tales series, the degree to which the content of this series could be considered educational is questionable.
The content of the videos is not particularly educational, and the only educational aspect of the videos is the encouragement to write your own stories at the conclusion of each of the videos.
The main focus of these 'games' was to encourage children to write their own stories, and to be creative in their writing.

With respect to being able to play these games online, of course \href{https://vimm.net}{vimm.net} remains the best place to emulate all of the games.
However, \href{https://archive.org}{archive.org} is also a good place to have a look at playing the games, hosting all of the games except Write Away 8.

These games, like 16 Tales, can be more accurately described as interactive video programs, as the user has no control over the video content.
The user can only pause, rewind, and fast forward the video content, and select which video programs to watch.

I was not able to find any information regarding merchandise for the Write Away series.
I suspect this is the case as a result of the Write Away series not being directly developed by The Lightspan Partnership, but rather by the children who sent in their stories.

Unlike the 16 Tales series, which I personally felt to be quite dry, I found all of the Write Away video programs to have a certain charm to them, even if I wasn't the target audience, and even if I personally wouldn't watch them were it not for documentation purposes.
The fact that the actors and actress are made to perform scripts written by children, and wth the limited resources available to them, I found to be quite endearing.
A nice texture to some of the Write Away videos is the 'Tales from the Script' episodes, which appear sporadically throughout the series.
The 'Tales from the Script' videos, in production and performance, essentialy no different to the other videos.
However, they are all horror-themed, and the videos are all introduced by Paul, one of the recurring actors in the series.

The development of this series, like the rest of the Lightspan games, remains, to me, one of the strangest stories of the PlayStation 1 era, and perhaps gaming in general.

\section{Lightspan Credits}
President: Carl Zeiger;
Vice President, Product Design: Marguerite Hillman;
Vice President, Affiliate Programming: Inabeth Miller;
Director of Interactive Develmpment: William Volk;
Program Manager: Jim Marshall;
Curriculum Program Manager: Kevin Clark;
Interactive Development Program Manger: Tom Rahts;
Lead Programmer: Rob Trickey;
Lead Designer/Artist: Charlene Alexander;
Quality Assurance Analysts: Quynh Nguyen, Jim Wadsworth;
Assistant Producer: Jenny Merrill;
Information Systems: Darren Craun;
Associate Producer: Christina Bavetta;
Director of Sony Programming: Robert Warren;
Artist/Producer: Ken Anderson;
Artists: Lisa Limber, Sean Jackson;
Media Conversion: William O'Bannon, Janene Wittmayer