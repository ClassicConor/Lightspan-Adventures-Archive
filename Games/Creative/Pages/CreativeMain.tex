\chapter{Creative - Overview}

Lightspan released a number of series of video games over the course of their production run.
One of these series was the Creative series, which was designed to help children develop their creativity, albeit in an limited, rudimentary format.

A total of four games were released for the Creative series:

\begin{itemize}
    \item Creative Camp
    \item Creative Isle
    \item Creative Journey
    \item Creative Voyage
\end{itemize}

Each of these games, although played in a different setting, had the same basic gameplay mechanics.
The player, upon loading the game, would immediately be placed in a particular environment within the game, and, using the blue toolbox provided from above, would be able to place people and objects within the environment, allowing them to control the movement.
The player is be able to place their creations in a variety of environments within each of the game locations, which would be accessed by clicking on the various locations on the screen with the in-game cursor, allowing the player to move to the next location, resetting the screen in the process.

Given the incredibly limited nature of the gameplay, in combination with the limited number of locations for each of the games, it is possible to explore the entirety of each of the games in a relatively short amount of time.
Each of the games in the Creative provides the player with a series of characters to play with.
As well as this, the games provide the player with a series of powers that can affect the characters and objects in the game, such as increased speed, or walls that can be placed to block the characters from moving in a particular direction.
Many of these abilities that the player have are able to be enhanced with a number between 1 and 10, 1 being the weakest and 10 being the strongest.
The player is also able to write messages into the game itself.

A final note on the Creative series of games is that, unlike many of the other games released by Lightspan, there is no educational content within these games.
That is to say, although many of the other games released by company have the potential for integration into the curriculum, these games have no such potential.

In spite of Lightspan releasing four entirely separate PlayStation 1 games, there is remarkably little to say for all of these combined.