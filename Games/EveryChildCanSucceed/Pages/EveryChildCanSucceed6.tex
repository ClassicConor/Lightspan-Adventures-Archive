\chapter{Every Child Can Succeed 6}

\begin{figure}[H]
    \centering
    \includegraphics[width=\textwidth/2]{./Games/EveryChildCanSucceed/Images/EveryChildCanSucceed6CD.jpg}
    \caption{Every Child Can Succeed 6 CD}
\end{figure}

The sixth of the Every Child Can Succeed games published and released by The Lightspan Partnership for the PlayStation 1.

Every Child Can Succeed 6 features three video programs:

\begin{itemize}
    \item Northwest Passages: Alk and Beacon Hill in Seattle
    \item Charles Rice Learning Center: an Oasis in Dallas
\end{itemize}

\clearpage
\newpage

\section{Northwest Passages: Alki and Beacon Hill in Seattle}

\subsection{Audio Summary}

Northwest Passages: Alki and Beacon Hill in Seattle looks at how two schools serving a low-income, immigrant population district with over 20 different spoken langauges have raised achievement well above the district average.

\subsection{Transcription}

Take one underperforming school.

"We were on a downhill shoot."

Mix in a committed administration, a management council with broad powers, and a new principal with a vision.

"Pat came here with a commitment to teaching all children well."

Blend in some innovative teaching techniques and a reinvigorated staff. The result is one of the highest achieving schools in the city of Seattle, the Alki Elementary School.

It's 60 miles to the south, yet Mount Rainier somehow dominates the Seattle skyline. The man-made office tower seems small by comparison. In this city on the Puget Sound, the commitment to education starts at the very top. When you look at the performance of the city of Seattle, Mayor Norman Rice took office in 1990 and convened an education summit that included the business community, educators and ordinary citizens.

And he asked Seattle residents to approve a special tax. The money would be earmarked for education. In the middle of a recession, the voters overwhelmingly said yes.

"Really, it's in my self-interest. People choose to live in a city for about three different reasons: one, education; two, public safety; three, the amenities. So if I want people to stay in my city, come to my city, live in this city, education has to be a high priority of this mayor, and so it's one that's going to be there for the long term."

That priority is exemplified by an elementary school that serves the working class community of West Seattle. The neighborhood is called Alki, the school is Alki Elementary. Not long ago, Alki exemplified something else altogether.

Pat Sander was named principal four years ago. During her first month on the job, a newspaper article listed Alki as one of the 20 worst schools in Seattle.

"Well, probably the best thing that happened for me was that the scores had been falling, and the newspaper article came out, because it gave the parents, the staff and myself a place to rally around that was an objective place - that we needed to make a difference and do something different for children."

She took over a school in which almost half the students are children of immigrants, mostly from Southeast Asia and Mexico. Visit the lunchroom and you'll hear 20 different languages and dialects. In addition, a majority of students qualify for free or reduced lunch. But Pat Sander expected academic excellence, and demanded the same from her teaching staff.

"What I did was actually start to focus on the instructional act. I'm a firm believer that if you can get the instruction and the instructional act going in the right direction, that some of the other things will come along, that student self-image increases as they find that they're being successful."

One of Pat Sander's first moves was to get the non-English speaking students into regular classrooms as quickly as possible.

"We've tried to do an immersion-more technique, and instead of having those children pulled out where they're in a room with other children that are speaking the conglomerations of languages and dialects, we have looked at instructional placements that are appropriate to them, and mainstreamed them in with the rest of the children, and so consequently the English language then seems to take off much more quickly with those children."

The same approach applies to children with learning disabilities or physical handicaps. They spend a small part of the day with special ed teacher Tove Andvik, or in individual instruction. But the majority of their time is spent in regular classrooms.

"They need to be in class. They will learn more by learning from their peers and from their teachers, and by learning how to use the textbooks, and how to listen and how to take notes."

Not only was there a major effort to get everyone in the mainstream, the mainstream itself was radically restructured. Alki and some other Seattle schools follow the Early Childhood Model, an idea pioneered by Louise McKinney who directs the Office of Academic Achievement.

"In our Early Childhood Model schools, we do not believe in a cage for every age. And so we say taught at an appropriate level of instruction, in other words, at an appropriate instructional level for every child, which might not be grade level appropriate. So we organize the curriculum so that it is not grade leveled."

Second and third graders are grouped together in what's called the primary level. Fourth and fifth graders at the intermediate level. There's less emphasis on what grade a child is in, more emphasis on how well a child is learning.

"I try and get into the classrooms every day and monitor what's going on in there. We move the children as they need to be moved. Through the frequent monitoring, if we start to assess that a child's taking a growth spurt in education, then we'll move them to a group that's more appropriate to them. If we find out that maybe they've slowed down a little bit, and the group that they're with is moving at a faster pace, then we'll move the child again. So it's not a tracking system once they're put into a group, but we monitor that and watch where they're going."

But the Early Childhood Model is more than just ungraded classrooms, it's an entire philosophy.

"The learning of the child is what matters. But you see there was a time when this nation wasn't called upon to educate all of its children. We were called upon to educate the leaders which were about 20\% of our student population, and then we had to educate about 30\% of our student population to assist that 20 percent, and the rest of the folks could go into factories or they could be throwaways. But in this information age, all of that has changed. There are no factories for the 50 percent to go into. This is a day where all children must learn, all children must be able to be fully employed, all of our children have to become thinking individuals because they're no rote memory kinds of job for just hard working people who aren't able to think in a critical way."

The overriding belief at Alki is that all students can learn not only the basic skills, they can also learn to be critical thinkers. Colleen Dumas' class is one example. In this exercise called Private Eye, kids look at a magnified image of an everyday object, then draw what they see. The analogies are intended to stretch the imagination.

"Analogies, showing relationship is an important skill. I'm finding that the analogies and their ability to see relationship transfers. It transfers into their science, into their social studies, it integrates the curriculum more."

The idea of an integrated curriculum is in evidence throughout Alki. The theme in Fred Pruitt's class is the tropical rainforest. He uses that theme to teach not only science and geography but also language and spelling. When the class breaks into groups, the students are asked to compose a letter about the rainforest. Later in the week, the integration even extends to mathematics.

"I'm going to be with these children for one year, and my goal is to help them want to learn, and to learn as much as they can in that one year, 'cus they're not going to get that chance again."

The integrated curriculum starts early at Alki. Diane Tompkinson teaches first graders. She's known for making the most of every minute in the classroom, an example is calendar math. It's the 18th day of the month - a perfect opportunity to work with the number 18. Another technique is called story painter. While Diane Tompkinson reads a story, two children paint their vision of what the story's all about.

"You get a lot better listening skills when they try to imagine what is going on in the story and particularly when they know they're going to have an opportunity to paint their favorite part, and that we are going to work with a story and describe that chapter."

When an idea like story painter proves effective, it's passed on to the rest of the staff and the entire district. Joan Dorr is a staff development specialist from the Office of Academic Achievement. In this session, she reads a story while Alki teachers do the painting. This is one element in a continual search, a search for new and better ways to teach.

"The more strategies and methods you have, the more children you reach, and the more interesting it becomes. And I think that's a key thing we're finding nowadays is letting children know there is more than one path to the house, and they're all acceptable and appropriate and so that you have children taking risks in learning."

Laurie Eba and Diane Tompkinson are teaching their first graders about sea life. They're about to take a field trip to the beach. The first step is getting the students ready to learn. The next day, they're off on the field trip. 50 first graders, newly certified as beachologists.

"They become totally engrossed in what it is that they're doing, and the motivation is there, the feeling that, god we want to learn all we can about it, and it just seems to create a feeling in them that they can do it, and they want to do it, and it has meaning to them, and I think that's the whole point of what we're doing for kids is to create meaning for them."

Later in the week, the students use language arts and spelling to put together their own books about the field trip, and their study of science continues as well. This week filled with learning was due partly to the parents of these children, some of whom went along on the field trip to help supervise. And the role of parents at Alki goes far beyond field trips.

Alki is part of the accelerated schools model developed at Stanford University. The model calls for a site management council, somewhat analogous to a board of directors. The council includes the principal, staff members, and four parents. One of those parents, Bill Stahlzer, is the chairman.

"We have the power to make decisions in this school about what will happen in this school. We as a site council have decided that our role will not be advisory but in fact that we will make decisions about what happens at the school consistent with district policy."

In addition to serving as chairman of the site management council, Bill Stahlzer also reorganized the computer room at Alki.

"I can make a difference, and parents can make a difference at the school. I would say probably of the 330 something children in the school last year, 250 to 300 of them were on the computer, you know, every week at least once or twice, okay? And that happened because of parents, okay, because of my helping get it started and there were parents down here supervising, because we don't have staff to do it, you know staff doesn't have extra time, we couldn't get a specialist, so it was done with parents."

Another parent on the site management council is Ngy Hul, a native of Cambodia who came to America in 1975.

"This is the first time that I [have] heard [of a] site counsel [where they] have parents who are minority to be part of it, especially [someone] like me, who's refugees and speak broken English to be part of it, you know. All the [other] school, I heard that [they] did not even want to hear what the parent had to say."

Ngy Hul's daughter is at the intermediate level at Alki. He speaks for her and for all the other children of refugees.

"It's very important that all of us must work together, be open mind[ed,] and try to understand each other, so when I [first got] involved in this one not just only [the]psych counsel, I try[tried] to someone say well, you know, telling them about my culture and all the people culture, so then we can try to understand each other and can work well together, you know, rather than I hate that person but I never talk to that person, you see, before we hate someone and disclose someone and understand that person before we hate them."

Throughout Alki there's an obvious respect for the many different cultures that come together here. At the same time, the overall goal is to give these children the tools they'll need to succeed in their new country.

Four years after Alki was named one of the worst schools in Seattle, it's now among the best. Alki is well above the district average in math and reading, and despite the bilingual population, the school equals the district average in language arts. Pat Sander feels the same thing can be achieved anywhere.

"You take the group of people, you find the unity of purpose, you take the empowerment and you build on the strengths. And in no two places are they going to be the same, so you take a look at what's the best of what's out there that you know, and you look at what's going to work with the people that you're working with and that student population that you have, and you figure out as a facilitator how to make that happen in the school that you're at."

Across the water from Alki there's another remarkable elementary school that's in the middle of a dramatic turnaround. Three years ago, the Beacon Hill School ranked near the bottom of all Seattle schools. It's now above the district average in math and catching up quickly in reading and language arts. Not coincidentally, Beacon Hill's improvement began when Gary Tubbs took over as principal.

"This is my 17th year, I've never seen a child who can't learn, and I've worked with special education children, I've worked with children of all poverty and affluent levels, those children can learn too, we just haven't in some cases found the right combination to enable them to learn."

One quarter of Beacon Hill students are white. Most of the rest are either black or Asian, and 71 percent are on free or reduced lunch status. Like Alki, Beacon Hill follows the Early Childhood Model, which means children aren't given labels.

"The rationale behind not labeling children is that the labels seem to stick. People are very comfortable with labels for one reason or another, and people have done so out of the goodness of their heart. We'll put all these kids together and we'll give them all the special attention so they can catch up to the mainstream. It doesn't happen, the gap widens. It's keeping them in the mainstream and supporting them in any ways they need to be supported so that it can be successful in the mainstream, that actually gets them closer to surviving in an independent manner in the mainstream."

At Beacon Hill, students are grouped by grade level in the way that's common to most elementary schools. But there's an uncommon priority placed on high-level thinking. Gretchen Jacobson has her class form fictitious companies to invent new products. These are third graders but they're learning about concepts like marketing and advertising.

"A very small percentage of a child's time spent in school, especially in the urban setting. It's really important to get as much as you can in there, in depth, not coverage but in depth, and by doing the things I do in my classroom, I'm pulling the outside world and integrating it in, I feel as if I'm, you know, I'm getting a lot in there. They're still learning their basic skills, but they're seeing the connections there."

The local business community is encouraged to lend a hand. Davindra Chainani works for the software giant Microsoft. He devotes a few hours each week to work with Gretchen Jacobson's children.

"What I want for my kids is for them to feel as if they have some control over their fate. That they really can make things move and be a self-starter, and an independent learner, and I think if you can teach youngsters to be independent learner then that's going to carry them through the rest of their life and they're going to find success at any level."

Gary Tubbs believes every child can learn, and much of his time is devoted to making sure that happens. Once a week he convenes a child study team to talk about special situations. The team tries to solve any problems that stand in the way of a child's learning. He's also devoted to maintaining discipline. Sometimes it's done in a group, sometimes one-on-one.

"Behavior that is disruptive to, that interferes with the teachers' teaching, or interviews interferes with other children's learning needs to be dealt with immediately and specifically and fairly and lovingly and it's not just one person's job. Again it's one of those dynamics of a successful school that everybody works together."

This principle also likes to lead by example and tries to teach as often as possible.

"I have a real hard time leading people down a path that I haven't been myself. It keeps my skills sharp so that I can be a better supervisor for the teachers on my staff. It raises my level of credibility with the staff because I'm willing to get up and teach at a moment's notice and work with their kids and get right down on the carpet with the kids and I'm able to then go in and evaluate a difficult situation because I've been there myself."

When Gary Tubbs arrived at Beacon Hill, he drew up an action plan that detailed where the school was, and where he expected it to go.

"My role as a principal is to keep the dream alive in this school and that's one of the reasons for the action plan, is to be the cheerleader. And it's not just putting the vision down on a piece of paper and saying now we have a vision, it's truly believing in it and truly having it in your heart as well as in your head that you live it every day and it becomes where all of your energy goes - you continually work to channel a greater percentage of your energy into that vision."

Gary Tubbs at Beacon Hill. Pat Sander at Alki. They've led their schools to higher levels than most people thought possible.

"Not one of us can do it alone, the parents know certainly the most about their children, they know more than we do, the staff brings that educational background and professional experience, the curriculum and instruction, and hopefully I bring the administrative background that that can maybe pull it all together."

In other words, running an effective school is kind of like jumping rope - it requires exquisite teamwork. It also requires a clear vision, and a leader who is never quite satisfied.

"As soon as it looks like you're getting to your vision, I think your vision becomes clarified and it's further away than you even thought it could possibly be, but that in itself is progress."

Despite its magnificent setting, Seattle faces most of the same problems that plague other American cities. One way out, perhaps the only way, is through superior education for every child.

"I think that if [our] education system doesn't speak to all of our citizens, then it is not a a good education system, and I think what we believe in very strongly is every child can learn, and if we put them in the right environment and give them the right support around them, those children can learn as well as anybody and that's what we're out to prove and that's what we're proving."

And two elementary schools, Alki and Beacon Hill, are leading the way. Both schools are shining examples of what can happen when an entire community decides that education is truly the key to the city.

\section{Charles Rice Learning Center: An Oasis in Dallas}

\subsection{Audio Summary}

Charles Rice Learning Center: An Oasis in Dallas shows how an inner city school has students who consistently win city-wide math and computer contests, and particpate in a nationally known performing choir.

\subsection{Transcription}

From a distance it's a shining city. But look to the south and Dallas loses its luster. This is the city's underside, not just geographically. Unemployment, crime, drugs - South Dallas is home to all the familiar urban diseases. But amid the poverty, amid the hopelessness, there sits an oasis in South Dallas. An oasis of learning.

Walk into the Charles Rice Learning Center and it's the first thing you notice, the trophy case. The awards aren't for athletics but for academics, and above the trophies a portrait of the person most responsible: Louise Smith.

Louise Smith was named principal in 1984. She took over a school that was, in a word, mediocre.

"Sometimes a lot of these schools, they are comfortable with their level where they are, they are comfortable with mediocrity and I'm very uncomfortable with mediocrity. If I find people who are satisfied with mediocrity then I stay away from them because that's my discomfort zone, when I see people who are dissatisfied with just status quo."

Status quo at Charles Rice was underachieving teachers, underachieving students, which surprised almost no one. Not much is expected from schools in South Dallas, and not much is expected from students.

"Kids from this area, they are the ones who fill up the jails. This is the highest crime rate area in Dallas, drug houses abound and the children all know where the dope houses are. They can drive down the street and point to you, 'this is the dope house', or 'this is the dope house'."

Despite the crime, despite the drugs, Louise Smith set a goal of academic excellence.

Charles Rice had been designated as a learning center. That meant more funds for facilities and equipment. But turning a school around required more than just money.

"Money definitely is not the answer. It's not the building, it's not the materials that you have in the building, it's the people. It's the people within the buildings that make it work or not."

So step one was finding the right people, namely dedicated teachers. This is Gail McVeigh's sixth grade social studies class. The students are learning about barter in ancient Africa, using some modern materials.

Gail McVeigh arrived at Rice at the same time as Louise Smith.

"She is dogged in her pursuit of excellence. She lays the cards on the table when she interviews you. If you cannot meet her expectations, there are no hard feelings but you'll have to find another place to go."

"I never actually had to get rid of a teacher, but the way we were able to get this school on such a high standard, some teachers who came saw that they couldn't keep up, so they asked to leave."

There was also a conscious effort to actively engage students in learning. Linda Jennings' fifth grade math class is one example. The chore of reducing fractions is turned into a game.

Then there was the matter of discipline. The school motto at Charles Rice is "Excellent academics, excellent behavior."

"No child can learn if they're talking when the teacher is talking, they miss out on too much. And we work on that right along with academics. We carry it both right along together, and we work on both equally and strong."

Good behavior is paramount at Charles Rice. A specialist is even brought into council students individually and as a group.

More than anything else, Louise Smith actually expected Charles Rice to achieve academic excellence. That's a term you hear often at this school: high expectations. Victor Washington has taught math at Rice for eight years.

"If you don't expect anything out of them then I don't give you anything, but if you push and push and push and expect this and tell them, you know, I'll help you in any way that I can, then they'll give you a lot."

Expectations were so high that Charles Rice started competing against elementary schools throughout the district, including those from the wealthy suburbs. For example, there's the annual Math Olympiad. Rice students trained for the event after school under Victor Washington and Irene Redmond.

When Rice first entered the Math Olympiad, Louise Smith expected nothing less than first place.

"Certainly, certainly, whenever I send kids out from Charles Rice to a competition, you see, I would motivate them first. I would build them up to tell them they're going to win, and then I would tell them when they were going out that I expected them to bring trophies back to me."

But almost no one else expected much from this inner city school.

Mary Lester is director of mathematics for the entire district and runs the Math Olympiad.

"Charles Rice, when they first entered, they were so confident that they would win, [I said we chose Rice], you know, no, they're not going to win. And yet they persisted, and each year, that year they did win some individual first place awards, and since then they become a power to be recognized in the city and people don't take Charles Rice lightly anymore."

"I just like to prove to people that my kids are just as good as anyone else's and they can win, which they have all the time, so they can win, doesn't matter what color they are, you know, they can win and they can compete with Hispanic kids, white kids, or whatever."

The reach for excellence extended into every subject and every activity. This is the kindergarten through third grade choir, preparing for a tour of the East Coast. So within the course of a few years, Louise Smith and her teachers turned this school into a showcase. In fact, in 1986, William Bennett, then Secretary of Education, visited Rice and cited it as an example of what can be achieved in urban education.

Among the elementary schools in Dallas, Charles Rice ranks in the top 15 percent in both math and language. But there's one catch to this story, Louise Smith is no longer principal of Charles Rice. She's been promoted to area director and is now in charge of 13 schools, Rice among them.

The new principal is Dr. Geraldine Hughes, a woman known for her affectionate ways.

"I'm a hugger. That means that I will walk by and I'll hug you. It's simply says to staff members that I'm concerned about you as a person as well as as a teacher. And I hug children on a continuing basis. Too many times we don't show children how we really feel, and that hug is my way of showing them how I feel about them."

Hugging aside, this new principal brought with her the same high expectations as Louise Smith. That was the first thing she told her teachers and students.

"I simply said to them the principles have changed but the rules have not. I believe children want people to demand certain things of them. If I look back, those persons or teachers I remember are those who were tough, and sometimes that's what has to happen, it has to be tough love, but we care enough for the children to demand the very best from them."

You can look around the school and see that not much has changed. There's the same emphasis on learning, the same emphasis on good behavior. The insistence on proper behavior extends beyond the classroom, it includes every activity associated with a school. On this day, two kindergarten classes were off on a field trip. This was a visit to the shop that makes all the street signs in Dallas. Field trips are seen as a valuable connection to the world beyond the borders of South Dallas, so the school organizes as many as possible, often to the symphony and other cultural events. And no matter where Rice students go, there's always a sense of discipline and order.

And that is just fine with the parents of most of these children. Carlisa Dixon, for example, is a fifth grade honor student. She hopes someday to be a writer. Her mother Glorious, who single-handedly raised Carlisa and an older son, expects her children to achieve and to behave.

"I don't want kids to be disruptive in class, I don't want anybody abusing my child, but if my child gets out of line, and my child needs to be disciplined at that particular moment, yes you have my permission to do that. And my kids know that, and then they know that if they get in trouble then they come home and they get in trouble again. Kids have to know that parents and teachers are working cooperatively. I mean all a kid needs to know is that one of those elements is not working in harmony with the other one, and they're going to take advantage of that."

"I believe that we have to continue to work so that first we take responsibility for students' behavior and ultimately they take responsibility for their own."

But getting a child's attention is one thing, keeping it is another, so the primary focus here is still on one thing: top-notch instructions.

"If I were to look at a Charles Rice teacher, and say this person portrays what we want: they have a love of teaching that transcends just the regular hours, they enjoy seeing students learn, they enjoy the art of teaching, and they also know the science of teaching.

Science specialist Georgia Beatey is a perfect example. She's been a teacher for 20 years, the last eight of them at Charles Rice.

"We're not a textbook oriented school where we just simply, you know, start at chapter one and proceed through chapter 14. And there are a lot of people in education who came about their teaching credentials that way, you know, you just plod through, and we're not like that, we're very different."

Rice is the only elementary school in Dallas with a science specialist. It means the regular science teachers bring their students to the lab once a week, and it means Georgia Beatty teaches 350 children. In this class she uses two simple ingredients, Elmer's glue and a borax solution, to make a substance that resembles silly putty. The whole point is to teach the students about the different properties of matter.

"It's putting practical things to use, whatever they are. Now, why other schools don't do it, well some teachers are afraid of science, they're afraid to get dirty, they're afraid to make a mistake, they're afraid to say 'I don't know'. There are many schools in Dallas and elsewhere where they don't teach science, and you know the kids, it's one of their favorite subjects."

It's evident that two decades of teaching haven't dimmed Georgia Beatty's enthusiasm.

"Seeing kids, when the light dawns, 'ah that's what this is', seeing it come on on their own, that's exciting. I like what I do, and if I stop liking it then I need to do something else."

That love of teaching is evident throughout Charles Rice. Raymond Greer's specialty is music, but he goes beyond that.

"A lot of people think that music and band and the strings, they do not assist the main quote-unquote math, science and they do not fit in, but you can teach anything with music."

Everything is aimed at producing well-rounded students - students who are proficient in more than just the basic skills.

"It goes back to the basic philosophy that 'I am here for the children', and whatever I can do to expand their life, to expose them to new literature, to new ideas in life that are educationally sound, that's my job, that's just being a teacher."

Firm discipline, sound teaching techniques, high expectations, strong leadership, it all seems relatively simple.

"It is simple. It's not a unique formula, it takes people garnering their energies together for this end, and the end being that you're producing high-achieving, self-motivated students who you can be proud of, who can do this thing out here that they call education."

One program intended to produce self-motivated students is something called Project Seed, a requirement for every student in grades four through six. At first glance Project Seed is simply a different way of teaching advanced mathematics. But actually this is much more than just math.

William Glee has been a Project Seed specialist for 10 years.

"We use mathematics as a tool, and the tool is to raise the self-confidence and self-esteem. You can raise their self-esteem by teaching them how to play chess, but when you teach them the mathematics and get them excited about their learning, education, because education is the key for a lot of these kids to be successful."

Project Seed instructors are all trained mathematicians. They take turns monitoring each other's classes, always trying to improve their teaching, and everything is taught using the Socratic method, questioning, then questioning some more.

As for the curious hand signals, they've become almost a trademark of Project Seed.

"When I look out to a class I see this, I know we have a few students that are disagreeing. But if I see students doing like this, that means I know they agree, so at any given time in a class, I can see exactly what they're thinking. So at all times I'm more like a conductor - they do all the manipulation, they do all the work, they do all the breaking down. I just interject a question when needed, just to keep the ball rolling."

Another part of Project Seed is the participation of the regular math teacher, in this case Victor Washington. He and William Glee see themselves as a team, working toward the same goal.

"The first year I had Project Seed, I wasn't very happy with it. Then year after year, I've noticed how the kids enjoy it, how the instructors teach, which I like a whole lot and the kids scores are better, so I say I'm sure I'm not doing this by myself, so Seed must have something to do with it, and now I welcome it all the time."

Project Seed has been effective among all minority students. This is a class at the Elario Martinez School in West Dallas. The population here is almost entirely Hispanic.

"I look at the top students in all ethnic groups. The students that start out in the fourth grade and the minority schools in the deprived areas, when they reach middle school and high school, those top students can compete with any type of student in any school in the nation."

Project Seed seems to contribute to higher test scores. Higher confidence is somewhat harder to measure, but it's apparent in the faces of these students.

"They're so excited about what they're doing, they believe in what they're doing, they believe that they can do [it], and it's just that goal in their eyes, and that's the magic. The magic happens with the students, and what the students see that they can do."

Charles Rice is not a school without a weakness: its Achilles heel is reading. In recent years reading scores haven't quite measured up to those in math and language, so right now the school is trying to strengthen that area. Gail McVeigh teaches language arts as well as social studies. She's constantly looking for new ways to get her students to read.

"You know I may read half a novel and then tell the students they'll have to find out by themselves, by reading the rest of it what happens at the end."

Gail McVeigh also encourages her students to write as much as possible. So when her students visit the computer lab, they're expected to go beyond simple keyboard exercises.

"The students love to write about what I have read or what they have read, and I allow them a lot of leeway in what they will write about. You know, to them it's an opportunity to express themselves and to relate in some way to literature."

The emphasis on reading is evident throughout Charles Rice. Every school day starts with 10 minutes during which students and teachers drop everything and read to the accompaniment of classical music.

"The basic idea is that children should see and have some time for what we call sustained silent reading, but the children get to select any material they want for that period. They also get to see their teacher, and their principal and others stopping at a point in time to read, and the idea is that everybody in our building will read for 10 minutes at that time."

So Charles Rice keeps rolling along. For that matter, so does Louise Smith. She now travels among her 13 schools offering guidance and wisdom. And even though she hasn't taught class in more than 20 years, Louise Smith can't resist taking over a classroom once in a while.

"Whenever you're an educator you're first and foremost a teacher. As principal of a school, I'm a teacher of teachers, as an area director I'm a teacher of principals, so I'm constantly teaching."

After eight years as principal at Rice, Louise Smith is officially an outsider, but her presence and her influence remain.

"Even though she's not principal here, they would still say you know, 'Louise Smith would not have that' and you would not do that, you wouldn't do it if Louise Smith was here, so you're going to act the way are you supposed to act, you're going to act the way if Louise Smith was still principal here, so she still has a big impact here. I don't think she's, she hasn't left yet, it will be a couple of years down the line but, right now she's still here."

But with every passing day, this school belongs less to Louise Smith, more to Geraldine Hughes.

Things seem to be in place for Charles Rice to maintain its record of success. There are still outstanding teachers, a committed administration, caring parents, and everyone is striving together for the same thing.

"It's excellent, that's all they want, excellence, and I think that's just something that has been embedded in everybody, teachers, and I think that Dr. Hughes is you know just in a wonderful position now to come in and keep that going. This is her first year, you know, parents are gonna watch because parents at Charles Rice expect that certain things happen."

"Louise Smith did pass on the baton to me, and she kind of moved to the side and said, you know now Dr. Hughes, you do it. And I feel like it is a race, it's one that you have to ever be diligent, and you have to look behind you and in front of you, and you have to keep moving, and you have to move fast because there's so much to learn and so little time to do it."