\chapter{Every Child Can Succeed 7}

\begin{figure}[H]
    \centering
    \includegraphics[width=\textwidth/2]{./Games/EveryChildCanSucceed/Images/EveryChildCanSucceed7CD.png}
    \caption{Every Child Can Succeed 7 CD}
\end{figure}

The final of the Every Child Can Succeed games published and released by The Lightspan Partnership for the PlayStation 1.

Every Child Can Succeed 7 features three video programs:

\begin{itemize}
    \item L. A. Stories: Bennet-Kew and Euclid Avenue
    \item Thorncliffe Community School: Student-Centered Learning in Edmonton
\end{itemize}

\clearpage
\newpage

\section{L. A. Stories: Bennet-Kew and Euclid Avenue}

\subsection{Audio Summary}

L. A. Stories: Bennett-Kew and Euclid Avenue investigates how two schools in the Los Angeles area are achieving success with increasing bi-lingualism and the urban problems of gang violence and crime.

\subsection{Transcription}

\section{Thorncliffe Community School: Student-Centered Learning in Edmonton}

When I first came here, this was in '74, and the first test score state test scores came out. Our third graders were reading at a three percentile. It took us four years to get them over the 50 percentile. "Well, let's go where you belong," no one wanted to come to this school, either as a teacher nor as an administrator. And I certainly can tell you I didn't want to be here. I wouldn't trade this school for a minute now, not for a minute.

This program tells the story of two schools in Greater Los Angeles. Our first stop is the Bennett Q Elementary School in Inglewood, California, just outside the Los Angeles city limits. There is a solid middle-class core in the region served by Bennett Q, but the school also serves a large poor transient population. Nearly 60 percent of the students who attend Bennett Q are eligible for the subsidized lunch program.

Despite the demographic odds, Bennett Q's achievement scores have improved considerably. For example, third-grade math scores have moved from the lowest to the highest quartile in the state. This success has taken place under the leadership of Principal Nancy Ichinaga. "People have a notion that kids from minority backgrounds, in particular, from urban settings and in urban public schools, really are not or cannot do very well. And I really have to refute that point of view."

"Our kids do as well or better than kids just about anywhere. First of all, we believe that the school must be a safe place for all kids. We also expect the kids to come to school believing that it's their time to be serious students. And we do believe that every child can learn to read, to write, and to be a successful student."

"Lulu, you need to work with five. Remember, let's start all over again. We have worked out a system where everybody is taught specific things that make them successful students. And it's what every teacher does in there for me. The system begins in kindergarten where Teacher Charlotte Watanabe is laying the foundations for later success in reading and math."

"Kindergarten is very important. Very good. Do you think you'll be able to finish it? Oh, terrific! We want to see that. We don't make them feel good about themselves by making them succeed in whatever they do in the classroom. You know they're going to start feeling like failures from the very beginning. So, we want them to succeed here. We want them to feel good about what they do. So, whatever tasks they have, we give them a lot of praise."

"What we do here in kindergarten with all these manipulatives is to build a good foundation of the understanding of the concepts we're building. So that when they go up to first grade, second grade, to the upper grades, they'll have a strong understanding of what exactly they're doing in those grades when they are working with computations."

"The building block approach continues through the primary grades so that by the time students reach Howard Rothenberg's fifth grade, they have a solid grounding in the basics that they need to achieve success in math. Obviously, if you don't have the foundations from first, second, third grade, you can't begin fourth grade where you need to start. And you have to go back and remediate, and you're losing time and the children. And that's what happens in most inner-city schools. The children are years behind because every year has to go back and repeat what they did not learn previously. Well, we do not have that problem here for the most part."

"Whenever new students enroll at our school, I do a quick check. I do it because I want to know where they're at academically. I also want to get to know them, and I want to meet the parents and do a little work on letting them know what they're all about. Okay, I see two problems. Number one, you know math. Okay, number one, you got to learn your fractions. And I want to talk to your mom so she can go to the drugstore and buy you a workbook on fractions, and you can work on it. And I'll teach you how to. One of the very important things about our school is that our kids are held accountable for their learning. In other words, we have expectations by grade level for every kid in spelling."

"She's a little behind. Think about a 3.6. Okay, now in math, she's right. She's... and the kids and the parents know from the very, very beginning that if they are in kindergarten, these are the things they have to master before the year is over. And if they don't, they probably will not go on to the next grade. So, a kid doing real good in fourth grade would be able to do all these problems."

"Bennett Q's curriculum focuses on math and reading. Reading presents a special challenge since nearly 40 percent of the students are Hispanic, many with English as their second language. Special classes like this one are available to help Spanish-speaking children improve their English comprehension. 'I do believe that if you want to learn English, you have to learn English. I don't believe that you're going to learn English by learning Spanish.'"

"Parents, when they came to our school, expected the kids to be taught in English. They didn't expect us to be able to do it in Spanish. Let me see a big frown. Every time I say let me see a frown, it makes us want to smile, doesn't it? Let me see a smile. Which is better? Wow! Oh, I knew you were going to say that. Okay, teach reading."

"The Bennett Q curriculum relies heavily on phonics. It's a curriculum that the faculty strongly supports since they helped to write it. By the time they reach third grade, Bennett Q students are reading and discussing books like 'Charlotte's Web' under the guidance of Teacher Mary Lou Reingold."

"Okay, discussion question. In 'Charlotte's Web,' as in every book, there's a theme or a main idea, and one of the most important themes in this story is about friendship. A really important aspect of friendship is sacrifice. And who can give me an example in 'Charlotte's Web' of when Charlotte... when did she sacrifice?"

"When she said she wasn't... um, it was too inconvenient for him for her to go, but she went anywhere, nowhere, to the fair. Exactly. Okay, now think in your life. When have you been sacrificing? Remember, I know you've all helped people. You're great helpers. But when have you helped when you didn't want to?"

"When I was... um, I was sleeping at home and my knee, she broke the lamp where she was throwing balls, and um, she came to me in my bedroom. She woke me up, and I told her I didn't want to help her to stay out of trouble because I was too sleepy. But did you help her? How come? Because I didn't want her to get in trouble. That's neat. That's a great example."

"It's exciting to see these children being able to read and to be able to digest a story like 'Charlotte's Web' and get the meaning out of it. The school's special because it's run so well. Nancy, the principal, is ahead of her time. She knows where we all need to be moving, so she keeps us on

track."

"I monitor what happens in the classrooms by periodic tests which the teachers all give, and every teacher in every grade level gives the same test. By carefully monitoring the performance of each student, Nancy Ichinaga ensures consistency across the curriculum and helps teachers identify and correct any weaknesses in their approach."

"Okay, to find out or to make out. Okay, so do that. Okay, look at your post here. Nancy and Cheryl Draper brainstorm with new fourth-grade teachers to find better ways to assess the writing skills of their students. Could you guys kind of take a look at them and come up with something that's more usable for you guys? Could we even give it instead of one, three, or four? Could we give it a content score and a mechanic score?"

"These assessments really help us stay on focus. It helps particularly the new teachers because you know they really have no idea what it is they're supposed to teach, and this way they learn this is what we need to cover. And so they're they're more successful. You know right about real stuff."

"Nancy shows her support for her teachers in many ways, in small ways and in large ways. Whenever someone's having a problem, and Nancy needs to deal with it, I feel she approaches it as a group, as a team effort. You know, we can, we need to work on this. You know what can we do? How can I help you to change whatever it is she feels needs to be a little better?"

"Whenever new students enroll at our school, I do a quick check. I do it because I want to know where they're at academically. I also want to get to know them, and I want to meet the parents and do a little work on letting them know what they're all about. Okay, I see two problems. Number one, you know math. Okay, number one, you got to learn your fractions."

"And I want to talk to your mom so she can go to the drugstore and buy you a workbook on fractions, and you can work on it. And I'll teach you how to. One of the very important things about our school is that our kids are held accountable for their learning. In other words, we have expectations by grade level for every kid in spelling."

"She's a little behind. Think about a 3.6. Okay, now in math, she's right. She's... and the kids and the parents know from the very, very beginning that if they are in kindergarten, these are the things they have to master before the year is over. And if they don't, they probably will not go on to the next grade. So, a kid doing real good in fourth grade would be able to do all these problems."

"Bennett Q's curriculum focuses on math and reading. Reading presents a special challenge since nearly 40 percent of the students are Hispanic, many with English as their second language. Special classes like this one are available to help Spanish-speaking children improve their English comprehension."

"'I do believe that if you want to learn English, you have to learn English. I don't believe that you're going to learn English by learning Spanish.' Parents, when they came to our school, expected the kids to be taught in English. They didn't expect us to be able to do it in Spanish."

"Let me see a big frown. Every time I say let me see a frown, it makes us want to smile, doesn't it? Let me see a smile. Which is better? Wow! Oh, I knew you were going to say that. Okay, teach reading. The Bennett Q curriculum relies heavily on phonics. It's a curriculum that the faculty strongly supports since they helped to write it."

"By the time they reach third grade, Bennett Q students are reading and discussing books like 'Charlotte's Web' under the guidance of Teacher Mary Lou Reingold. Okay, discussion question. In 'Charlotte's Web,' as in every book, there's a theme or a main idea, and one of the most important themes in this story is about friendship."

"A really important aspect of friendship is sacrifice. And who can give me an example in 'Charlotte's Web' of when Charlotte... when did she sacrifice? When she said she wasn't... um, it was too inconvenient for him for her to go, but she went anywhere, nowhere, to the fair. Exactly. Okay, now think in your life. When have you been sacrificing?"

"Remember, I know you've all helped people. You're great helpers. But when have you helped when you didn't want to? When I was... um, I was sleeping at home and my knee, she broke the lamp where she was throwing balls, and um, she came to me in my bedroom. She woke me up, and I told her I didn't want to help her to stay out of trouble because I was too sleepy."

"But did you help her? How come? Because I didn't want her to get in trouble. That's neat. That's a great example. It's exciting to see these children being able to read and to be able to digest a story like 'Charlotte's Web' and get the meaning out of it. The school's special because it's run so well. Nancy, the principal, is ahead of her time."

"She knows where we all need to be moving, so she keeps us on track. I monitor what happens in the classrooms by periodic tests which the teachers all give, and every teacher in every grade level gives the same test. By carefully monitoring the performance of each student, Nancy Ichinaga ensures consistency across the curriculum and helps teachers identify and correct any weaknesses in their approach."

"Okay, to find out or to make out. Okay, so do that. Okay, look at your post here. Nancy and Cheryl Draper brainstorm with new fourth-grade teachers to find better ways to assess the writing skills of their students. Could you guys kind of take a look at them and come up with something that's more usable for you guys? Could we even give it instead of one, three, or four? Could we give it a content score and a mechanic score?"

"These assessments really help us stay on focus. It helps particularly the new teachers because you know they really have no idea what it is they're supposed to teach, and this way they learn this is what we need to cover. And so they're they're more successful. You know right about real stuff."

"Nancy shows her support for her teachers in many ways, in small ways and in large ways. Whenever someone's having a problem, and Nancy needs to deal with it, I feel she approaches it as a group, as a team effort. You know, we can, we need to work on this. You know what can we do? How can I

help you to change whatever it is she feels needs to be a little better?"

"Whenever new students enroll at our school, I do a quick check. I do it because I want to know where they're at academically. I also want to get to know them, and I want to meet the parents and do a little work on letting them know what they're all about. Okay, I see two problems. Number one, you know math. Okay, number one, you got to learn your fractions."

"And I want to talk to your mom so she can go to the drugstore and buy you a workbook on fractions, and you can work on it. And I'll teach you how to. One of the very important things about our school is that our kids are held accountable for their learning. In other words, we have expectations by grade level for every kid in spelling."

"She's a little behind. Think about a 3.6. Okay, now in math, she's right. She's... and the kids and the parents know from the very, very beginning that if they are in kindergarten, these are the things they have to master before the year is over. And if they don't, they probably will not go on to the next grade. So, a kid doing real good in fourth grade would be able to do all these problems."

"Bennett Q's curriculum focuses on math and reading. Reading presents a special challenge since nearly 40 percent of the students are Hispanic, many with English as their second language. Special classes like this one are available to help Spanish-speaking children improve their English comprehension."

"'I do believe that if you want to learn English, you have to learn English. I don't believe that you're going to learn English by learning Spanish.' Parents, when they came to our school, expected the kids to be taught in English. They didn't expect us to be able to do it in Spanish."

"Let me see a big frown. Every time I say let me see a frown, it makes us want to smile, doesn't it? Let me see a smile. Which is better? Wow! Oh, I knew you were going to say that. Okay, teach reading. The Bennett Q curriculum relies heavily on phonics. It's a curriculum that the faculty strongly supports since they helped to write it."

"By the time they reach third grade, Bennett Q students are reading and discussing books like 'Charlotte's Web' under the guidance of Teacher Mary Lou Reingold. Okay, discussion question. In 'Charlotte's Web,' as in every book, there's a theme or a main idea, and one of the most important themes in this story is about friendship."

"A really important aspect of friendship is sacrifice. And who can give me an example in 'Charlotte's Web' of when Charlotte... when did she sacrifice? When she said she wasn't... um, it was too inconvenient for him for her to go, but she went anywhere, nowhere, to the fair. Exactly. Okay, now think in your life. When have you been sacrificing?"

"Remember, I know you've all helped people. You're great helpers. But when have you helped when you didn't want to? When I was... um, I was sleeping at home and my knee, she broke the lamp where she was throwing balls, and um, she came to me in my bedroom. She woke me up, and I told her I didn't want to help her to stay out of trouble because I was too sleepy."

"But did you help her? How come? Because I didn't want her to get in trouble. That's neat. That's a great example. It's exciting to see these children being able to read and to be able to digest a story like 'Charlotte's Web' and get the meaning out of it. The school's special because it's run so well. Nancy, the principal, is ahead of her time."

"She knows where we all need to be moving, so she keeps us on track. I monitor what happens in the classrooms by periodic tests which the teachers all give, and every teacher in every grade level gives the same test. By carefully monitoring the performance of each student, Nancy Ichinaga ensures consistency across the curriculum and helps teachers identify and correct any weaknesses in their approach."

"Okay, to find out or to make out. Okay, so do that. Okay, look at your post here. Nancy and Cheryl Draper brainstorm with new fourth-grade teachers to find better ways to assess the writing skills of their students. Could you guys kind of take a look at them and come up with something that's more usable for you guys? Could we even give it instead of one, three, or four? Could we give it a content score and a mechanic score?"

"These assessments really help us stay on focus. It helps particularly the new teachers because you know they really have no idea what it is they're supposed to teach, and this way they learn this is what we need to cover. And so they're they're more successful. You know right about real stuff."

"Nancy shows her support for her teachers in many ways, in small ways and in large ways. Whenever someone's having a problem, and Nancy needs to deal with it, I feel she approaches it as a group, as a team effort. You know, we can, we need to work on this. You know what can we do? How can I help you to change whatever it is she feels needs to be a little better?"

"Whenever new students enroll at our school, I do a quick check. I do it because I want to know where they're at academically. I also want to get to know them, and I want to meet the parents and do a little work on letting them know what they're all about. Okay, I see two problems. Number one, you know math. Okay, number one, you got to learn your fractions."

"And I want to talk to your mom so she can go to the drugstore and buy you a workbook on fractions, and you can work on it. And I'll teach you how to. One of the very important things about our school is that our kids are held accountable for their learning. In other words, we have expectations by grade level for every kid in spelling."

"She's a little behind. Think about a 3.6. Okay, now in math, she's right. She's... and the kids and the parents know from the very, very beginning that if they are in kindergarten, these are the things they have to master before the year is over. And if they don't, they probably will not go on to the next grade. So, a kid doing real good in fourth grade would be able to do all these problems."

"Bennett Q's curriculum focuses on math and reading. Reading presents a special challenge since nearly 40 percent of the students are Hispanic, many with English as their second language. Special classes like this one are available to help Spanish-speaking children improve their English comprehension."

"'I do believe that if you want to learn English, you have to learn English. I don't believe that you're going to learn English by learning Spanish.' Parents, when they came to our school, expected the kids to be taught in English. They didn't expect us to be able to do it in Spanish."

"Let me see a big frown. Every time I say let me see a frown, it makes us want to

smile, doesn't it? Let me see a smile. Which is better? Wow! Oh, I knew you were going to say that. Okay, teach reading. The Bennett Q curriculum relies heavily on phonics. It's a curriculum that the faculty strongly supports since they helped to write it."

"By the time they reach third grade, Bennett Q students are reading and discussing books like 'Charlotte's Web' under the guidance of Teacher Mary Lou Reingold. Okay, discussion question. In 'Charlotte's Web,' as in every book, there's a theme or a main idea, and one of the most important themes in this story is about friendship."

"A really important aspect of friendship is sacrifice. And who can give me an example in 'Charlotte's Web' of when Charlotte... when did she sacrifice? When she said she wasn't... um, it was too inconvenient for him for her to go, but she went anywhere, nowhere, to the fair. Exactly. Okay, now think in your life. When have you been sacrificing?"

"Remember, I know you've all helped people. You're great helpers. But when have you helped when you didn't want to? When I was... um, I was sleeping at home and my knee, she broke the lamp where she was throwing balls, and um, she came to me in my bedroom. She woke me up, and I told her I didn't want to help her to stay out of trouble because I was too sleepy."

"But did you help her? How come? Because I didn't want her to get in trouble. That's neat. That's a great example. It's exciting to see these children being able to read and to be able to digest a story like 'Charlotte's Web' and get the meaning out of it. The school's special because it's run so well. Nancy, the principal, is ahead of her time."

"She knows where we all need to be moving, so she keeps us on track. I monitor what happens in the classrooms by periodic tests which the teachers all give, and every teacher in every grade level gives the same test. By carefully monitoring the performance of each student, Nancy Ichinaga ensures consistency across the curriculum and helps teachers identify and correct any weaknesses in their approach."

"Okay, to find out or to make out. Okay, so do that. Okay, look at your post here. Nancy and Cheryl Draper brainstorm with new fourth-grade teachers to find better ways to assess the writing skills of their students. Could you guys kind of take a look at them and come up with something that's more usable for you guys? Could we even give it instead of one, three, or four? Could we give it a content score and a mechanic score?"

"These assessments really help us stay on focus. It helps particularly the new teachers because you know they really have no idea what it is they're supposed to teach, and this way they learn this is what we need to cover. And so they're they're more successful. You know right about real stuff."

"Nancy shows her support for her teachers in many ways, in small ways and in large ways. Whenever someone's having a problem, and Nancy needs to deal with it, I feel she approaches it as a group, as a team effort. You know, we can, we need to work on this. You know what can we do? How can I help you to change whatever it is she feels needs to be a little better?"

\subsection{Audio Summary}

Thorncliffe Community School: Student-Centered Learning in Edmonton observes how the principal and staff of an urban school have created a climate of order and calm in which all the students have become positive and enthusiastic students.

\subsection{Transcription}

Does our team work? Do we focus on success? Are we becoming the best that we can be? What is the key to success?

And the reason why we are doing all of this is because we are the students at Thorncliffe Community School. Say they're building their future. It's a far cry from the past when Thorncliff was one of the most dangerous schools in Edmonton, Alberta. Kids were just out of control at school; there was no consequence that would fit their action.

Enter Lou Yanu, who was named principal four years ago. He brought in some new teachers, an entirely new attitude, and a new approach. This was to be a student-centered school.

In a student-centered school, the belief, the generic belief, is that every child can succeed. And they are succeeding. Discipline problems are down, academic achievement is up, and school spirit is off the charts. Thorncliffe Community School has taken off.

It's a big day at Thorncliffe. This is a showdown between two floor hockey teams: the students against the teachers. At stake, the Kuba Saw Cup. As usual, the head cheerleader is Thorncliff Principal Lou Yanu. Right now, the 300 Thorncliff students are finishing a series of intensive assessment exams, which is exactly why this game is being played.

If we can introduce something that children can enjoy and loosen up in regards to and have a little bit of fun with, then it just makes coming to school and going through the assessment period just a little easier for them. And in the gymnasium during the time of the cup, you can just see it on the faces of the students how much they're enjoying this.

The students are enjoying it, so are the teachers, especially when they score the first goal. It's the teachers one to nothing. More about the game later.

Thorncliffe is a school that's undergoing a radical transformation, one that began when Lou Yanu arrived four years ago. What are you doing to mentally prepare them? At the time, Thorncliff was one of the worst schools in Edmonton.

The school sits on the west side of the city, in the shadow of the enormous West Edmonton Mall. By U.S standards, the surrounding neighborhood might not seem poverty-ridden, but three-quarters of Thorncliffe students are considered low-income, and nearly half come from single-parent families. This is a troubled neighborhood with one of the highest crime rates in Edmonton, and many of those troubles were carried over into Thorncliff.

It was a school that in many ways was very dysfunctional. Teachers were suspending students from the classroom to the tune of over 300 a month, out of the classroom for being disruptive and out of control. While some students were unruly, many others were frightened.

James Ross is a sixth grader at Thorncliffe. He remembers how things were before. On the playgrounds, there's a lot of violence and fighting all the time, and you couldn't go outside without being afraid.

Lou Yanu's first priority was to eliminate the fear. He did away with corporal punishment but set up strict rules for behavior. Any student breaking those rules is disciplined quickly and fairly.

You told the boy not to touch your ball. What should you have done then besides kicking him? That's right. Now, when you told Michael not to touch your ball, how did you tell them that?

Ted Davis presides over daily detention. Children who have broken a rule spend their recess time here. Okay, you hit somebody, you're going to have a recess attention. You hit somebody outside the school, there's going to be a consequence for that. If you hit somebody when you're older and it's in a mall or in a dark shopping center parking lot, there's going to be a consequence for that action. And that's what we're trying to get these kids to realize. There's a consequence for all their actions.

Good morning, boys and girls. Discipline is also maintained by positive reinforcement. This assembly is to award monthly Diamond of Distinction certificates to students who exhibit exemplary behavior. Any student who earns a certificate is eligible for a drawing to be held the last day of school. The grand prize for the Diamond of Distinction award will be a brand new 18-speed bicycle. Any idea why you're here? I don't.

Another distinction at Thorncliffe is to be called to the principal's office. At this school, it usually means you did something right. For the most part, Mitch, you've made a lot of changes here at the school. You put in a lot of great effort, and that's why you're here this afternoon.

I received this. Let me tell you what a celebration it is for the students when they come down to the office. I am very impressed, and we have an opportunity to share the good news with one another. And then in each and every case, I take it upon myself in the presence of the student to inform the parent or guardian of that child, whether they be at home or at work. And it is just a wonderful celebration.

Well, that's just great. He just loves that school so much, and he's done so well since he's been going there. If it weren't for the teachers and yourself, he would not do as well as he does.

Lou Yanu also went about transforming the entire school atmosphere. Thorncliff was to be a place where all children could succeed. As part of our belief statement, we believe that all students have three generic needs: one, a need for competency and achievement; two, a need for structure, safety, control, and discipline; and three, a need for positive personalized interaction between peers and adults.

As a staff, then, if we collectively believe this, at our staff meetings, we make decisions as to programs, activities, and events that match that. Oh, wonderful. You're doing great. Can you feel it pulling positive interaction between children and adults?

It's evident in the classroom and on the playground. Lou Yanu does whatever it takes to make Thorncliff as enjoyable as possible. His enthusiasm is obvious; it's also highly contagious.

But discipline and an inviting atmosphere are only means to an end. The primary mission of Thorncliffe is education. If children enjoy coming to school, there's no question they will learn. If children do not enjoy coming to school and, in fact, hate being in class, regardless of how tight your curriculum is, they will learn less than they are capable of. So the teaching-learning process must be interesting; it also must be inviting and fun.

Every facial thing has a feeling. All of a sudden, your team's down there; you're down two to four. Casey's coming to bat. Also, you see tons of people getting up to leave. Fifth grade teacher Ben Nathanson has taken that idea of combining fun and learning and turned it into an art form. Are you mad? Yeah, well, show me mad. Your best friend's getting up to leave; you paid for him to get in.

His poetry class is studying "Casey at the Bat," complete with sound effects. And I want you guys to make noises if you're at a baseball game. And when I drop my hand, I want you to tell me, show me what you think a death-like silence is, okay? Are you ready? And that's what happened. That's unbelievable.

I believe in what I do. Imagine, I come every morning saying, who am

 I going to change? How am I going to make his life better? How am I going to get him to learn something? They challenge me, and I accept that challenge. And all I ask for is effort. But if I'm not going to come with the enthusiasm, they're not going to learn anything. So it's my personality to take the humor and look for the humor in somebody and draw it out of them. Did you change your expression? Ben Nathanson doesn't simply have his students read out loud. They take this poem, search for the emotional elements, then act them out. Casey's coming. Look here he comes. Look at him. Yeah, yeah, that's expression, folks, not...

I want them to get excited about reading. Words are fun; words can do things to you if you use a little drama with it, if you use expressions and make them feel it. When they go to their desk, you got them started; you got a fire built in them. Casey's coming to bat; you've got a chance to win. All right, DJ's jumping up for joy; they're going, counting his money. Man, I got money on this game. And you challenge them. I'm challenging those kids to find a different way to do it right. So you've got to believe what you're pretending to see. What are you pretending to see? Casey's coming. Show me, Casey.

And then they pick apart the story slowly, and they see it differently. That's exciting for a child versus, here's the book, read it, answer the questions. That's not exciting.

Back at the hockey game, the students have a penalty shot. The contest for the Kuba Saw Cup is tied: teachers one, students one. Okay, how was this story? How was her story different from the original fairy tale? This is another school-wide activity, something called shared reading. Each week, students gather in small groups to read stories they've written.

The opportunity for children to engage in quality writing skills, but also there is a social component to it as well. The children then have the opportunity to meet with all grade levels in the school and share their writing. And so it goes back to the idea that academics and behavior can fit very well together.

Thorncliffe's good behavior goes hand in hand with academic success. And if a student isn't succeeding, there's plenty of extra support available. You go ahead and try and distract them.

Pat Castoros teaches a skills class. If students are disruptive or have trouble paying attention, they come here for one period to learn some basic interpersonal skills. Keep trying, be real persistent, Corey. Corey keeps poking his pencil and distracting me. Oh no, so how does that make you feel? Sad, sad and mad. Well, you've done a good job by telling us what's going on instead of just poking them back, right?

The skills class has a single purpose: to get these students ready and able to learn. Back of the guy pushing them back and forth. If they are able to have some self-control and manage their behavior within a classroom and listen to the teacher, then they're picking up on the lessons that the teacher's trying to teach them. That in turn helps their academic skills, which in turn, later in life, can benefit them in the workplace.

Thorncliffe teachers talk often about preparing their students for later life. So when Ted Davis has his fourth graders read a story, they do it in small groups. The focus isn't on just reading, but also on cooperation.

We have to cooperate with each other; we're a team; we work together; we're a group. What we're trying to do at this school is prepare them for once they get out of this school and no matter where they go in their life, they're going to have to cooperate with somebody, whether it's standing in a bread line or applying for a job. And the idea of teamwork is very, very important.

Another effort to make learning as engaging as possible, when it's time to work on math, Ted Davis breaks his class into teams. My class was weak in the multiplication facts, so I try and bring in these little games like the multiplication relay where they go up to the board; they have fun with it, make it a bit more incentive.

Meanwhile, the student hockey team works together to break the tie. With time running out, the students hold a two-to-one lead over the teachers.

There's a concentrated effort at Thorncliffe to improve self-esteem, an effort that begins on the first day of school. Joanne Moulter wants her first graders to experience success very early, even a simple success like solving a math problem.

Before you can teach, you have to have the kids feeling good about themselves. I think they go in hand; I don't think you can have the academics just academic. The kids are not going to be motivated to learn; they have to feel good about themselves. And when they do that, then you can teach them; then they will learn and achieve.

Self-esteem is also enhanced by support from other students. In Pauline Drinkwater's class, second graders read a story and make analogies between themselves and the main character. The rest of the class is then invited to make comments. The emphasis is on the positive.

The focus on self-esteem continues in the later grades. Ben Nathanson is teaching his fifth graders how to solve math problems. Right now, we know that half a dozen cookies cost 270. But most of the knowledge is produced by the students themselves.

Is he missing one more thing in that problem? Is there anything else in the mathematical problem you need to draw? Give them the sheet so you can read

 the problem quickly. Are you sure, Ray, that's the only thing you need? Hands, if you know there's something missing, just one person. Holly's not sure; she's halfway down. Now she's positive; she's guessing. Mr. Robinson looks confident; Mr. Paquette, you don't look so confident. Are you happy with it? Yeah. Ask your partner if he's happy with it. Yeah, okay, so elevator down. You're not happy with it. No, step one. We're still step one. Mr. Robinson, it doesn't tell you that you have to find out what the price of one cookie is. So should you not draw an arrow, draw one, and say cost, question mark, like what are we supposed to find out as far as you're concerned? We got the answer right here, two dollars, 70 cents, half a dozen. What a waste of time, because all this information is not leading back to here. So could you have drawn that? Yeah, how many took that extra step? How many did not? How many have this? If you took it one more step, I vow to you, congratulations. You took it an extra step; you thought this much more. Don't get us wrong here with our philosophy about self-esteem. Just leave it like that. Academics is important to us. I think if you check our records, I think you'll see the academic standing of a lot of these kids, and I think we can stand tall with our heads up, saying we're not just touchy-feely. We're not into that. We're into the human being and doing the best that they possibly can do, and academics is very important to us at this school, but self-worth, self-esteem, starts it all off.

There's a major effort at Thorncliffe to connect students with the world outside of school. Donna Granger's class wants to get rid of graffiti in Edmonton. Graffiti isn't just what it says in the dictionary; graffiti is all about feelings.

These students don't just talk about graffiti; they try to do something about it. That includes making a phone call to City Hall.

The Thorncliffe Rocket Club is putting on a show. The rocket club was the idea of school counselor Pat Fizzell.

At Thorncliffe, school, we invest children in the daily act of living, in the daily act of making their world a little bit better.

Things are going a little bit better for the teachers; they score to even the hockey game at two goals apiece. This big match will be decided by a shootout.

I'm asking you to begin to read on your own and take some responsibility for this for yourself, okay? I don't want you to be dependent on me. These are well within your reading levels, and I'm expecting you to work on them to get started.

The role of the coach, like a hockey coach, is to bring the highest quality of players to play on that team. And the Thorncliffe team has players that are of a quality that could go out on any one given moment and win the Stanley Cup of school.