\chapter{Every Child Can Succeed 7}

\begin{figure}[H]
    \centering
    \includegraphics[width=\textwidth/2]{./Games/EveryChildCanSucceed/Images/EveryChildCanSucceed7CD.png}
    \caption{Every Child Can Succeed 7 CD}
\end{figure}

The final of the Every Child Can Succeed games published and released by The Lightspan Partnership for the PlayStation 1.

Every Child Can Succeed 7 features three video programs:

\begin{itemize}
    \item L. A. Stories: Bennet-Kew and Euclid Avenue
    \item Thorncliffe Community School: Student-Centered Learning in Edmonton
\end{itemize}

\clearpage
\newpage

\section{L. A. Stories: Bennet-Kew and Euclid Avenue}

\subsection{Audio Summary}

L. A. Stories: Bennett-Kew and Euclid Avenue investigates how two schools in the Los Angeles area are achieving success with increasing bi-lingualism and the urban problems of gang violence and crime.

\subsection{Transcription}

"When I first came here, this was in '74, and the first test scores, state test scores came out, our third graders were reading at a three percentile. It took us four years to get them over the 50 percentile."

"No one wanted to come to this school, either as a teacher nor as an administrator, and I certainly can tell you I didn't want to be here. I wouldn't trade this school for a minute now, not for a minute."

This program tells the story of two schools in Greater Los Angeles. Our first stop is the Bennett Q Elementary School in Inglewood, California, just outside the Los Angeles city limits. There is a solid middle-class core in the region served by Bennett Q, but the school also serves a large poor transient population. Nearly 60 percent of the students who attend Bennett Q are eligible for the subsidized lunch program.

Despite the demographic odds, Bennett Q's achievement scores have improved considerably. For example, third-grade math scores have moved from the lowest to the highest quartile in the state. This success has taken place under the leadership of Principal Nancy Ichinaga. The keys: a belief that every child can succeed, and a system to make it happen.

"People have a notion that kids from, minority kids in particular, from urban settings and in urban public schools, really are not or cannot do very well. And I really have to refute that point of view. My kids do as well or better than kids just about anywhere. First of all, we believe that the school must be a safe place for all kids. We also expect the kids to come to school believing that it's their time to be serious students. And we do believe that every child can learn to read, to write, and to be a successful student. We have worked out a system where everybody is taught specific things that make them a successful student, and it's what every teacher does."

The system begins in kindergarten where Teacher Charlotte Watanabe is laying the foundations for later success in reading and math.

"Kindergarten is a very important grade. If we don't make them feel good about themselves by making them succeed in whatever they do in the classroom, you know, they're going to start feeling like failures from the very beginning, so we want them to succeed here, we want them to feel good about what they do, so whatever tasks they have, we give them a lot of praise at it. What we do here in kindergarten with all these manipulatives is to build a good foundation of the understanding of the concepts we're building, so that when they go up to first grade, second grade, to the upper grades, they'll have a strong understanding of what exactly they're doing in those grades when they are working with computations."

The building block approach continues through the primary grades so that by the time students reach Howard Rothenberg's fifth grade, they have a solid grounding in the basics that they need to achieve success in math. Today's lesson deals with fractions.

"Obviously, if you don't have the foundations from first, second, third grade, you can't begin fourth grade where you need to start. And you have to go back and remediate, and you're losing time and the children. And that's what happens in most inner-city schools, the children are years behind because every year has to go back and repeat what they did not learn previously. Well, we do not have that problem here for the most part. We have a very special curriculum at each grade level, and every teacher knows what he or she is responsible to teach so that the following year's teacher can take out [that that spot], and the kids, you know, are pretty much at grade level."

"Whenever new students enroll at our school, I do a quick check. I do it because I want to know where they're at academically. I also want to get to know them, and I want to meet the parents and do a little work on letting them know what we're all about. One of the very important things about our school is that our kids are held accountable for their learning. In other words, we have expectations by grade level for every kid. And the kids of the parents know from the very, very beginning that if they are in kindergarten, these are the things they have to master before the year is over. And if they don't, they probably will not go on to the next grade."

Bennett Q's curriculum focuses on math and reading. Reading presents a special ,challenge since nearly 40 percent of the students are Hispanic, many with English as their second language. Special classes like this one are available to help Spanish-speaking children improve their English comprehension.

"I do believe that if you want to learn English, you have to learn English. I don't believe that you're going to learn English by learning Spanish. Parents, when they came to our school, expected the kids to be taught in English. They didn't expect us to be able to do it in Spanish."

To teach reading, the Bennett Q curriculum relies heavily on phonics. It's a curriculum that the faculty strongly supports since they helped to write it.

"What we do is make sure everyone learns. When Nancy came in, and things were sort of haphazard, the first thing that she did was to start rewriting curriculum, and we've all done a lot of that together as teachers, so that everyone can be successful.

By the time they reach third grade, Bennett Q students are reading and discussing books like 'Charlotte's Web' under the guidance of Teacher Mary Lou Reingold.

"It's exciting to see these children being able to read and to be able to digest a story like 'Charlotte's Web' and get the meaning out of it. The school's special because it's run so well. Nancy, the principal, is ahead of her time. She knows where we all need to be moving, so she keeps us on track."

"I monitor what happens in the classrooms by periodic tests which the teachers all give, and every teacher in every grade level gives the same test."

By carefully monitoring the performance of each student, Nancy Ichinaga ensures consistency across the curriculum, and helps teachers identify and correct any weaknesses in their approach.

Here, Nancy and Cheryl Draper brainstorm with new fourth-grade teachers to find better ways to assess the writing skills of their students.

"These assessments really help us stay on focus. It helps particularly the new teachers because you know they really have no idea what it is they're supposed to teach, and this way they learn this is what we need to cover. And so they're they're more successful."

"Nancy shows her support for her teachers in many ways, in small ways and in large ways. Whenever someone's having a problem, and Nancy needs to deal with it, I feel she approaches it as a group, as a team effort. You know, we can, we need to work on this. You know, what can we do? How can I help you to change? Whatever it is she feels needs to be a little better. And whether it's an individual, or as a grade level, we usually kind of brainstorm and come up with some ideas on things that can be changed. If you get a good principal who knows what she's talking about, and know curriculum and knows her people, then anyone can do it. And no one should take excuses, we don't have excuses for why children can't learn."

"Our main focus is to make our kids successful. We don't teach because we need a job, we teach primarily because school is for kids."

"We have a very nurturing environment. We think that's very important. We have a place where kids can work hard, where they can feel good about themselves, where they can be successful."

"I think that a child who doesn't feel good about himself cannot feel good about the world. So, we try to make every one of our kids successful, at least academically. So, whatever may be wrong with another part of his life, at least this part is good."

Our second Los Angeles school is located in a predominantly Mexican-American section of the city. The school is Euclid Avenue Elementary. There are more than 1200 students attending Euclid Avenue. Ninety-three percent of them are Hispanic. Over 90 percent are eligible for the subsidized lunch program, and the majority are from single-parent families.

If you want to keep up with what's happening at Euclid Avenue School in the heart of East Los Angeles, just follow these Reeboks. But be prepared to hurry, Esther McShane and her school are on the move.

"I do not spend my time during the day during paperwork. The action is in the classroom. The action is in the yard. The action's in the eating area. I have high needs to be with my students and my staff."

Until Esther Castrita McShane took over as principal, Euclid was a very low-achieving school, with little expectation of future improvement. Today, Esther and her staff are doing what it takes to achieve success at Euclid Avenue Elementary School. When she came to Euclid four years ago, Esther's first step was to establish a safe environment for the children. She fenced in the school grounds and established strict rules of conduct throughout the school.

Sometimes enforcing the rules requires a one-on-one talk with a student. Sometimes it requires a session with both the students and the parents.

"I believe that all of us are responsible for our choices, and that I have a role and to help people understand the choice that they make in their behavior."

Maintaining order also means maintaining the appearance of order, by trying to stay one step ahead of the graffiti that plagues the area, for example. Every weekend, local gangs and rival gangs from other neighborhoods leave their mark. The graffiti is a constant reminder of the obstacles that Euclid Avenue students face every day.

Many of the children come to Euclid speaking no English at all. To cope with the differences in language, Euclid places a strong emphasis on bilingual education.

"The sole purpose of bilingual education is to acquire English, to be literate in English, to converse, and to compete in the English-speaking world. In order for someone to learn a second language, there has to be a certain level of literacy so that they can transfer understandings, concepts to another language. The bilingual program here is very successful. It's successful because it's a philosophy that has been embraced by all of us."

The faculty wasn't always united on issues like bilingual education. In fact, one of the first obstacles that Esther McShane had to overcome on the road to excellence was a divided faculty.

"When Esther first came to this school, we were in a state of chaos. Teachers were battling with each other, and just totally upset with administration. And of course, the students were losing out. Test scores were down. Attitudes were pretty bad, and when Esther first came in, she decided that we needed to bring in the Achievement Council in order to get some necessary changes done."

"There was racial tension. There was a tension of teachers that were not talking to each other in the halls. There was no focus on looking at kids and achievement. There were the teachers that were with the bilingual program versus the non-bilingual teachers. All those things were going on. It was a fractured school, very fragmented."

Emilia Tenoco is head of the local teachers union and a classroom teacher at Euclid. She was here when Esther became principal.

"Ms. McShane brought some outside mediators, which were part of this Achievement Council, to come to see if maybe we can work this out. And what the point that we got to was not really having to change over our beliefs but at least respect the other person. They began to experience that it is okay to be different. It is okay to have our own beliefs. Then they were able to go beyond the cultural, racial, and language barriers, and they began to communicate about educating children."

"We are only here for one reason, and that is to educate children. And I need to provide the climate, the environment where learning can take place."

In order to provide that environment, Esther has focused her attention on the classroom. Her strategy is to produce successful students by fostering excellent teaching.

"What I look for in a classroom is to see time on task. Why is it that the child is engaged in that activity? Is that activity challenging? We need to begin to concentrate on what we need to do to upgrade the educational program in every classroom."

Esther follows up her observation of teacher Linda Murakami with a formal review.

"I think what she's really good at is observing teachers and seeing their strengths, and trying to draw from their strengths and getting us to share with each other."

Esther encourages teachers to share by providing them with time to observe each other in the classroom. Afterwards, she will meet with the teachers to discuss their observations.

"I think to be an effective administrator, you have to be a teacher. This is a large classroom. The entire school is a large classroom. I have teachers, I have children, I have community, I have parents."

One way to involve the community in Esther's extended classroom is to conduct educational programs for parents and their children. This is a session on problem solving skills. Programs like these help lay the foundations for success by integrating parents into the life of the school. Attaining success also means preventing failure, and that means helping children overcome obstacles to learning. This student needs help in settings and meeting daily goals. Individual attention is an important part of Euclid's program. Every week, a committe of teachers and specialists meet to discuss children who are experiencing difficulty in the classroom.

Vice Principal Robert Cordova chairs these meetings.

Esther has established a safe environment that is conducive to learning. She has united the faculty, encouraged professional development, and involved parents in the life of the school. Her next challenge is to improve student achievement, the most important job of all. To help improve achievement, Esther has established a bilingual gifted and talented program. Her plan is that the teaching strategies developed in the program will soon be used throughout the school. Teacher Geraldine Allen works with one of these classes.

"I want my children to believe in themselves and to believe that they can accomplish anything they choose to, and I realize that my job is to teach them strategies for accomplishing these things."

Allen breaks the class into cooperative groups. One will produce a video, another a newspaper front page, and others will rewrite famous fairy tales. The purpose: to explore different points of view.

"We use a lot of cooperative learning. I depend on the children teaching themselves, and with me as the guide. We're moving. It's a slow move, but we're moving."

With the help of teachers like Geraldine Allen, Esther McShane has put Euclid School on the path to success. But there is still a distance to go.

"There won't be a peace of mind until we begin to really upgrade children's achievement in this school, which means every single staff member needs to be on top of every child, that we need to know the child, and we need to be sure that we are providing the services that will allow the child to grow to their optimum level every year."

\section{Thorncliffe Community School: Student-Centered Learning in Edmonton}

\subsection{Audio Summary}

Thorncliffe Community School: Student-Centered Learning in Edmonton observes how the principal and staff of an urban school have created a climate of order and calm in which all the students have become positive and enthusiastic students.

\subsection{Transcription}

The students at Thorncliffe Community School say they're building their future. It's a far cry from the past when Thorncliff was one of the most dangerous schools in Edmonton, Alberta.

"Kids were just out of control at school; there was no consequence that would fit their action."

Enter Lou Yaniw, who was named principal four years ago. He brought in some new teachers, an entirely new attitude, and a new approach. This was to be a student-centered school.

"In a student-centered school, the belief, the generic belief, is that every child can succeed."

And they are succeeding. Discipline problems are down, academic achievement is up, and school spirit is off the charts. Thorncliffe Community School has taken off.

It's a big day at Thorncliffe. This is a showdown between two floor hockey teams: the students against the teachers. At stake, the Kuba Saw Cup. As usual, the head cheerleader is Thorncliff Principal Lou Yaniw. Right now, the 300 Thorncliff students are finishing a series of intensive assessment exams, which is exactly why this game is being played.

"If we can introduce something that children can enjoy and loosen up in regards to, and have a little bit of fun with, then it just makes coming to school and going through the assessment period just a little easier for them. And in the gymnasium during the time of the cup, you can just see it on the faces of the students how much they're enjoying this."

The students are enjoying it, so are the teachers, especially when they score the first goal. It's the teachers one to nothing. More about the game later.

Thorncliffe is a school that's undergoing a radical transformation, one that began when Lou Yaniw arrived four years ago. At the time, Thorncliff was one of the worst schools in Edmonton. The school sits on the west side of the city, in the shadow of the enormous West Edmonton Mall. By U.S standards, the surrounding neighborhood might not seem poverty-ridden, but three-quarters of Thorncliffe students are considered low-income, and nearly half come from single-parent families. This is a troubled neighborhood with one of the highest crime rates in Edmonton, and many of those troubles were carried over into Thorncliffe.

"It was a school that in many ways was very dysfunctional. Teachers were suspending students from the classroom to the tune of over 300 a month, out of the classroom for being disruptive and out of control."

While some students were unruly, many others were frightened. James Ross is a sixth grader at Thorncliffe. He remembers how things were.

"Before on the playgrounds, there was a lot of violence and fighting all the time, and you couldn't go outside without being afraid."

Lou Yaniw's first priority was to eliminate the fear. He did away with corporal punishment but set up strict rules for behavior. Any student breaking those rules is disciplined quickly and fairly.

Ted Davis presides over daily detention. Children who have broken a rule spend their recess time here.

"You hit somebody, you're going to have a recess detention. You hit somebody outside the school, there's going to be a consequence for that. If you hit somebody when you're older and it's in a mall or in a dark shopping center parking lot, there's going to be a consequence for that action. And that's what we're trying to get these kids to realize. There's a consequence for all their actions."

Discipline is also maintained by positive reinforcement. This assembly is to award monthly Diamond of Distinction certificates to students who exhibit exemplary behavior. Any student who earns a certificate is eligible for a drawing to be held the last day of school. Another distinction at Thorncliffe is to be called to the principal's office. At this school, it usually means you did something right.

"Let me tell you what a celebration it is for the students when they come down to the office, and we have an opportunity to share the good news with one another, and then in each and every case, I take it upon myself, in the presence of the student to inform the parent or guardian of that child, whether they be at home or at work. And it is just a wonderful celebration."

Lou Yaniw also went about transforming the entire school atmosphere. Thorncliffe was to be a place where all children could succeed.

"As part of our belief statement, we believe that all students have three generic needs: one, a need for competency and achievement; two, a need for structure, safety, control, and discipline; and three, a need for positive personalized interaction between peers and adults. As a staff, then, if we collectively believe this, at our staff meetings, we make decisions as to programs, activities, and events that match that."

Positive interaction between children and adults. It's evident in the classroom and on the playground. Lou Yaniw does whatever it takes to make Thorncliffe as enjoyable as possible. His enthusiasm is obvious; it's also highly contagious. But discipline and an inviting atmosphere are only means to an end. The primary mission of Thorncliffe is education.

"If children enjoy coming to school, there's no question they will learn. If children do not enjoy coming to school and, in fact, hate being in class, regardless of how tight your curriculum is, they will learn less than they are capable of. So the teaching-learning process must be interesting; it also must be inviting and fun."

Fifth grade teacher Ben Nathanson has taken that idea of combining fun and learning, and turned it into an art form. His poetry class is studying "Casey at the Bat," complete with sound effects.

"I believe in what I do, and I come every morning saying, who am I going to change? How am I going to make his life better? How am I going to get him to learn something? They challenge me, and I accept that challenge. And all I ask for is effort. But if I'm not going to come with the enthusiasm, they're not going to learn anything. So it's my personality to take the humor and look for the humor in somebody and draw it out of them."

Ben Nathanson doesn't simply have his students read out loud. They take this poem, search for the emotional elements, then act them out.

"I want them to get excited about reading. Words are fun; words can do things to you. If you use a little drama with it, if you use expressions and make them feel it, when they go to their desk, you got them started; you got a fire built in them. And you challenge them. I'm challenging those kids to find a different way to do it. And then they pick apart the story slowly, and they see it differently. That's exciting for a child versus, here's the book, read it, answer the questions. That's not exciting."

Things are exciting back at the hockey game: the students have a penalty shot. The contest for the Kuba Saw Cup is tied: teachers one, students one.

This is another school-wide activity, something called shared reading. Each week, students gather in small groups to read stories they've written.

"Obviously it is a very academic directed program. The opportunity for children to engage in quality writing skills, but also there is a social component to it as well. The children then have the opportunity to meet with all grade levels in the school and share their writing. And so it goes back to the idea that academics and behavior can fit very well together."

At Thorncliffe, good behavior goes hand in hand with academic success. And if a student isn't succeeding, there's plenty of extra support available.

Pat Kostouros teaches a skills class. If students are disruptive or have trouble paying attention, they come here for one period to learn some basic interpersonal skills. The skills class has a single purpose: to get these students ready and able to learn.

"If they are able to have some self-control and manage their behavior within a classroom and listen to the teacher, then they're picking up on the lessons that the teacher's trying to teach them, that in turn helps their academic skills, which in turn, later in life, can benefit them in the workplace.

Thorncliffe teachers talk often about preparing their students for later life. So when Ted Davis has his fourth graders read a story, they do it in small groups. The focus isn't on just reading, but also on cooperation.

"You come into my classroom, we have to cooperate with each other; we're a team; we work together; we're a group. What we're trying to do at this school is prepare them for once they get out of this school and no matter where they go in their life, they're going to have to cooperate with somebody, whether it's standing in a bread line or applying for a job. And the idea of teamwork is very, very important."

This is another effort to make learning as engaging as possible. When it's time to work on math, Ted Davis breaks his class into teams.

"My class was weak in the multiplication facts, so I try and bring in these little games like the multiplication relay where they go up to the board; they have fun with it, make it a bit more incentive. Again the team effort, you work together."

Meanwhile, the student hockey team works together to break the tie. With time running out, the students hold a two-to-one lead over the teachers.

There's a concentrated effort at Thorncliffe to improve self-esteem, an effort that begins on the first day of school. Joanne Moulter wants her first graders to experience success very early, even a simple success like solving a math problem.

"Before you can teach, you have to have the kids feeling good about themselves. I think they go in hand; I don't think you can have the academics just academic. The kids are not going to be motivated to learn; they have to feel good about themselves, and when they do that, then you can teach them; then they will learn and achieve."

Self-esteem is also enhanced by support from other students. In Pauline Drinkwater's class, second graders read a story and make analogies between themselves and the main character. The rest of the class is then invited to make comments. The emphasis is on the positive.

The focus on self-esteem continues in the later grades. Ben Nathanson is teaching his fifth graders how to solve math problems. But most of the knowledge is produced by the students themselves.

"Don't get us wrong here with our philosophy about self-esteem. Academics is important to us, but I think if you check our records, I think you'll see the academic standing of a lot of these kids, and I think we can stand tall with our heads up, saying we're not just touchy-feely. We're not into that. We're into the human being and doing the best that they possibly can do, and academics is very important to us at this school, but self-worth, self-esteem, starts it all off."

There's a major effort at Thorncliffe to connect students with the world outside of school. Donna Granger's class wants to get rid of graffiti in Edmonton. These students don't just talk about graffiti; they try to do something about it. That includes making a phone call to City Hall.

This is another learning activity that takes place outside of the classroom; the Thorncliffe Rocket Club is putting on a show. The rocket club was the idea of school counselor Pat Fizzell.

"The idea that we put children in a little box called school, and we invest their childhood there, I don't believe has been overly successful. At Thorncliffe School, we invest children in the daily act of living, in the daily act of making their world a little bit better."

Things are going a little bit better for the teachers; they score to even the hockey game at two goals apiece. This big match will be decided by a shootout.

While the general belief at Thorncliffe is that all students can succeed, there is support for children who are having difficulty. Brian Ferguson teaches a group of students with emotional and behavioral problems. The idea is to eventually get these kids back into a regular classroom. In the meantime, they're taught the regular curriculum and offered as much attention and encouragement as possible. One more special class is made up of children who have fallen at least two years behind their classmates.

"I guess I would describe them as I have a family of 11 children. They range in age from 9 to 13."

Monica Ullman teaches the class. Again, the goal is to have these students catch up and move on.

"They come feeling very left out, some of them actually come with quite a chip on their shoulder saying things like 'I'm stupid, end of story.' They believe that and they would repeat that quite often at the beginning of the year. My philosophy as a teacher is, whether it's with special needs students or with regular students, is that all children can have successes if the situation is set up very carefully at the beginning so that those successes are very meaningful to them."

There's a continual effort at Thorncliffe to monitor all students, an effort that goes beyond the usual standardized tests. The school has developed its own assessment program in which each child is tested at the beginning of the school year and again at the end of the year.

"We are more interested in individual growth for students over the course of the year than comparing them to other learning groups either in the city or in the province. We are able to do that by identifying the change that takes place between their achievement level at the beginning of the year and at the end of the year, the classic pre and post-test assessment profile."

If testing shows a child is falling behind, teachers at Thorncliffe search for new and different strategies. All around the school there were reminders to the staff, they're expected to do anything it takes to reach every child.

"It's a team effort. I will not say to the child 'you did not succeed', it's 'we did not succeed' and that includes the parent. It's a triangle and the kid has to know it's a triangle and it's the school, the teacher, the parent and the child and we truly believe that here at Thorncliffe."

"If a child in the classroom is having difficulty learning, the teachers have committed themselves to working with that child certainly within the classroom but outside the classroom as well as networking with the parents to ensure that that child will succeed."

Parents have played a critical role in Thorncliffe's transformation, whether it's tutoring children or monitoring the playground.

Cheryl Rose is president of the parental advisory council, which has more than a hundred members. Once a month, the council runs a bingo game at the West Edmonton Mall. They raise enough money to pay for most of the school's field trips.

"We have a tendency to find out that budgets have been cut and the schools don't always have enough funding to do exactly what they want to do, and whatever they've decided to do which isn't saying that it isn't enough for the kids, but we just like to give that little extra so that, you know, the kids get the best that they can possibly get."

Back at the hockey game, the shootout is nearly over. The students score to go ahead by one. The teachers are down to their last chance. If they miss this shot the game is over. The students win the game and the famous kuba saw cup, at least until next year.

Lou Yaniw has accomplished his initial goal, he's changed the environment of an entire school and the attitude of an entire student body.

Thorncliffe community school has become a haven, a place to enjoy, and a place to learn.

"Once a school has the base foundation of strong solid supportive discipline, as well as a vibrant spirited inviting culture and climate, academics, hard, well fine-tuned academics, can be brought into a school with much more enthusiasm. With the base foundation in place, teachers are more empowered and enthusiastic, and find the energy to engage in curriculum development and redevelopment."

Lou Yaniw is given credit for turning around an entire school, but he says he's done just one thing, assemble an all-star team of teachers.

"The role of the coach, like a hockey coach, is to bring the highest quality of players to play on that team and the Thorncliffe team has players that are of a quality that could go out at any one given moment and win the Stanley Cup of School."