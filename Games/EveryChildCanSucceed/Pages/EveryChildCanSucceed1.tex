\chapter{Every Child Can Succeed 1}

\begin{figure}[H]
    \centering
    \includegraphics[width=\textwidth/2]{./Games/EveryChildCanSucceed/Images/EveryChildCanSucceed1CD.png}
    \caption{Every Child Can Succeed 1 CD}
\end{figure}

The first of the Every Child Can Succeed games published and released by The Lightspan Partnership for the PlayStation 1.

Every Child Can Succeed 1 features two video programs:

\begin{itemize}
    \item Promises to Keep: Achieving Success in Our Schools
    \item Using Every Child Can Succeed
\end{itemize}

\clearpage
\newpage

\section{Promises to Keep: Achieving Success in Our Schools}

\subsection{Audio Summary}

Promises To Keep: Achieving Success in Our Schools, demonstrates that schools can overcome discouraging obstacles and help students achieve excellence.

\subsection{Transcription}

Hello, I'm Roscoe Lee Browne, and this is Mrs. Allen's class at the Euclid Elementary School in Los Angeles.
I've come here today to talk with you about something vitally important to all of us: achieving success in our schools.

Success in most fields depends on a combination of talent, timing, hard work, and good luck.
It's something that no one can promise or guarantee.
But what we can promise, and what we have promised, is that through our system of public education, we will give all our children the kind of foundation that prepares them to succeed in life.

Are we doing it? Are we keeping the promise? Apparently not.
We've all heard how American students are performing poorly compared to students from other countries.
It seems that academic success is getting harder to come by for everyone.
But for poor children, especially for poor minority children, the possibility of getting a good basic education has become more and more of an impossibility.
And that says more about how we are educating our children than it does about the children themselves.

There's plenty of evidence to show that problems of poverty, race, and adverse social conditions can be overcome.
They can be overcome by good teaching, and by schools that provide the right direction at the right time.
Nothing can replace the feeling of achievement that comes with success in learning.
It provides a foundation of confidence that can make all the difference in a person's life.
Achievement in school and making a difference - that's what this project is all about.

It's called 'Every Child Can Succeed.'
Think about that: every child can succeed.
It's a statement of faith in the power of teaching, it's a call to action for our nation's schools, and it's a promise we are making to all our children.

These children are on their way to the future - their future and ours.
The schools we send them to are supposed to be expressions of our commitment to a new generation.
But are all of our children being well served?
If not, are we up to the challenge?

"There was a time when this nation wasn't called upon to educate all of its children.
We were called upon to educate the leaders, which were about 20 percent of our student population, and then we had to educate about 30 percent of our student population to assist that 20 percent, and the rest of the folks could go into factories or they could be throwaways.
But in this information age, all of that has changed.
This is a day where all children must learn, all children must be able to be fully employed, all of our children have to become thinking individuals because there are no rote memory kinds of jobs, there's no work for just hard working, earnest people who aren't able to think in a critical way.
Those days are gone forever."

There are many places in North America where principals, teachers, and children conduct the vital business of education under appalling conditions.
Amid the deep poverty of rural areas all across the continent, and in poor urban neighborhoods filled with noise and violence, where drug dealing and other crimes are part of everyday life.
What do we expect of children in these environments?

"We say that we believe that all children can learn, and I believe that almost every educator wants desperately to believe that and may sometimes even believe that they believe it.
And yet, they nurture doubts that they can produce the educational outcomes in situations that are difficult.
For example, we hear that if students are poor, that there is a correlation between poverty and school success.
we've heard that if students are bilingual, that this bilingualism will be a deficiency in the school setting and that that will account for low performance.

"We've heard that single parent families are a problem, there's no father in the home or no mother in the home, the child is doomed to failure because of the fact that parents cannot provide the kind of support in that environment.
If they live in neighborhoods that are culturally deprived, that are drug infested.
On and on and on.

And of course, we would like to see the reversal of every one of those variables, but not a single one of those accounts for the low performance of school children, either collectively or individually.
We know that it is possible doing things that schools do, to overcome the damaging and negative effects of these variables."

In his book 'Savage Inequalities,' Jonathan Kozal simply dismisses explanations that point to deficits in the child as reasons for failure.
According to Kozol, the problem lies not with children, but with the educational services that they receive.

Some schools reject the notion that poverty, minority status, and a single parent situation inevitably spell academic failure for a child.
Instead, they believe that all children can learn and that it's the school's job to teach them.
Louise Smith, former principal of Charles Rice School in Dallas, subscribes to that point of view:

"The perception of a school in south Dallas or in any inner city area in an urban school district is that inner city kids cannot learn, that they are deficient in some way, and I'm here to make sure that that is a myth, that it is not true, absolutely not true.
These kids can learn, they're all born with the brain and it's our job to enrich that brain, to nourish it, and to come out with an educated person."

Doris Brevard, principal of the Van School in Pittsburgh, agrees that it is the school's responsibility to educate the child:

"So often the African-American child has a defeatist attitude.
For so many years you heard constantly that the African-American child could not learn, and it was our aim here to prove that the African-American child could achieve just as well as any other child, and that we didn't have to use any special methods, and we don't have any hidden secrets, we're just using the curriculum that's provided by the board, but our secret is we meet the needs of our children."

"If our schools are to be everything that they can be, it is absolutely essential that we have a deep and abiding faith in the abilities of children, and in the abilities of educators, to stimulate children to the point of academic excellence."

Despite the myths, there are schools succeeding with all children.

How many positive examples do we need to prove that it is possible to attain academic success in difficult circumstances?
According to Ron Edmonds, a leader of the effective schools movement, the answer is just one.
If we can find one school operating in adverse conditions whose students are achieving at or above national norms, that should be evidence enough that all children can learn.

That's the premise of a project called 'Every Child Can Succeed.'
As part of that project, we tried to find an example of success among schools serving poor and minority students.
We found more than one.
We found successful schools, and schools on the road to success in Dallas, Texas and Seattle, Washington.
In Pittsburgh, Pennsylvania and Edmonton, Alberta, Canada - eight schools in all from Maine to California.

According to the conventional wisdom, the children in these schools ought to be failing.
They come from environments that exhibit a familiar litany of problems: poverty, drugs, crime-ridden streets, and unstable family situations.
And yet in the schools we visited, success was the rule, not the exception.

In our sample schools, scores on standardized achievement tests place the school well above the norm in math and reading, often in the top quartile.
While they are not the only criteria for judging success, standardized tests do provide an objective and comparable measure of academic achievement.
They can demonstrate a school's determination to become or to remain successful.

Van School in Pittsburgh, for example, under the leadership of Doris Brevard, has had a long history of success as measured by test results.
Over the last 18 years, her school has consistently ranked at the top of the Pittsburgh public school system.
But for Mrs. Brevard, high test scores are evidence of something more fundamental:

"A successful school is a school within which a child is totally educated and every child meets his or her potential.
Well of course you can get any test score you want, that's not a problem.
But I feel I make the statement often that the Van student can take any test and the score will be the same on that test as it is on the California Achievement Test, because we teach the curriculum.
Therefore the child will be able to pass successfully any test."

In every successful school we saw, teachers and principals express the same kind of confidence in their students and in themselves.

"We think we can make a difference.
We think the teachers can make a difference, and that we think that we can enable and create environments that help students to be confident and competant in their learning."

The 'Every Child Can Succeed' project has several components.
One component is: a series of six half-hour video programs depicting successful schools.
These programs allow the viewer to visit one or more elementary schools that are working successfully with students.

The purpose of these programs and their associated print is not to provide models for replication, but to stimulate viewers to reflect on their basic assumptions about schools, teaching and learning.

"By viewing the successful schools examples that we have brought to you, it is my belief that it will raise the level of hope and faith in the teaching process and will stimulate educators to think in even more expansive terms about the kinds of things that they can do to take responsibility for the academic achievement of their children."

What we discovered is what we suspected all along: successful schools have a lot in common with each other.
While each school is unique, there are some essential elements that cut across broundaries of race, geography and economic condition.

Another component of the 'Every Child Can Succeed' project is called the 'Essential Elements,' consisting of nine 20-minute video programs highlighting elements critical to the operation of successful schools.
The Essential Elements component is accompanied by print materials to aid in discussion and planning.
The purpose of this component is to help principals and teachers to develop action plans for implementing these essential elements in their own schools.

"I am confident that we have arrived at a set of essential elements that will assist educators to analyze their own existing practices and to plan for the future in very effective ways."

All these essential elements work together to create a successful school.
They vividly demonstrate what is possible even under the most difficult conditions.
Take a topic like staff development.
In the staff development program, you see how these schools support professional growth.
From informal sessions between principals and teachers, to formal in-service sessions at which teachers share techniques that have proven to be successful.
You'll see how principal Gary Tubbs in Seattle models good teaching by demonstrating it himself in the classroom.
And how Euclid School in Los Angeles is using teacher observation with follow-up discussions to improve teaching.

Another essential element is productive school climate and culture.
This program shows how successful schools consciously establish and maintain an atmosphere in which learning can take place.
It starts with a vision.

"All of the folks here share a common vision.
They have a very similar philosophy, and that's every one of them will teach and every child in this building is going to learn."

"The dream is the vision that all children will be academically successful.
That, of course, is a vision that needs to be constantly addressed and revisited, and be kept alive in our heads, but also brought down to practical down-to-earth terms on how does that look in the classroom?
How does that turn into instruction?
How does that turn into curriculum?
How does we make learning meaningful so that it actually gets children a step closer to that vision."

The climate and culture program goes on to show how the vision is realized in successful schools.
When Esther McShane became principal at Euclid Avenue Elementary in Los Angeles, it was the challenge of her life.
The school's environment was the worst she had seen.

"Professionally or academically, it was a school that was almost a joke.
It was also a very filthy school.
There was no pride.
You know, when you look at a situation where there's no pride, you need to start there."

McShane began to implement the school's mission for success by creating a more supportive environment for teaching and learning.
Her actions showed the staff and students that their new principal cared about them.

"And if somebody shows you that they truly care, then people begin to feel they're important.
I'm important.
I can live up to whatever she thinks I'm capable of doing."

Of course, a clean environment is only the beginning.
At many successful schools we visited, the disciplined policy is also tied to the school's mission of academic success.

"Behavior comes first.
In order for kids to learn, they must know how to behave in school.
They must know how to come in school, sit, listen to the teacher in order to learn.
No child can learn if they're talking when the teacher is talking.
They miss out on too much."

At the successful schools we visited, the school's mission is implemented through a carefully designed instructional program.
Nancy Ishinaga sees it as a three-step process.

"Number one is that we do believe that every child can learn to read, to write, and to be a successful student, and this belief is throughout the school.
Everybody believes that, we just have to figure out how to do it.
And how to do it is the next thing."

To implement her mission, Nancy Ishinaga personally tests the academic skills of each new student, and she meets with the child's parents.
Then the students enter a sequenced and rigorous academic program.
Nancy Ishinaga visits each classroom periodically and meets with teachers to assess how the students are doing.
This is part two of her process.

"And maybe the third thing is that we try to build an environment in which everybody feels that they are important, and the fact that what everybody does contributes to the whole school, so that any school which wants to succeed can do this.
It's not a matter of personalities, it's a matter of having a system by which we educate our kids."

All the components of the 'Every Child Can Succeed' project point to one very simple conclusion: teaching children in at-risk situations isn't easy, but it isn't mysterious either.
It's being done successfully in a number of schools across North America by principals and teachers who are committed to educating children, and who are working together to achieve that goal.

"Our problem is really how to disseminate the information about people who have already solved problems.
Some educators believe that the achievement of children is due primarily to internal factors, or to their social, cultural and economic background.
In fact, we know that the power of teaching is responsible for outstanding student success, and we believe that this series of tapes demonstrates that clearly."


We know that the dream of success is possible, and we know that there are essential elements that can give it shape and substance.
So what's next?

The truth is that successful schools are all too often fighting against extremely difficult odds.
Most of the schools we visited achieved their success without any additional funding; funds tended to follow the success rather than preceded, let alone prompted it.
Nonetheless, there are things that a school district can do to encourage success.

The first thing is to believe that every child can succeed.
In this respect, the district is no different from the schools themselves.
A firm belief in children and in the power of teaching is a prerequisite to successful schools.
The district must provide the resources to support change and establish an environment that fosters it.
There are no shortcuts.

Every Child Can Succeed is a project designed to encourage the effective practices of some schools to serve as models for success in all schools.
The materials we have developed for this project are intended to supplement other efforts being made to improve schools for all children.
Mayor Norman Rice of Seattle sums up the importance of such efforts for all of us:

"I think that if an education system doesn't speak to all of our citizens, then it is not a good education system.
And I think what we believe in very strongly is every child can learn, and if we put them in the right environment and give them the right support around them, those children can learn as well as anybody, and that's what we're out to prove and that's what we're proving."

The proof is right before our eyes.
It's in every classroom of every successful school.
It's in the style of principals who are determined to make their schools work
It's in the words and gestures of every teacher who is determined that their students will learn.
And it's in the faces of the children themselves; the children who succeed despite fierce odds against them.
By understanding what works for them, we will be better equipped to make sound decisions for ourselves and the children for whom we are responsible.

Is it easy?
No, it isn't easy.
It takes constant effort to achieve and to maintain success.
It takes dedication, leadership, vision, and determination.
And it takes time - in some cases as long as five years or more, to turn a school around academically.

But it's an effort we have to make.
Effective education for all our children is a promise that we must keep.
What's at stake is something too valuable to risk: our future.
For the sake of that future, we need to resolve the educational problems this country faces.
The successful schools presented in this series prove that the resolution is within our grasp.
The rest is up to us.

\section{Using Every Child Can Succeed}

\subsection{Audio Summary}

Using Every Child Can Succeed describes the products components, and demonstrates ways of incorporating the resources into school improvement workshops.

\subsection{Transcription}

We want to tell you some stories - stories about schools that are thriving even though they confront some of the worst problems in education today.
How do they do it, and what can we learn from them that will help improve other schools?

Maybe you want to improve your school and you'd like some help.
That's why we produced Every Child Can Succeed.
It's a package of video and print materials designed for administrators, parents, teachers and others to use as part of an effort to improve schools.

In part one of this program, we'll examine the materials in the package.
And in part two, we'll demonstrate how you can use them to help create positive change in your school.

Every Child Can Succeed has three components: the demonstration component, the successful schools component, and the essential elements component.

The demonstration component is for school trustees, administrators, teachers, and parent and community groups.
This component includes two video programs.
The first is 'Promises to Keep: Achieving Success in Our Schools.'
The other video program in the demonstration component is 'Using Every Child Can Succeed,' which you're watching now.

Besides the video programs, the demonstration component also contains a guide for educational policymakers, including advice on how to support change at the local level.
The purpose of the demonstration component is to encourage policymakers to adopt plans and commit resources to meet the needs of all students.

The second component in Every Child Can Succeed is the successful schools component.
It's for administrators, teachers, community members, and parents.
The successful schools component features six 30-minute video programs.
Wouldn't it be great if your entire staff could go on a field trip to eight diverse and successful elementary schools?
Well, here's the next best thing, we'll bring the schools to you.

The programs are: The Van School: A History of Success in Pittsburgh, Charles Rice Learning Center: An Oasis in Dallas, LA Stories: Bennett, Queue and Euclid Avenue, New Sun Cook School: Achievement in the Woods of Maine, Thorncliffe Community School: Student-Centered Learning in Edmonton, and Northwest Passages: Alki and Beacon Hill in Seattle.

Along with the six video programs, the successful schools component includes a facilitator's guide.
The guide has suggested workshop agendas, handouts, background details on the featured schools and tips on how to use the videos to stimulate discussion.
The purpose of this component is to encourage viewers to consider and maybe even reconsider their fundamental assumptions about schools and schooling.

"If we don't believe that every child can succeed, if we don't have that attitude, all these resources and wonderful computers and so forth aren't going to do us any good because we we will use those resources for certain children and not for other children.
So I think it's an attitude shift."

The third component in 'Every Child Can Succeed' is the essential elements component.
It's primarily for principals and teachers.
This component has nine 20-minute video programs.
The videos explore elements that research has identified as essential for school improvement.
The essential elements are: instructional leadership, school leadership, high expectations, parent involvement, productive climate and culture, monitoring student progress, learning essential skills, effective instructional strategies, and staff development.

Along with the nine video programs, the essential elements component contains a facilitator's guide with workshop agendas, handouts, and suggestions on planning for change in local schools.
The purpose of this component is to help principals and teachers develop action plans for implementing the essential elements in their own schools.

The package also includes a book of readings.
Inside there's a large collection of background readings for use with both the successful schools and essential elements components.

There is no one correct way to use the Every Child Can Succeed materials.
The components have several different audiences and purposes, but the primary purpose of the series is to support improvement and reform at the local school level.
Here's a suggested format that uses all of the materials effectively:

At the first session, show the 'Promises to Keep' video program from the demonstration component.
After you've watched the video, discuss it, then decide whether your staff wants to see more materials.
If they do, schedule another session.

To begin the second session, conduct the focusing activity in the facilitator's guide, which asks 'Can every child succeed in school?'

After the discussion, move into the successful schools component.
Next, show one of the successful school videos.
There are suggestions for how to view and discuss the video in the facilitator's guide.
After watching the program, focus the discussion on the school in the video.

In the third session, you might watch another successful schools video.
After the video, follow the same discussion format as before.

At the fourth session, compare the schools you've seen.
Use the suggestions and handouts in the facilitator's guide to help focus the group discussions.

By the fifth session, you should be ready to set priorities and decide what essential elements to examine in detail.
Show the related video program.

After you've watched the program, use the tips in the facilitator's guide to generate discussion and to begin planning for school improvement.
Of course besides covering the suggestions in the guide, you're always free to add your own techniques and insights on school improvement.

At additional meetings, you can focus on other essential elements.
Always remember, the essential elements work together at successful schools.
Therefore, you might want to break into groups that will plan how to implement the various elements in your school.
Again, use the videos and print to guide your discussion and planning.

Do you believe your school can do better?
The educators and parents in these schools believe it.
They are driven to create ongoing programs and strategies that equal success for children.
Our videos and print will show you how they did it and help your school plan ongoing efforts to do the same.
We supply the materials, you supply the energy, the commitment and the persistence to make them work, because every child can succeed and will with your support.